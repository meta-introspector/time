\documentclass{article}

\usepackage{microtype}
\usepackage{graphicx}
\usepackage{booktabs} %

\usepackage[breaklinks]{hyperref}

\newcommand{\theHalgorithm}{\arabic{algorithm}}

\usepackage[accepted]{icml2023}


\usepackage{amsmath}
\usepackage{amssymb}
\usepackage{mathtools}
\usepackage{amsthm}

\usepackage{url}
\usepackage[utf8]{inputenc} %
\usepackage[T1]{fontenc}    %
\usepackage[breaklinks]{hyperref}      %
\usepackage{url}            %
\def\UrlBreaks{\do\/\do-}
\usepackage{booktabs}       %
\usepackage{amsfonts}       %
\usepackage{amsmath}
\usepackage{amssymb}
\usepackage{mathtools}
\usepackage{amsthm}
\usepackage{nicefrac}       %
\usepackage{microtype}      %
\usepackage{xcolor}         %
\usepackage{wrapfig}
\usepackage{graphicx}
\usepackage{caption}
\usepackage{multirow,makecell}

\usepackage{xspace}
\usepackage{subcaption}
\newcommand{\algname}[1]{{\sc #1}\xspace}
\usepackage{algorithm}
\usepackage{algorithmic}

\newcommand{\newstuff}[1]{{\color{blue}#1}}

\usepackage[capitalize,noabbrev]{cleveref}

\theoremstyle{plain}
\newtheorem{theorem}{Theorem}[section]
\newtheorem{proposition}[theorem]{Proposition}
\newtheorem{lemma}[theorem]{Lemma}
\newtheorem{corollary}[theorem]{Corollary}
\theoremstyle{definition}
\newtheorem{definition}[theorem]{Definition}
\newtheorem{assumption}[theorem]{Assumption}
\theoremstyle{remark}
\newtheorem{remark}[theorem]{Remark}

\usepackage[textsize=tiny]{todonotes}


\icmltitlerunning{SWARM Parallelism: Training Large Models Can Be Surprisingly Communication-Efficient}

\begin{document}

\twocolumn[
\icmltitle{SWARM Parallelism: Training Large Models\\ Can Be Surprisingly Communication-Efficient}



\icmlsetsymbol{equal}{*}

\begin{icmlauthorlist}
\icmlauthor{Max Ryabinin}{equal,hse,yandex}
\icmlauthor{Tim Dettmers}{equal,uw}
\icmlauthor{Michael Diskin}{yandex,hse}
\icmlauthor{Alexander Borzunov}{hse,yandex}
\end{icmlauthorlist}

\icmlaffiliation{hse}{HSE University}
\icmlaffiliation{yandex}{Yandex}
\icmlaffiliation{uw}{University of Washington}

\icmlcorrespondingauthor{Max Ryabinin}{mryabinin0@gmail.com}

\icmlkeywords{Machine Learning, ICML}

\vskip 0.3in
]



\printAffiliationsAndNotice{\icmlEqualContribution} %

\begin{abstract}




Many deep learning applications benefit from using large models with billions of parameters. Training these models is notoriously expensive due to the need for specialized HPC clusters. In this work, we consider alternative setups for training large models: using cheap ``preemptible'' instances or pooling existing resources from multiple regions. We analyze the performance of existing model-parallel algorithms in these conditions and find configurations where \textit{training larger models becomes less communication-intensive}.
Based on these findings, we propose SWARM parallelism\footnote{SWARM parallelism is a backronym for Stochastically Wired Adaptively Rebalanced Model Parallelism.}, a model-parallel training algorithm designed for poorly connected, heterogeneous and unreliable devices. SWARM creates temporary randomized pipelines between nodes that are rebalanced in case of failure. 
We empirically validate our findings and compare SWARM parallelism with existing large-scale training approaches.
Finally, we combine our insights with compression strategies to train a large Transformer language model with 1B shared parameters (${\approx}13$B before sharing) on preemptible T4 GPUs with less than 200Mb/s network.





 
\end{abstract}

\vspace{-12pt}
\section{Introduction}\label{sect:intro}
\vspace{-4pt}

Our investigation begins with a thought experiment. Imagine a deep neural network with capacity 1000 times greater than today's most powerful architectures: for example, a language model trained on all digitally available texts or a generative model for all images ever uploaded to the Internet. How can we train such a model?

\vspace{-1.5pt}

Viewed from a historical perspective, the 1000-fold increase in capacity is not unrealistic. Over the past decade, the deep learning community has made remarkable progress by training large models on abundant data, and the scale of those models keeps growing. Since the advent of the ImageNet challenge \cite{imagenet_cvpr09} with 1.3M labeled images, the typical size of convolutional neural networks increased from a few megabytes to hundreds of megabytes \cite{alexnet, resnet, huang2019gpipe}. Recent studies report even larger models for datasets with hundreds of millions of images \cite{kolesnikovlarge, jft300data}.

\vspace{-1.5pt}

Another trend from natural language processing is to train large Transformer-like language models~\cite{bert, roberta, kaplan2020scaling}. The data for this task is nearly unlimited, allowing researchers to train models with tens or even hundreds of gigabytes of parameters~\cite{brown2020language,shoeybi2019megatron,zellers2019defending,tnlg}. While we may not need the 1000-fold increase at the moment, planning for it will prepare us for the next big leap in model capacity.

\vspace{-1.5pt}

To be specific, let us focus on training large Transformer networks for the language modeling task. At the time of writing, the largest conventional model for that task is GPT-3 with 175 billion parameters. Scaling it up 1000 times gives us 175 trillion; depending on whether you use single or half-precision, this requires 300--600 terabytes of memory just to store the model. No modern mass-produced hardware accelerator is up to such task. Even high-end servers with 16x V100 accelerators can store only 0.15\% of that model in combined GPU memory, let alone train it.

The dominant way of growing neural network size has so far been to scale up: deploy more powerful computational accelerators in specialized tightly interconnected clusters. However, this approach will only work up to a point. Models such as T-NLG~\cite{tnlg} and Megatron-LM~\cite{shoeybi2019megatron} were already trained on DGX-SuperPOD --- a supercomputer with hundreds of Tesla V100 GPUs spread over tens of servers. As for GPT-3~\cite{brown2020language}, a single \textit{training run} was estimated to cost 4.6 -- 12 million dollars~\cite{gpt3costlambda,gpt3cost}. 

Even today, the need for costly hardware weighs heavily on the research community. Most researchers cannot contribute to the development of large neural networks because conducting the necessary experiments would be too expensive for them. If we continue to increase the model size by scaling up, eventually the only labs that can conduct competitive research will be those with massive budgets.

However, there is another solution: to scale out. Instead of using a supercomputer, researchers could crowdsource the computation from volunteers with regular PCs. %
This paradigm is known as volunteer computing and was successfully applied to solve problems in biology \cite{larson_crowd}, high energy physics \cite{adam2015atlas} and other subject areas. While a single volunteer PC may be slow and unreliable, the combined floating-point performance of such projects is on par with largest supercomputers \cite{gross_folding}.

The main challenge of volunteer computing is how to utilize this performance. Unlike server pods, consumer-grade PCs communicate over the Internet, which is significantly slower, especially in terms of latency. They are also more prone to failures as they lack many reliability features of their server-grade counterparts. Therefore, volunteer computing was traditionally used for tasks that have high computation to communication ratio and can recover from individual node failures.

Unfortunately, existing paradigms of distributed training require nodes to continuously transfer large amounts of intermediate data \cite{Dettmers20158BitAF,Sun2019OptimizingNP}, making them unsuitable for volunteer computing. In this work, we take a different approach. Instead of adopting the existing distributed training strategies, we identify the advantages of volunteer computing and design a new strategy that capitalizes on them.

We summarize the contributions of our paper as follows:

\vspace{-6px}
\begin{minipage}{0.55\textwidth}

\begin{itemize}[leftmargin=*]
    \item We propose Decentralized Mixture of Experts (DMoE) --- a layer designed for training with vast amounts of unreliable consumer-grade hardware;%
    \vspace{1px}\item We describe a framework for training large neural networks composed of DMoE layers;%
    \vspace{1px}\item We confirm the efficiency and reliability of this approach using formal guarantees and experiments;
    \vspace{1px}\item The PyTorch source code that can be used to reproduce our results is available online\footnotemark.
\end{itemize}
\end{minipage}
\hspace{5px}
\begin{minipage}{0.45\textwidth}
\vspace{-6px}
    \centering
    \raisebox{\dimexpr \topskip-\height}{\includegraphics[width=180px]{resources/teasseract3.pdf}}
    \captionof{figure}{High-level scheme of Decentralized Mixture of Experts. See Section \ref{sect:method} for details.}
    \label{fig:teaser}
\end{minipage}
\footnotetext{\url{https://github.com/mryab/learning-at-home}}

\vspace{-14px}
\section{Related work}\label{sect:related}
\vspace{-4px}

\subsection{Volunteer computing}\label{sect:related_volunteer}
\vspace{-4px}

Using volunteer hardware has long been a viable alternative to high-performance computing. Since the development of BOINC \cite{anderson2004boinc} research organizations with sufficient public outreach have been able to run massive scientific computations on devices provided by volunteers. Successful projects such as Folding@home can have over $10^5$ active participants, rivaling the floating-point performance of world's fastest supercomputers\footnote{In January 2019, Folding@home reported 146,091 teraflops; in November 2019, the top-1 supercomputer ``Summit'' reported 148,600 teraflops; see \url{top500.org/lists/2019/11} .}. In fact, Folding@home was the first ``supercomputer'' to reach both 1 and 10 petaflops milestones~\cite{folding_timeline}.

However, unlike traditional HPC, the volunteer nature of these projects imposes some additional limitations. First, the majority of volunteers are only available part-time. 
For instance, a participant can provide an office workstation that only contributes compute outside of business hours. 
Second, volunteer hardware is heterogeneous: different nodes may have different performance, memory limits, and even operating systems. Finally, participants usually communicate over the Internet, which is 2--3 orders of magnitude slower than typical HPC connections. As a result, both compute nodes and communication channels are not nearly as reliable as in traditional supercomputers. 

Due to the limitations mentioned above, volunteer computing works best for tasks that can be split into many independent chunks. A single Folding@home task is to run a physical simulation of a protein for a specified number of frames. Together, volunteers can perform hundreds of thousands of concurrent tasks and only need to communicate with the server to submit their results. Other projects like SETI@home and Einstein@home follow a similar pattern.%

Based on the existing volunteer computing projects, we formulate the following usage scenario:
\vspace{-4px}
\begin{itemize}[leftmargin=*]
    \item \textbf{Large pool of weak computers:} the infrastructure consists of $10^3 \sim 10^6$ heterogeneous PCs\footnote{Typical specifications: 2--8 CPU cores, 4--16GB RAM, and a single customer-grade GPU with 2--12GB of memory and 4--14 float32 TFLOPS (based on \url{https://pcpartpicker.com} and \url{https://techpowerup.com})};
    \item \textbf{Communication:} nodes communicate with speed and reliability of a home internet connection\footnote{We assume 20--250ms latency and 100Mbps symmetric bandwidth, $0.33\%$ packet loss based on \cite{speedtest,li2017case}};
    \item \textbf{Frequent node failures:} a compute node may fail to process a task for a variety of reasons. We expect 5--20\% of computers to have at least one failure a day under normal operating conditions.
\end{itemize}
\vspace{-6px}

\subsection{Distributed training}\label{sect:related_distributed}
\vspace{-3px}

To analyze the existing distributed training approaches from the perspective of volunteer computing, we broadly divide them into several categories.

\textbf{Synchronous data parallel training} \cite{valiant1990bridging}\textbf{.} Each worker stores a copy of model parameters, computing gradients for a fraction of the training batch. The gradients are then averaged across workers and applied to the model, making up the same update on all machines. Due to its simplicity and scalability, this method has been widely used to reduce the training time of large neural networks to the order of minutes \cite{goyal2017accurate,You2020Large}. 
    
However, with low-end or midrange hardware it is not always possible to store the entire model on each worker. In addition, gradient communication, even when overlapped with computation, requires a high-speed connection between all participants, often faster than hundreds of megabytes per second, which is unrealistic when considering typical household Internet connections.
    
\textbf{Asynchronous training} \cite{recht2011hogwild, zhang2015staleness} usually involves a single parameter server and multiple compute nodes fetching the latest parameters, processing batches, and submitting updates back to the server. This technique improves worker throughput, but this improvement comes at a cost. If several workers submit simultaneous updates, they might get applied in an arbitrary order, which leads to the issue of \textit{stale gradients} \cite{stale_gradients_can_win} and possibly hinders model convergence.

\textbf{Model parallel training.} Each node stores a fraction of model layers, each training batch is processed by all nodes in a sequential order determined by the layer distribution scheme. The training batch can be divided into several micro-batches and processed in a pipeline fashion, significantly increasing hardware utilization \cite{huang2019gpipe,zero,pipemare,pipedream}.
    
Unlike the two previous paradigms, this method allows training models that exceed the memory limit of any individual worker. Notable examples of successful model parallel training for large neural networks are \cite{huang2019gpipe} and \cite{shoeybi2019megatron}, yet these systems also have a high-speed network between workers. On top of that, model parallelism is highly vulnerable to node and network failures: if a single worker in a chain turns off or stops sending outputs, the training stops entirely. 

It is possible to combine data and model parallelism to mitigate the outlined issues to some degree, but the requirement for fast worker interconnect holds even in that case. In light of this, the method we design has to maintain high throughput even in the presence of slow and unreliable network connections, possibly sacrificing the latency (time to process a given batch) as a necessary tradeoff. 

This constraint may be justified by the following observation: the wall-clock training time of a neural network (with model and optimizer fixed) mostly depends on how many batches it processes per second. As we show in Section \ref{sect:exp_convergence}, the effect of stale gradients can be mitigated with the right architecture. We summarize the desired properties in Table \ref{tab:distributed}.

\begin{table*}[t]
\caption{Comparison of distributed training schemes in the volunteer computing context. ``Desired'' denotes the algorithm with properties that would be beneficial for this setting. ``Only workers'' means that the system has central components that are not fault-tolerant.}
\setlength{\tabcolsep}{3pt}
\hspace{-6pt}\begin{tabular}{cccccccc} 
\toprule
 \multirow{2}{*}{Training method}& Model            & Training       & \multirow{2}{*}{Scalability}    & \multirow{2}{*}{Fault tolerance}             & Worker         & \multicolumn{2}{c}{Network}  \\
         & size limit       & throughput     &                &             & hot-join       & Bandwidth     & Latency                   \\ 
\midrule
Data parallel  & Worker           & \textbf{High } & Medium         & \textbf{Full}         & \textbf{Yes }  & \textbf{High}        & Low                       \\
Asynchronous   & Worker           & \textbf{High } & \textbf{High}  & Only workers\textbf{} & \textbf{Yes }  & Medium        & \textbf{Any}              \\
Model parallel & \textbf{System}  & Medium         & Low            & No                    & No             & High          & Low                       \\
Federated      & Worker           & Low            & \textbf{High}  & Only workers\textbf{} & \textbf{Yes }  & \textbf{Low}        & \textbf{Any}              \\
Desired        & \textbf{System}  & \textbf{High } & \textbf{High}  & \textbf{Full}         & \textbf{Yes }  & \textbf{Low}  & \textbf{Any}              \\
\bottomrule
\end{tabular}
\label{tab:distributed}
\vspace{-12pt}
\end{table*}

\textbf{Federated learning.} The problem of utilizing large quantities of consumer devices for training a single model has also been discussed within the context of data-private learning. Federated learning \cite{mcmahan2017communication} attempts to mitigate the issue by keeping the data on devices, training a local version of the model, and sending only the parameter updates. These updates are encrypted so that the server can only decrypt their average across several devices.

\vspace{-1px}

Unsurprisingly, federated learning sacrifices performance for privacy. Secure aggregation procedures \cite{bonawitz2017practical} require multiple workers to communicate and scale quadratically with their number. These properties hardly align with the scenario from Section \ref{sect:related_volunteer}, making federated learning a poor fit for jointly training large models.

\textbf{Deep learning with volunteer computing.} To the best of our knowledge, there are three projects that use volunteer computing for training neural networks. The first work~\cite{desell2017} leverages volunteer resources for evaluation of CNN architectures generated by evolution algorithms; each model is trained on a single device.
The second study~\cite{volunteer_dl_async} relies on standard asynchronous training and is therefore inapplicable to models that do not fit into a single consumer-grade GPU. Moreover, the architecture described in that study is only partially decentralized, relying on a centralized parameter server that communicates with all nodes. Lastly, the project known as Leela Chess Zero~\cite{lc0}, relies on volunteer hardware to play massive amounts of chess games for generating self-play data used in reinforcement learning. However, the model itself is trained on a single central server.

Our primary insight from this section is that existing methods for training general large neural networks do not fit well into the volunteer computing scenario. However, there is a subclass of deep learning architectures which is much better suited for this task.

\vspace{-2px}
\subsection{Mixture-of-Experts}\label{sect:related_moe}
\vspace{-2px}

Mixture-of-Experts (MoE) was first proposed almost three decades ago as a method to train multiple neural networks (``experts'') for a common task \cite{moe_first}. The intent is for each expert to specialize in making predictions for a small subset of data. Presented with an input, MoE first determines which experts are best suited to process that input using a separate \textit{gating function}. Then it applies the chosen experts and aggregates their outputs into the final prediction. This work has sparked many follow-ups that reveal different MoE structures \cite{jordan1994hierarchical, yao2009hierarchical,moe_lifelong,rasmussen2002infinite} and individual expert types \cite{moe_svm,moe_dirichlet}.

A subsequent study~\cite{eigen2013learning} demonstrates that Mixture-of-Experts can be used as a layer within larger neural networks and trained jointly by backpropagation. Depending on the task, individual experts can utilize convolutional, recurrent, or other specialized layers. Such MoE can have a large number of experts, but it only needs to compute a few of them to process any given input.

Shazeer et al.~\cite{shazeer2017outrageously} (and later~\cite{Lepikhin2020GShardSG}) brought that idea to the extreme by training ``outrageously'' large mixtures with thousands of experts. The drastic increase in capacity allows authors to achieve superior performance in large-scale machine translation and language modeling. The paper also addresses problems that arise with increased mixture size. When trained na\"ively, the gating function learns to use a small fraction of available experts for all inputs, not taking full advantage of the available capacity. The authors alleviate this issue by adding a regularization term that promotes ``load-balancing'' across all experts.

However, scaling this approach from thousands to millions of experts reveals additional problems in the design of a gating function. In order to choose the most appropriate experts for the task, MoE predicts a ``priority'' value for each expert and selects the ones with the highest priority. As the number of experts approaches millions, such a gating function itself becomes computationally intractable, especially in our decentralized setting.

A popular solution to this problem is to structure the set of experts in a search-friendly way. For instance, Hierarchical Mixture-of-Experts~\cite{jordan1994hierarchical} organizes experts in a tree-like structure. Selecting the best experts is then reduced to a beam search over this tree, which scales logarithmically in the number of experts. More recent study by Lample et al. \cite{pkm} explores this idea at scale by organizing over a million keys in a factorized 1024-by-1024 grid. For this grid, the gating function only needs to predict two vectors of size 1024. This work also demonstrates that such layers can benefit Transformer models in the masked language modeling task.

However, these works require a centralized infrastructure for training. When the gating function picks appropriate experts for the input at hand, it must somehow find these experts across all nodes. In our scenario, even maintaining the dynamic ``address book'' of all active experts would be infeasible for any single participant.

\nocite{puigcerver2020scalable}

\vspace{-2px}

\subsection{Distributed Hash Tables}\label{sect:related_dht}

\vspace{-2px}

Fortunately, there is a way to implement bookkeeping in a decentralized system --- the distributed hash table (DHT). This is a family of distributed data structures that store key-value pairs across multiple computers in a network. A single computer within such structure only needs to ``know'' $O(\log N)$ out of $N$ computers; at the same time it can look up any key with at most $O(\log N)$ requests to his peers. There are several DHT variants, but they all have common properties:
\vspace{-4px}
\begin{itemize}[leftmargin=*]
    \item \textbf{Decentralization:} nodes form and maintain DHT without any central coordination;
    \item \textbf{Scalability:} DHT can scale to millions of active nodes that are continually joining and leaving; 
    \item \textbf{Fault tolerance:} a failure in one or a few nodes does not affect DHT integrity and availability;
\end{itemize} 

A DHT-like protocol was first proposed in 1998 by \cite{tewari1998beyond} and popularized in early 2000s by four protocols: CAN~\cite{can}, Chord~\cite{chord}, Pastry~\cite{pastry} and Tapestry~\cite{tapestry}. By far, the most popular DHT variation is Kademlia~\cite{kademlia} with numerous applications such as BitTorrent, I2P, and Ethereum. A more recent work~\cite{kaashoek2003koorde} further improves theoretical performance for either lookup time or the number of connections; however, this version is less widespread due to being significantly harder to implement.


\begin{figure}[ht]
    \centering
    \subfigure[Initialization]{\includegraphics[width=0.32\textwidth]{figures/grad_ratio_init.png}}
    \subfigure[Near Convergence]{\includegraphics[width=0.32\textwidth]{figures/grad_ratio_conv.png}}
    \subfigure[Misclassification]{\includegraphics[width=0.32\textwidth]{figures/grad_ratio_misc.png}}
    \caption{Gradient Ratio}
    \label{fig:grad_ratio}
\end{figure}



\section{Theoretical Analysis and Alternative Loss functions}

In the previous section we described how \Endd can be done by maximising the log-likelihood of the ensemble's output distributions under a conditional Dirichlet Prior. However, we empirically observed significant convergence issues when applying this approach to tasks with large numbers of classes. Thus, in this section we examine the gradients of the Dirichlet NLL loss and propose an alternate training approach which overcomes them.


\textbf{First-Order Analysis}

The setup which will consider in our analysis is the following. First, we have a Prior Network model which is initialized such that it always returns a uniform Dirichlet distribution ($\bm{\alpha} = \bm{1}$), while the target distribution whose probability is being maximized is a sparse K-length vector of probabilities:
\begin{empheq}{align*}
    \bm{\pi}_{tgt} = \big[1-\epsilon, \epsilon/(K-1), \epsilon/(K-1), \cdots \big]^{\tt T},\quad \epsilon = \text{1e-4}
\end{empheq}
Second, we have a Prior Network which is \emph{near convergence} with the following output distribution:
\begin{empheq}{align*}
    \bm{\alpha}_{cnv} =&\ \bm{\pi}_{cnv} \cdot \alpha_0,\ \alpha_0 = 90K,\quad 
    \bm{\pi}_{cnv} =\ \big[1-5\epsilon, \frac{5\epsilon}{K-1}, \frac{5\epsilon}{K-1}, \cdots \big]^{\tt T}
\end{empheq}
Finally, we have a Prior Network which has made a strong mistake, which represents a situation which could occur somewhere in the middle on training, far from convergence:
\begin{empheq}{align*}
    \bm{\alpha}_{msc} =&\ \bm{\pi}_{msc} \cdot \alpha_0,\ \alpha_0 = 90K,\quad 
    \bm{\pi}_{msc} =\ \big[\frac{5\epsilon}{K-1}, \frac{5\epsilon}{K-1}, \cdots, 1-5\epsilon \big]^{\tt T}
\end{empheq}

First, lets consider the standard cross-entropy loss between a predicted and target discrete distribution and it's gradient with respect to the logit $z_k$:
\begin{empheq}{align}
    \mathcal{L}^{\text{CE}} =&\ -\sum_{k=1}^K \hat \pi_k \ln\big(\frac{\alpha_k}{\alpha_0}\big),\quad 
    \frac{\partial\mathcal{L}^{\text{CE}}}{\partial z_k} =\ \frac{\alpha_k}{\alpha_0} - \hat \pi_k 
\end{empheq}

Second, consider the NLL loss of a Dirichlet distribution and its gradient with respect to logit $z_k$:
\begin{empheq}{align}
   \mathcal{L} \small{=} \sum_{k=1}^K\Gamma(\alpha_k) \small{-}(\alpha_k \small{-} 1)\sum_{m=1}^M\frac{\ln\pi_k^{(m)}}{M} \small{-} \Gamma(\alpha_0), \   \frac{\partial\mathcal{L}}{\partial z_k} \small{=} \big(\psi(\alpha_k) \small{-} \psi(\alpha_0)  \small{-}\sum_{m=1}^M\frac{\ln\pi_k^{(m)}}{M}\big) \cdot \alpha_k
\end{empheq}

Finally, consider the dimensionality normalized ratio of the gradient with respect to the logit 1 and logit 2, which represents the relative contribution of the gradients with respect to the class we are interested in modelling to the long tail. 
\begin{empheq}{align}
\begin{split}
        \rho = \frac{1}{K} \Big| \frac{\partial\mathcal{L}}{\partial z_1} \Big| \Big/ \Big|\frac{\partial\mathcal{L}}{\partial z_2}\Big|
\end{split}
\end{empheq}
Figure~\ref{fig:grad_ratio} shows that, at initialization, as the number of classes is increased the standard cross-entropy loss primarily focuses on the high probability class and ignores the long tail. In contrast, the Dirichlet NLL loss displays a diminishing contribution. This means that the loss will focus on modelling the probability distribution of the high-probability classes only after it \emph{perfectly} models the long tail. As the loss is also very sensitive, it means that on complex tasks the model is perpetually stuck modelling the probabilities of tail classes. Note that even near convergence, the ratio $\rho$ is far smaller for the NLL criterion than for discrete cross-entropy. Finally, if a significant error is made on the training data, $\rho$ becomes very large for cross-entropy, and increasingly small for Dirichlet NLL as the number of classes increases. This analysis shows that a desirable property of the loss which ensures good convergence is that the ratio $\rho$ is high and either constant or increasing as the number of classes grows, otherwise the model focuses on modelling the distribution of tail-class probabilities across the ensemble.

An additional issue to consider is that the NLL noise is also noisy, as for each input $\bm{x}$ we only have a few discrete distributions - it may be necessary to use far more samples to get a good estimate of the ensemble's distribution. Furthermore, this distribution may be poorly matched to the Dirichlet, which introduces additional issues. Thus, a natural solution to consider would be to introduce a \emph{Proxy Dirichlet Distribution} to which we can minimize either the \emph{KL-divergence} or \emph{reverse KL divergence}. We leave discussion of the details of the Proxy Dirichlet until later and only consider the gradients which arise from minimizing either loss.  

For this analysis we consider a target Dirichlet distribution with parameters $\bm{\beta} = \bm{\pi}_{tgt}*\beta_0$ where $\beta_0 = 100K$. The explicit forms of the KL-divergence between two Dirichlet distributions, as well the gradient of the forward and reverse KL-divergence are provided below:
\begin{empheq}{align}
\begin{split}
       & \mathcal{L}^{\text{KL}} =\ \ \sum_{k=1}^K\Gamma(\alpha_k) - \sum_{k=1}^K\Gamma(\beta_k) + \Gamma(\beta_0) - \Gamma(\alpha_0) + \sum_{k=1}^K(\beta_k - \alpha_k)\Big(\psi(\beta_k)-\psi(\beta_0)\Big)
\end{split} \\
\begin{split}
       & \mathcal{L}^{\text{RKL}} =\ \ \sum_{k=1}^K\Gamma(\beta_k) - \sum_{k=1}^K\Gamma(\alpha_k) + \Gamma(\alpha_0) - \Gamma(\beta_0) + \sum_{k=1}^K(\alpha_k - \beta_k)\Big(\psi(\alpha_k)-\psi(\alpha_0)\Big)
\end{split} \\
\begin{split}
        &\frac{\partial\mathcal{L}^{\text{KL}}}{\partial z_k} =\ \big(\psi(\alpha_k) - \psi(\alpha_0) - \psi(\beta_k) + \psi(\beta_0)\big) \cdot \alpha_k
\end{split} \\
\begin{split}
        &\frac{\partial\mathcal{L}^{\text{\tiny RKL}}}{\partial z_k} =\ \big((\alpha_k - \beta_k)\psi'(\alpha_k) - (\alpha_0 - \beta_0)\psi'(\alpha_0)\big) \cdot \alpha_k
\end{split}
\end{empheq}

Figure~\ref{fig:grad_ratio} additionally displays the ratio $\rho$ for both the forward and reverse KL-divergence losses. The forward KL-divergence displays the same issues as the NLL loss and $\rho$ continues to decrease as the number of classes in increased. This is unsurprising, as the NLL is equivalent to the KL-divergence in the limit. However, the \emph{reverse KL-divergence} displays the desirable properly that $\rho$ grows and stabilizes as the number of classes is increased. This suggests that if we were to minimize the \emph{reverse KL-divergence} to an appropriately chosen \emph{Proxy-Target Dirichlet distribution}, then we would be able to avoid convergence issues. 

% \textbf{Second-Order Analysis}

% In addition to the first-order analysis provided above, we also conduct a second order analysis by considering the eigenvalues of the Hessian of the loss.


\textbf{Proxy-Dirichlet distribution}

\begin{figure*}[ht]
    \centering
    \subfigure[Naive \Endd]{\includegraphics[scale=0.067]{figures/naive-distillation-2.png}}
    \subfigure[Proxy \Endd]{\includegraphics[scale=0.067]{figures/proxy_distillation.png}}
    \caption{Schematic of Distillation Approaches}
    \label{fig:distillation overview}
\end{figure*}

It is important to remember that the ensemble may be poorly modelled via a Dirichlet distribution, so it is necessary to ask which properties of the ensemble we are actually interested in capturing. Clearly, we would like to capture the mean of the ensemble, as that typically has better predictive accuracy and calibration. Additionally, we would like to capture \emph{bulk-diversity properties} of the ensemble, such that the measures of divergence derived from the Proxy Dirichlet are similar to those of the original ensemble and therefore provide a similar rank-ordering of data. At the same time, we are \emph{not} interested modelling properties like multi-modality and skew. 

Clearly, obtaining the mean of the ensemble is trivial. Obtaining an estimate of the precision $\beta_0$ is more challenging. One approach based on Sterling's approximation is described in~\cite{minka2000estimating} and proposes the following estimate:
\begin{empheq}{align}
\begin{split}
        \hat \pi_k (\bm{x})=&\ \frac{1}{M}\sum_{m=1}^M {\tt P}(y=\omega_k|\bm{x}, \bm{\theta}^{(m)}) \\
        \tilde \beta_0(\bm{x}) =& \frac{K-1}{2 \sum_{k=1}^K\hat \pi_k (\ln \hat \pi_k - \frac{1}{M}\sum_{m=1}^M\ln \pi_k^{(m)})},\ \bm{\beta}_k (\bm{x}) = \ \hat \pi_k(\bm{x}) \cdot \tilde \beta_0(\bm{x}) + 1
\end{split}
\end{empheq}

We found that it is important to also add 1 to all the target concentration parameters. Figure~\ref{fig:grad_ratio_smooth} shows that for the reverse KL loss, adding 1 to \emph{both} the target Proxy-Dirichlet as well as \emph{the model} yields an improved ratio $\rho$ both at initialization and near convergence. Heuristically, it seems to make the loss more linear and stable by preventing the digamma and trigamma functions $\psi$ and $\psi'$ in the reverse-KL loss from dropping into the highly non-linear regime when $\alpha_k < 1$ and $\beta_k < 1$.
\begin{figure}[ht]
    \centering
    \subfigure[Initialization]{\includegraphics[scale=0.49]{figures/grad_ratio_init_smooth.png}}
    \subfigure[Near Convergence]{\includegraphics[scale=0.49]{figures/grad_ratio_conv_smooth.png}}
    \caption{Gradient Ratio}
    \label{fig:grad_ratio_smooth}
\end{figure}

Note, while the solution may seem to similar to work done in \cite{malinin-rkl-2019}, the fundamental underlying reason for using this loss is altogether different. Here, the issue is due to large gradients from low-probability tail classes, while in~\cite{malinin-rkl-2019} the reverse KL loss is used to avoid inducing a multi-modal target Dirichlet distribution in expectation. 

\begin{empheq}{align}
\begin{split}
{\tt KL}[{\tt p}(\bm{\pi}|\bm{x},\bm{\theta}) \| {\tt p}(\bm{\pi}|\bm{\hat \beta})] =&\ \underbrace{ \beta_0\cdot\mathbb{E}_{{\tt p}(\bm{\pi}|\bm{x},\bm{\theta})}\big[-\sum_{k=1}^K\hat \pi_k\ln \pi_k\big]}_{\text{Reconstruction term}} + \underbrace{{\tt KL}[{\tt p}(\bm{\pi}|\bm{x},\bm{\theta}) \| {\tt p}(\bm{\pi}|\bm{1})]}_{\text{Prior}} +Z
\end{split}
\end{empheq}
% \textbf{Alternative solutions (if it fits) }

% If not, we'll move that to the appendix (along with comparisons)

% \begin{itemize}
%     \item Top-k aggregation
%     \item Softplus parametrization
% \end{itemize}

\section{Experiments}
\label{sec:experiments}

In this section, we evaluate \Endd via minimization of Reverse KL-divergence between the model and a Proxy Dirichlet target. We apply distribution distillation to ensembles of convolutional networks trained on the ImageNet dataset and to ensemble of Transformer models trained on WMT'17 En-De. Our goal here is to demonstrate that given an ensemble, we can successfully distribution-distill it into a single model. Note that we do not provide results for \Endd accomplished by optimizing Dirichlet NLL or forward KL-divergence, because we could not get them to even begin to converge on the tasks considered here. 

\subsection{Setup}
\label{sec:experiments_setup}
We consider two large-scale tasks involving classification: 1000-class image classification and sequence-to-sequence modeling of natural language. For each task, we first train the ensemble of regular models and then distill it with \Endd. For comparison, we also report the average single-model performance along with the following baselines:

\begin{itemize}
\item \textbf{Ensemble} refers to the performance of an ensemble of independently trained models, which was previously shown to yield high quality uncertainty estimates~\cite{deepensemble2017} and to outperform more sophisticated methods using only a few models~\cite{ashukha2020pitfalls}.
\item \textbf{Ensemble Distillation} (EnD) is a common approach to model and ensemble distillation, first proposed in~\cite{hinton2015distilling}. It involves training the student model with the soft target distribution of averaged ensemble predictions. Notably, we do not add the cross-entropy loss for ground truth labels, because we focus on the comparison of distillation objectives and not only classification performance.
\end{itemize}

% TODO discuss why not MC-dropout?
We do not use Hydra~\cite{hydra} or similar multi-head approaches for distilling each separate ensemble member, because with a high number of models in the ensemble and even 1000 classes the computation overhead is no longer negligible. In all experiments with \Endd, we add 1 both to the predicted parameters of the Dirichlet distribution and the Dirichlet proxy parameters.
% : we evaluate the performance of versions without these modifications in Section~\ref{sec:experiments_ablation}

Both for error rejection and out-of-distribution detection, we use several information-theoretic measures uncertainty; in particular, we use entropy of the expected predictive distribution (EoE) for total uncertainty and Reverse Mutual Information (RMI) for knowledge uncertainty throughout this section.
Derivations of these measures both for \Endd and ensembles are available in~\cite{malinin-thesis} and~\cite{malinin-structured-2020}.
For Single and EnD single-model baselines, we use
entropy of the output distribution as the only valid uncertainty estimate.
% the same measures of uncertainty by interpreting exponents of logits as parameters of a Dirichlet distribution. 
% As we show later, in some setups the performance of such models can be surprisingly competitive with that of \Endd and even ensembles; we leave the study of this phenomenon to future work.

\subsection{Large-scale image classification}
\label{experiments:imagenet}

For the first experiment, we run distillation of the ensemble that contains 10 ResNet-50~\cite{resnet} models trained on the ImageNet~\cite{imagenet} image classification dataset. We use the standard training setup outlined in~\cite{touvron2019FixRes}; specifically, we train for 90 epochs using stochastic gradient descent with momentum of 0.9 and a learning rate of $0.1\times B/256$ (first proposed in~\cite{goyal2018accurate}), where B is the per-device batch size multiplied by the number of GPUs. 
In our experiments, we use a single-GPU batch size of 256 and 8 NVIDIA V100 GPUs. The learning rate is divided by 10 every 30 epochs. For data augmentations, we use a standard combination of random resized crops and horizontal flips implemented in the Albumentations library~\cite{albumentations}.
In all experiments, we found it beneficial to initialize the last batch normalization $\gamma$ in each residual branch to zero, which agrees with previous results~\cite{goyal2018accurate, zhang2018residual, rezero}.

For a thorough evaluation of all methods, we use several different characteristics of performance. First, we measure the in-domain classification accuracy on the original ImageNet validation subset~\cite{imagenet}, which is commonly used for comparison of image classification models. Second, we compare the robustness of all approaches to different domain shifts, also measured by accuracy on datasets corresponding to these shifts. In particular, we use adversarial examples from ImageNet-A~\cite{hendrycks2021nae}, corrupted and perturbed versions of original ImageNet validation data from ImageNet-C~\cite{hendrycks2019robustness}, and artistic renditions from ImageNet-R~\cite{hendrycks2020many}. Next, these domain shift and the original validation dataset are used to compare calibration of models with Expected Calibration Error (ECE).
Finally, we measure the out-of-distribution detection error in terms of Receiver Operating Characteristic area under curve (ROC AUC) on the domain shift datasets together with ImageNet-O~\cite{hendrycks2021nae}.

% \begin{itemize}
% \item In-domain classification accuracy
% \item Robustness is specifically classification accuracy for out-of-domain data
% \item Expected calibration error (ECE)
% \item Out-of-domain detection error: , measured in terms of the area under the ROC AUC curve
% \end{itemize}

% Finally, we also evaluate the out-of-distribution detection error, measured in terms of Receiver Operating Characteristic area under curve (ROC AUC).


We report the results for all metrics in Tables~\ref{tab:imagenet_pred} and~\ref{tab:imagenet_ood} for prediction quality and out-of-distribution detection respectively.
Here, the metrics on ImageNet-C are averaged over all degrees of corruption; in Figure~\ref{fig:imagenet_breakdown}, we provide the detailed results of evaluation on each degree separately.
For out-of-distribution detection, we also provide the results of the Dirichlet Proxy to verify that this approximation of the ensemble predictive distribution does not significantly affect its performance.

Table~\ref{tab:imagenet_pred} shows that \Endd is capable of accurate emulation of the ensemble in terms of classification performance: in terms of accuracy, the method displays results on par or slightly better than regular distillation while also having smaller calibration errors. Also, in Table~\ref{tab:imagenet_ood}, it can be seen that for most datasets (except the hardest ImageNet-O) Proxy-Dirichlet distillation can closely match the out-of-distribution performance of the ensemble. As expected, both distillation methods outperform training a single model from scratch while having the same computational complexity.

Furthermore, Figure~\ref{fig:imagenet_breakdown} shows that as the domain shift increases, all models suffer from a drop in accuracy and calibration quality; notably, EnD and \Endd have the same calibration performance on original data, but Dirichlet network distillation has lower calibration errors for the highest degrees of corruption. Unsurprisingly, the further the data is from the original training images, the better the models are at out-of-distribution detection.

\begin{table}
\centering
\small
\caption{Prediction quality results for image classification.}
\label{tab:imagenet_pred}
\begin{tabular}{lcccccccc}
\toprule
{} & \multicolumn{2}{c}{ImageNet-val} & \multicolumn{2}{c}{ImageNet-A} & \multicolumn{2}{c}{ImageNet-C} & \multicolumn{2}{c}{ImageNet-R} \\
{} &          Acc &      ECE &        Acc &       ECE &        Acc &       ECE &        Acc &       ECE \\
\midrule
Single   &     75.9±0.1 &  4.8±0.1 &    4.4±0.2 &  51.1±0.3 &   39.1±0.7 &  11.3±0.7 &   35.0±0.2 &  21.3±0.4 \\
Ensemble &         79.0 &      2.3 &        3.9 &      42.0 &       43.5 &       4.5 &       38.8 &       9.8 \\
EnD      &         77.0 &      1.6 &        3.8 &      46.6 &       40.6 &       5.9 &       36.9 &      16.1 \\
\Endd    &         77.1 &      1.6 &        3.9 &      42.8 &       40.6 &       4.5 &       37.0 &      11.8 \\
\bottomrule
\end{tabular}
\end{table}

\begin{table}
\centering
\small
\caption{Out-of-distribution detection results for image classification.}
\label{tab:imagenet_ood}
\begin{tabular}{lcccccccc}
\toprule
{} & \multicolumn{2}{c}{ImageNet-O} & \multicolumn{2}{c}{ImageNet-A} & \multicolumn{2}{c}{ImageNet-C} & \multicolumn{2}{c}{ImageNet-R} \\
{} &        EoE &   RMI &        EoE &   RMI &        EoE &   RMI &        EoE &   RMI \\
\midrule
Single   &   50.7±0.3 &     - &   85.8±0.1 &     - &   79.9±0.4 &     - &   83.0±0.2 &     - \\
Ensemble &       54.6 &  62.7 &       88.8 &  86.7 &       82.0 &  77.5 &       86.1 &  84.1 \\
Proxy    &       54.6 &  62.9 &       88.8 &  86.5 &       82.0 &  77.3 &       86.1 &  84.0 \\
EnD      &       48.4 &     - &       87.2 &     - &       80.8 &     - &       83.9 &     - \\
\Endd    &       52.0 &  53.2 &       86.8 &  84.6 &       80.1 &  76.9 &       83.7 &  81.4 \\
\bottomrule
\end{tabular}
\end{table}

\begin{figure}
    \centering
    \includegraphics[width=\textwidth]{figures/breakdown.pdf}
    \caption{Performance of image classification models depending on the level of ImageNet-C corruption.  No corruption corresponds to the original ImageNet validation data.}
    \label{fig:imagenet_breakdown}
\end{figure}

\subsection{Machine translation}
\label{experiments:nmt}
For this experiment, we train standard Transformer-big~\cite{vaswani2017attention} models on the WMT'17 English-German machine translation dataset with the vocabulary of 40,000 Byte-Pair Encoding tokens~\cite{sennrich-etal-2016-neural}. Each of the 10 ensemble members is trained with the setup described in~\cite{ott2018scaling}: in particular, we train them for 193,000 steps with Adam~\cite{adam} on 8 NVIDIA V100 GPUs with a batch size of 4096 tokens per GPU. We train all distillation models for 20,000 steps with the increased batch size of 32K tokens. Because our approach requires fitting all 10 ensemble members in GPU memory, we reduce the immediate batch size for each step to 1024, but compensate for it with gradient accumulation over 32 steps. For output generation and estimation of uncertainty measures (where applicable), we use beam search with beam size 5.

To compare the approaches in terms of translation quality, we use the BLEU score~\cite{papineni2002bleu} computed with SacreBLEU~\cite{sacrebleu} and sequence-level Prediction Rejection Ratio~\cite{malinin-thesis} on the newstest14 English-German test set. For out-of-distribution detection, we also compute ROC AUC and use several datasets with different characteristics and degrees of domain shift: sentences with permuted tokens in the input, LibriSpeech~\cite{librispeech} test-clean speech transcriptions, and source sentences from newstest14 in German and French languages respectively. We average the results of both distillation methods over 5 random seeds and provide standard deviations of all metrics.

\begin{table}
\centering
\small
\caption{Prediction quality results for machine translation.}
\label{tab:wmt_pred}
\begin{tabular}{lccc}
\toprule
{} &      BLEU &       EoE &       RMI \\
\midrule
Single   &  28.8±0.1 &  36.0±1.3 &  - \\
Ensemble &      30.1 &      30.2 &      26.0 \\
EnD      &  29.4±0.1 &  35.6±0.4 &  - \\
\Endd    &  29.5±0.1 &  35.9±0.8 &  35.8±0.5 \\
\bottomrule
\end{tabular}
\end{table}

\begin{table}
\centering
\small
\caption{Out-of-distribution detection results for machine translation.}
\label{tab:wmt_ood}
\begin{tabular}{lcccccccc}
\toprule
{} & \multicolumn{2}{c}{Permuted} & \multicolumn{2}{c}{Speech} & \multicolumn{2}{c}{German} & \multicolumn{2}{c}{French} \\
{} &       EoE &       RMI &       EoE &       RMI &       EoE &       RMI &       EoE &       RMI \\
\midrule
Single   &  80.7±1.5 &         - &  73.7±1.2 &         - &  32.8±2.8 &         - &  27.1±6.3 &         - \\
Ensemble &      83.7 &      97.4 &      67.8 &      73.7 &      39.5 &      82.4 &      25.0 &      73.6 \\
EnD      &  79.5±1.1 &         - &  75.9±0.6 &         - &  35.4±1.6 &         - &  15.6±3.2 &         - \\
\Endd    &  78.3±1.6 &  97.1±0.3 &  77.0±0.3 &  78.5±0.2 &  38.3±1.6 &  70.9±0.7 &  15.9±3.0 &  60.1±3.6 \\
\bottomrule
\end{tabular}
\end{table}

Table~\ref{tab:wmt_pred} further confirms the findings made in the previous section: \Endd via Dirichlet-Proxy outperforms regular ensemble distillation in terms of translation quality and sequence-level error detection. Furthermore, in Table~\ref{tab:wmt_ood} we see that, compared to image classification, the OOD performance gap between total uncertainty and knowledge uncertainty is significantly larger. This might be explained by a significantly larger output space (40,000 classes instead of 1000) or the sequential nature of NMT predictions: because the model generates candidates in a large output space of all possible sequences, its prediction entropy might be high regardless of presence of a domain shift.

% \subsection{Ablation study}
% \label{sec:experiments_ablation}


% \begin{table}[t]
% \centering
% \begin{tabular}{@{}lllll@{}}
% \toprule
%  &  & \multicolumn{3}{l}{OOD detection} \\
%  & Accuracy & Imagenet-C & Imagenet-R & Imagenet-A \\ \midrule
% END\textasciicircum{}2 &  &  &  &  \\
% END\textasciicircum{}2+RKL mediator etc &  &  &  &  \\ \midrule
% - target smoothing &  &  &  &  \\
% - shifted parametrization &  &  &  &  \\
% RKL -\textgreater Forward KL &  &  &  &  \\ \bottomrule
% \end{tabular}%
% \end{table}

% Several additional modifications are given in the Appendix\ref{TODOappendix_ablation}

\section{Conclusion}

This paper introduces \textsc{Petals}, a system for efficient collaborative inference and fine-tuning of large language models. We offer a user-friendly generation interface and a flexible API to access models served over the Internet. We use 8-bit compression that reduces the resource requirements to run very large models. In addition, we develop algorithms for reliable routing and load balancing.

% Since \textsc{Petals} is open-source, we would like it to evolve based on the community's feedback, incorporating relevant research advances and adding support for features in demand.
With the release of this system, we hope to broaden access to LLMs and pave the road to applications, studies or research questions that were previously not possible or simply too expensive.

% [Commented since the Discussion section has been moved to the main text]
% Running LLMs over the Internet raises a broad range of related questions. One of them is privacy: how to avoid revealing private data to outside peers. Another challenge is to ensure that participants can benefit from this system equitably, i.e. in proportion to their contribution.
% We discuss future problems such as privacy, security, and incentive structures in Appendix~\ref{sect:discussion}.


\pagebreak



\bibliographystyle{icml2023}
\bibliography{bibliography}


\section{Cost and performance estimate of \$2500 desktop PCs}
\vspace{-2px}

According to several PC building websites (\url{https://pcpartpicker.com}, \url{https://newegg.com}), most popular \$2250--2750 desktops are equipped with RTX 2080/2080Ti or GTX 1080Ti GPU. These GPUs are 50--80\% as fast as Tesla V100 for deep learning \cite{lambdabenchmarks}. As a rough estimate, the combined throughput of 10,000 desktops is 8--15 times that of server pod with 512 V100 GPUs.

\section{A primer on Distributed Hash Tables}
\vspace{-2px}

On a high level, DHT is a dictionary that can be accessed by every participant. Each key-value pair is stored on a small subset of peers determined by the hash function of the key.
\begin{itemize}
    \item Each participant has a unique identifier (ID) that is sampled uniformly from the space possible outputs of the hash function.
    \item When storing a $(key,\ value)$ pair, one should search for $k$ peers whose IDs are closest to $\mathrm{hash}(key)$. Then, request each of these $k$ peers to store the $(key,\ value)$ pair.
    \item When retrieving a value for a key, one should compute $\mathrm{hash}(key)$, search for peers with IDs similar to that hash value and request value from those peers.
\end{itemize}

Specific DHT variants such as Chord~\cite{chord} or Kademlia~\cite{kademlia} employ different hash types and different algorithms for finding nearest peers. For instance, Kademlia DHT selects nearest peers based on the XOR distance function: $d(x, y) = \mathrm{int}(x \oplus y)$.

Each participant is directly aware of only a small subset of DHT peers. When storing or retrieving a key, the participant requests additional peers from its neighbors in a semi-greedy search, minimizing XOR distance until it finds $k$ nearest peers. In Kademlia, nodes form a special navigable graph structure that lets them find nearest peers in at most $O(k + \log_2 N)$ requests to other DHT peers, where $N$ is the total number of participants. 

\section{Finding best experts across the DHT}\label{appendix:find_experts}
\vspace{-2px}

Recall that the gating function is defined as 
\[
g(x, f) = \sum_{i=0}^{d - 1} g_i(x)[u_i],
\]
where $g_0,\dots\,g_{d-1}$ are linear layers, $u_i$ is the $i$-th component of the expert unique identifier $\mathrm{uid}(f)$, and $[k]$ takes $k$-th component of a vector. Our objective is to find $k$ experts with largest $g(x, \cdot)$. In a centralized setting, one can find $k$ largest scores from each linear layer $g_i$ using the algorithm described in \cite{pkm}.

Unfortunately, in our case not all combinations of indices correspond to valid experts. Therefore, we developed a specialized beam search algorithm similar to the one used in machine translation. The core idea is to start with top-$k$ indices along the first grid dimension and add one dimension at a time.

In order for this algorithm to work, participants maintain the following information on the DHT:

\begin{itemize}
    \item For every expert UID, store its server address and the timestamp;
    \item For every prefix in expert UID, store all suffixes corresponding to active experts and the timestamp.
\end{itemize}

For instance, if there are 6 experts: "ffn.1.3", "ffn.2.1", "ffn.2.2", "ffn.2.6" and "ffn.3.2" and "ffn.3.5"; the DHT will contain the following information:

\begin{figure}[h!]
    \centering
    \setlength{\tabcolsep}{3pt}
    \renewcommand{\arraystretch}{1.2}
    \begin{tabular}{c|c|c|c|c|c|c|c|c|c}
    \toprule
    Key    & ffn.1.*   & ffn.2.*         & ffn.3.*      & ffn.1.3 & ffn.2.1 & ffn.2.2 & ffn.2.6 & ffn.3.2 & ffn.3.5 \\
    Value  & [3],$t_1$ & [1, 2, 6],$t_2$ & [2, 5],$t_3$ & \multicolumn{6}{c}{[Address of a server that hosts the given expert]}\\
    \bottomrule
    \end{tabular}
    \caption{DHT keys and values for 6 experts defined above, t corresponds to last update timestamp.}
\end{figure}

For higher grid dimensions, we store similar information for every grid prefix. For instance, an expert with UID "transformer.10.20.30" will affect 3 keys: "transformer.10.*", "transformer.10.20.*" and "transformer.10.20.30". Each prefix key stores at most as many values as there are indices in the next grid dimension, typically 100 or 256.

With this data structure, DMoE can use beam search to select the best experts. Algorithm \ref{alg:beam_search} starts from the leftmost dimension of the grid and processes one dimension at each step. The worst case complexity of this algorithm is $O(d k \log N)$ from  $O(d k)$ lookups to the DHT.


\begin{algorithm}[h]
   \caption{SelectExperts}
   \label{alg:beam_search}
\begin{algorithmic}
   \STATE {\bfseries Input:} $x, k, d, M,\ (g_0, \ldots, g_{d-1})$
   \STATE beam $ := [0, 1, ..., M - 1]$ \quad \quad \quad \quad \quad \quad \quad // all 1-prefixes
   \STATE scores $ := [g_0(x, 0) ... g_0(x, M - 1)]$ \quad \quad  \quad // initial scores
   \STATE // select $k$ best starting points
   \STATE beam, scores $:=$ TopK(beam, scores, k)
   \FOR{$i \in [1,\ \ldots,\ d - 1]$}
   \STATE // expand all candidates in beam
   \STATE new\_beam, new\_scores $ := [\ ], [\ ]$
   \FOR{prefix, score $\in$ beam, scores}
   \FOR{$j \in \mathrm{ActiveSuffixes(prefix)}$}
       \STATE new\_beam.add(prefix$ \bigoplus [j]$) // concat
       \STATE new\_scores.add(score $ + g_i(x, j)$)
   \ENDFOR
   \ENDFOR
   \STATE // select at most $k$ best prefixes
   \STATE beam, scores $:=$ TopK(new\_beam, new\_scores, k)
   \ENDFOR
   \STATE {\bfseries Return} beam
\end{algorithmic}
\end{algorithm}

The TopK function simply sorts the inputs by score and returns $k$ inputs with highest scores. In turn, the ActiveSuffixes function queries the DHT for a given prefix and returns a set of all active suffixes as described above. Assuming that servers re-publish their experts every $t$ seconds, the function can simply check whether the timestamp for a given prefix is less than $t$ seconds old.

\vspace{-4pt}
\section{On gradient checkpointing in Learning@home}\label{appendix:checkpoints}
\vspace{-2px}

In general, gradient checkpointing increases computation per training batch by approximately 1/3, but allows training larger models with the same GPU memory. More importantly, in our scenario checkpointing also removes the need to store intermediate activations. In our experiments, this has led to both significantly higher training throughput and a smaller memory footprint.

Without gradient checkpointing, we would have to store intermediate activations in memory. Since the GPU can only fit a few batches at a time, it quickly runs out of memory and is forced to wait for the backward pass. For Transformer layers (see Figure 4, top), this results in approximately 9 times less throughput at 100ms latency.

\vspace{-4pt}
\section{Reducing the network load}\label{appendix:networkload}
\vspace{-4pt}

One way to reduce the communication load is to convert tensors to a lower precision before transfer. Prior work in this area suggests that distributed training works even when communicating with 8-bit precision tensors~\cite{Dettmers20158BitAF, natural_compression}. Many popular architectures, including Transformers, can train entirely in that precision mode \cite{NIPS2019_8736}. Consequently, low precision communication appears as a logical way of reducing communication requirements.

In addition, the deep learning architectures discussed in this work rely on backpropagation for training. With the advancement of optimization methods allowing nearly independent layer-wise training~\cite{ma2019hsic,jaderberg2017decoupled,real2017large}, it might be even more suitable to use these techniques for asynchronous training with fewer restrictions on the architectures being used.

Another solution is to use experts that have a higher capacity to input size ratio. The architectures used in Section 4.1 are already somewhat biased in that direction, but they are far from optimal.

\end{document}
