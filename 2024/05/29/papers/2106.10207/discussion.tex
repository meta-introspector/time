\section{Conclusion}\label{sect:conclusion}

In this work, we proposed DeDLOC --- a collaborative deep learning approach that enables large-scale collective distributed training on whichever computers available to participants, regardless of hardware and network limitations.
We demonstrated with several experiments that this is a viable approach that maintains its efficiency in a broad range of conditions. Finally, we report the first real collaborative training run of such a scale and share our findings on volunteer activity to pave the road for similar experiments in the future.

An essential property of collaborative training is its environmental impact. While all distributed training experiments have a negative impact due to carbon emissions~\cite{Anthony2020CarbontrackerTA}, DeDLOC has one unique advantage. Due to the ability to utilize heterogeneous low-end devices, it can prolong the effective lifespan of existing computers. We discuss other aspects of environmental impact in Appendix~\ref{appendix:env_impact}.

One issue that needs to be addressed before starting collaborative experiments is the need to gather a community of volunteers. Although our proposed authentication mechanism (see Appendix~\ref{appendix:authorization}) allows acknowledging participants for their contributions (briefly discussed in Appendix~\ref{appendix:contribution_measurement}), the best approach to recruit volunteers is an open question: one needs to take into account both the resources of community members and their motivation for training a specific model.