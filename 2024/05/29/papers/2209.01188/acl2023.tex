% This must be in the first 5 lines to tell arXiv to use pdfLaTeX, which is strongly recommended.
\pdfoutput=1
% In particular, the hyperref package requires pdfLaTeX in order to break URLs across lines.

\documentclass[11pt]{article}

% Remove the "review" option to generate the final version.
\usepackage{ACL2023}

% This is not strictly necessary, and may be commented out.
% However, it will improve the layout of the manuscript,
% and will typically save some space.
\usepackage{microtype}

% This is also not strictly necessary, and may be commented out.
% However, it will improve the aesthetics of text in
% the typewriter font.
\usepackage{inconsolata}

\usepackage[utf8]{inputenc} % allow utf-8 input
\usepackage[T1]{fontenc}    % use 8-bit T1 fonts
\usepackage{hyperref}       % hyperlinks
\usepackage{url}            % simple URL typesetting
\usepackage{booktabs}       % professional-quality tables
\usepackage{amsfonts}       % blackboard math symbols
\usepackage{nicefrac}       % compact symbols for 1/2, etc.
\usepackage{microtype}      % microtypography
\usepackage{xcolor}         % colors

% --- Copied: begin
% Standard package includes
\usepackage{times}
\usepackage{latexsym}

\usepackage{graphicx}
\usepackage{subfigure}
\usepackage{booktabs} % for professional tables
\usepackage{lipsum}
\usepackage{amsmath}
\usepackage{amssymb, amsthm, latexsym}
\usepackage{multirow}

\usepackage{nicefrac}

% \usepackage{minted}
% To submit on arxiv, comment the line above and follow this: https://github.com/gpoore/minted/issues/113#issuecomment-888045507
% \usepackage[finalizecache,cachedir=.]{minted}
\usepackage[frozencache,cachedir=.]{minted}

\newminted{python}{%
    fontsize=\fontsize{8.25pt}{8.25pt}\selectfont,
    % options to customize output of pythoncode
}
% --- Copied: end


% If the title and author information does not fit in the area allocated, uncomment the following
%
%\setlength\titlebox{<dim>}
%
% and set <dim> to something 5cm or larger.

\title{\textsc{Petals}: Collaborative Inference and Fine-tuning of Large Models}

% Author information can be set in various styles:
% For several authors from the same institution:
% \author{Author 1 \and ... \and Author n \\
%         Address line \\ ... \\ Address line}
% if the names do not fit well on one line use
%         Author 1 \\ {\bf Author 2} \\ ... \\ {\bf Author n} \\
% For authors from different institutions:
% \author{Author 1 \\ Address line \\  ... \\ Address line
%         \And  ... \And
%         Author n \\ Address line \\ ... \\ Address line}
% To start a seperate ``row'' of authors use \AND, as in
% \author{Author 1 \\ Address line \\  ... \\ Address line
%         \AND
%         Author 2 \\ Address line \\ ... \\ Address line \And
%         Author 3 \\ Address line \\ ... \\ Address line}

\author{
  Alexander Borzunov\thanks{\ \ Equal contribution. Correspondence to:\newline \texttt{borzunov.alexander@gmail.com}}\\
  HSE University, Yandex \\\And
  Dmitry Baranchuk$^*$ \\
  Yandex \\\And
  Tim Dettmers$^*$ \\
  University of Washington\\\AND
  Max Ryabinin$^*$ \\
  HSE University, Yandex \\\And
  Younes Belkada$^*$ \\
  Hugging Face, ENS Paris-Saclay \\\And
  Artem Chumachenko \\
  Yandex \\\AND
  Pavel Samygin \\
  Yandex School of Data Analysis \\\And
  Colin Raffel \\
  Hugging Face
}

\begin{document}
\maketitle
\begin{abstract}
Many NLP tasks benefit from using large language models (LLMs) that often have more than 100 billion parameters. With the release of BLOOM-176B and OPT-175B, everyone can download pretrained models of this scale. Still, using these models requires high-end hardware unavailable to many researchers. In some cases, LLMs can be used more affordably via RAM offloading or hosted APIs. However, these techniques have innate limitations: offloading is too slow for interactive inference, while APIs are not flexible enough for research that requires access to weights, attention or logits. In this work, we propose \textsc{Petals}\footnote{\textsc{Petals} source code and documentation are available at \texttt{\href{https://petals.ml}{https://petals.ml}}} --- a~system for inference and fine-tuning of large models collaboratively by joining the resources of multiple parties. We demonstrate that this strategy outperforms offloading for very large models, running inference of BLOOM-176B on consumer GPUs with $\approx$~1 step per second, which is enough for many interactive LLM applications. Unlike most inference APIs, \textsc{Petals} also natively exposes hidden states of served models, allowing to train and share custom model extensions based on efficient fine-tuning methods.
\end{abstract}

\vspace{-12pt}
\section{Introduction}\label{sect:intro}
\vspace{-4pt}

Our investigation begins with a thought experiment. Imagine a deep neural network with capacity 1000 times greater than today's most powerful architectures: for example, a language model trained on all digitally available texts or a generative model for all images ever uploaded to the Internet. How can we train such a model?

\vspace{-1.5pt}

Viewed from a historical perspective, the 1000-fold increase in capacity is not unrealistic. Over the past decade, the deep learning community has made remarkable progress by training large models on abundant data, and the scale of those models keeps growing. Since the advent of the ImageNet challenge \cite{imagenet_cvpr09} with 1.3M labeled images, the typical size of convolutional neural networks increased from a few megabytes to hundreds of megabytes \cite{alexnet, resnet, huang2019gpipe}. Recent studies report even larger models for datasets with hundreds of millions of images \cite{kolesnikovlarge, jft300data}.

\vspace{-1.5pt}

Another trend from natural language processing is to train large Transformer-like language models~\cite{bert, roberta, kaplan2020scaling}. The data for this task is nearly unlimited, allowing researchers to train models with tens or even hundreds of gigabytes of parameters~\cite{brown2020language,shoeybi2019megatron,zellers2019defending,tnlg}. While we may not need the 1000-fold increase at the moment, planning for it will prepare us for the next big leap in model capacity.

\vspace{-1.5pt}

To be specific, let us focus on training large Transformer networks for the language modeling task. At the time of writing, the largest conventional model for that task is GPT-3 with 175 billion parameters. Scaling it up 1000 times gives us 175 trillion; depending on whether you use single or half-precision, this requires 300--600 terabytes of memory just to store the model. No modern mass-produced hardware accelerator is up to such task. Even high-end servers with 16x V100 accelerators can store only 0.15\% of that model in combined GPU memory, let alone train it.

The dominant way of growing neural network size has so far been to scale up: deploy more powerful computational accelerators in specialized tightly interconnected clusters. However, this approach will only work up to a point. Models such as T-NLG~\cite{tnlg} and Megatron-LM~\cite{shoeybi2019megatron} were already trained on DGX-SuperPOD --- a supercomputer with hundreds of Tesla V100 GPUs spread over tens of servers. As for GPT-3~\cite{brown2020language}, a single \textit{training run} was estimated to cost 4.6 -- 12 million dollars~\cite{gpt3costlambda,gpt3cost}. 

Even today, the need for costly hardware weighs heavily on the research community. Most researchers cannot contribute to the development of large neural networks because conducting the necessary experiments would be too expensive for them. If we continue to increase the model size by scaling up, eventually the only labs that can conduct competitive research will be those with massive budgets.

However, there is another solution: to scale out. Instead of using a supercomputer, researchers could crowdsource the computation from volunteers with regular PCs. %
This paradigm is known as volunteer computing and was successfully applied to solve problems in biology \cite{larson_crowd}, high energy physics \cite{adam2015atlas} and other subject areas. While a single volunteer PC may be slow and unreliable, the combined floating-point performance of such projects is on par with largest supercomputers \cite{gross_folding}.

The main challenge of volunteer computing is how to utilize this performance. Unlike server pods, consumer-grade PCs communicate over the Internet, which is significantly slower, especially in terms of latency. They are also more prone to failures as they lack many reliability features of their server-grade counterparts. Therefore, volunteer computing was traditionally used for tasks that have high computation to communication ratio and can recover from individual node failures.

Unfortunately, existing paradigms of distributed training require nodes to continuously transfer large amounts of intermediate data \cite{Dettmers20158BitAF,Sun2019OptimizingNP}, making them unsuitable for volunteer computing. In this work, we take a different approach. Instead of adopting the existing distributed training strategies, we identify the advantages of volunteer computing and design a new strategy that capitalizes on them.

We summarize the contributions of our paper as follows:

\vspace{-6px}
\begin{minipage}{0.55\textwidth}

\begin{itemize}[leftmargin=*]
    \item We propose Decentralized Mixture of Experts (DMoE) --- a layer designed for training with vast amounts of unreliable consumer-grade hardware;%
    \vspace{1px}\item We describe a framework for training large neural networks composed of DMoE layers;%
    \vspace{1px}\item We confirm the efficiency and reliability of this approach using formal guarantees and experiments;
    \vspace{1px}\item The PyTorch source code that can be used to reproduce our results is available online\footnotemark.
\end{itemize}
\end{minipage}
\hspace{5px}
\begin{minipage}{0.45\textwidth}
\vspace{-6px}
    \centering
    \raisebox{\dimexpr \topskip-\height}{\includegraphics[width=180px]{resources/teasseract3.pdf}}
    \captionof{figure}{High-level scheme of Decentralized Mixture of Experts. See Section \ref{sect:method} for details.}
    \label{fig:teaser}
\end{minipage}
\footnotetext{\url{https://github.com/mryab/learning-at-home}}
\input{design}
\input{internals}
\section*{Broader Impact}
\label{sect:broader}
\vspace{-4px}

The approach proposed in this work is only a prototype with limited direct consequences, but the long-term goal of training huge models with volunteer computing can have a lasting effect on both the research community and the general public.

\vspace{-6px}
\subsection*{Funding bias vs crowdsourcing bias} 
\vspace{-6px}
The main positive outcome we pursue is to let researchers harness volunteer computing and train models on the scale currently available only to large corporations. Ideally, a deep learning researcher with a promising idea will be able to amass the computation needed to realize this idea by involving volunteers. However, the project's appeal for volunteers depends on many factors such as subject area, current societal trends, and even researcher's personality.

For example, a project about teaching agents to play games~\cite{lc0} or fighting global pandemics~\cite{folding_covid} is likely to attract more resources than deep learning applied to soil science. In essence, volunteer computing is biased towards exciting or socially relevant research the same way as traditional HPC is biased towards the interests of those who fund it.

\vspace{-6px}
\subsection*{Alternative use and misuse} 
\vspace{-6px}
The proposed technology can be used with different economic models. If a deep learning system is immediately useful (e.g. for machine translation, information retrieval, etc), the participants could use it for their needs based on their contributions to training. This can take many forms: several labs combining their hardware and training larger models; a web-service that lets people contribute their compute instead of using ads/subscriptions; or simply a framework that someone can use to run distributed training across two or more datacenters.

Unfortunately, this also allows several opportunities for malicious use. If a machine is hacked, the attacker can use its compute unnoticed by the machine owner --- much the same way that botnets are currently used to mine cryptocurrencies. Furthermore, due to decentalized nature even legitimate Learning@home projects can be hijacked by hackers.

\vspace{-6px}
\subsection*{Security} 
\vspace{-6px}
Using crowdsourced hardware makes Learning@home susceptible to attacks from malicious participants. There are multiple attack vectors already known in P2P community: denial of service attacks, Sybil attacks, Eclipse attacks and more \cite{urdaneta2011survey, sybil_attacks_dht, dos_resistance, sybil_nodes}. Fortunately, there are variations of the DHT protocol that make it resistant to said attacks: if a reader wishes to learn more about DHT security, we recommend starting with \cite{urdaneta2011survey}.

Another source of vulnerability stems from the sequential nature of neural networks. If a single expert were to return incorrect (e.g. NaN) outputs or gradients, it could compromise the outputs of the entire network and even poison adjacent nodes through backpropagation. Recent studies expose similar attack patterns on federated learning systems \cite{bagdasaryan2018backdoor, bhagoji2018analyzing}.

The redundant nature of mixture-of-experts layers provides some degree of resistance against those attacks. A single malicious expert will only affect a small fraction of inputs that pass through this specific expert. Furthermore, a trainer with access to predictions from multiple experts could provide a higher degree of robustness by using statistical techniques (e.g., by ignoring outlier gradients). However, such techniques need to be carefully designed so as not to introduce harmful side effects.

\vspace{-6px}
\subsection*{The burden on the network} 
\vspace{-6px}
Finally, we would like to point out the potential harm that our approach can do to network infrastructure. The experiments we ran in Section \ref{sect:exp_throughput} saturate with the bandwidth of $100-200$Mbps, most of which is tensors passed between experts and trainers. 

This coincides with the typical home internet speed available in major cities of developed countries. However, not all ISPs design their infrastructure for users who always use up all their bandwidth. If too many Learning@home participants are located in one LAN or MAN, it can cause congestion or even failures in the network infrastructure. 

Similar situations frequently took place in late 2000s due to growing popularity of BitTorrent for file sharing. Fortunately, the network infrastructure is continually improving, which leads us to believe that this problem will eventually be solved. Until then, we describe several ways to reduce network load of Learning@home in Appendix E.



\section{Conclusion}

This paper introduces \textsc{Petals}, a system for efficient collaborative inference and fine-tuning of large language models. We offer a user-friendly generation interface and a flexible API to access models served over the Internet. We use 8-bit compression that reduces the resource requirements to run very large models. In addition, we develop algorithms for reliable routing and load balancing.

% Since \textsc{Petals} is open-source, we would like it to evolve based on the community's feedback, incorporating relevant research advances and adding support for features in demand.
With the release of this system, we hope to broaden access to LLMs and pave the road to applications, studies or research questions that were previously not possible or simply too expensive.

% [Commented since the Discussion section has been moved to the main text]
% Running LLMs over the Internet raises a broad range of related questions. One of them is privacy: how to avoid revealing private data to outside peers. Another challenge is to ensure that participants can benefit from this system equitably, i.e. in proportion to their contribution.
% We discuss future problems such as privacy, security, and incentive structures in Appendix~\ref{sect:discussion}.


% \clearpage
% \section*{Limitations}

% The main limitations of our work are related to processing sensitive data in the public swarm, since \textsc{Petals} does not guarantee data privacy and correctness of model outputs in this case. We recommend users working with sensitive data to use only trusted servers or set up an isolated \textsc{Petals} swarm.

% We discuss these limitations in more detail in Appendix~\ref{sect:discussion} and acknowledge that the development of methods for privacy-preserving and secure decentralized inference without performance penalties remains an open research problem.

\section*{Ethics Statement}
This work introduces a general-purpose algorithm for decentralized inference of large models, aiming to simplify access to the latest research in deep learning. Thus, we do not envision any direct negative impacts from our research aside from granting the broader public an ability to interact with LLMs trained on uncurated web-crawled data. However, all models we serve are already in open access and thus can be exposed via APIs or other means.

\section*{Acknowledgements}

The authors thank Zheng-Xin Yong, Ilya Dimov, Yozh, Teven Le Scao, Stas Bekman, and Haokun Liu for helpful discussions. We also thank Teven Le Scao for his help in designing Figure~\ref{fig:overview}. A part of the experiments was conducted on a personal server of Elena Voita.

\clearpage

% Entries for the entire Anthology, followed by custom entries
\bibliography{anthology,custom}
\bibliographystyle{acl_natbib}

\end{document}
