\documentclass[letterpaper]{article}


\usepackage[final,nonatbib]{neurips_2021}




\usepackage[utf8]{inputenc} %
\usepackage[T1]{fontenc}    %
\usepackage{hyperref}       %
\usepackage{url}            %
\usepackage{booktabs}       %
\usepackage{amsfonts}       %
\usepackage{nicefrac}       %
\usepackage{microtype}      %
\usepackage{xcolor}         %
\usepackage{graphicx}
\usepackage{subcaption}
\usepackage{booktabs} %
\usepackage{lipsum}
\usepackage{amsmath}
\usepackage{amssymb, amsthm, latexsym}
\usepackage{multirow}
\usepackage{wrapfig}


\usepackage{algorithm,algorithmic}



\usepackage{enumitem}
\usepackage{caption}
\setlist[itemize]{itemsep=0pt}


\newcommand{\Exp}{\mathbf{E}}
\newcommand{\Prob}{\mathbf{P}}
\newcommand{\R}{\mathbb{R}}
\newcommand{\eqdef}{\stackrel{\text{def}}{=}}
\newcommand{\ve}[2]{\left\langle #1 , #2 \right\rangle}
\def\<#1,#2>{\left\langle #1,#2\right\rangle}

\usepackage{thmtools}

\newtheorem{lemma}{Lemma}[section]
\newtheorem{theorem}{Theorem}[section]
\newtheorem{definition}{Definition}[section]
\newtheorem{proposition}{Proposition}[section]
\newtheorem{assumption}{Assumption}[section]
\newtheorem{corollary}{Corollary}[section]
\newtheorem{remark}{Remark}[section]





\newcommand\tagthis{\addtocounter{equation}{1}\tag{\theequation}}
\newcommand{\argmin}{\mathop{\arg\!\min}}

\newcommand{\cA}{{\cal A}}
\newcommand{\cB}{{\cal B}}
\newcommand{\cC}{{\cal C}}
\newcommand{\cD}{{\cal D}}
\newcommand{\cE}{{\cal E}}
\newcommand{\cF}{{\cal F}}
\newcommand{\cG}{{\cal G}}
\newcommand{\cH}{{\cal H}}
\newcommand{\cJ}{{\cal J}}
\newcommand{\cK}{{\cal K}}
\newcommand{\cL}{{\cal L}}
\newcommand{\cM}{{\cal M}}
\newcommand{\cN}{{\cal N}}
\newcommand{\cO}{{\cal O}}
\newcommand{\cP}{{\cal P}}
\newcommand{\cQ}{{\cal Q}}
\newcommand{\cR}{{\cal R}}
\newcommand{\cS}{{\cal S}}
\newcommand{\cT}{{\cal T}}
\newcommand{\cU}{{\cal U}}
\newcommand{\cV}{{\cal V}}
\newcommand{\cX}{{\cal X}}
\newcommand{\cY}{{\cal Y}}
\newcommand{\cW}{{\cal W}}
\newcommand{\cZ}{{\cal Z}}
\newcommand{\Var}{\mathrm{Var}}

\newcommand{\mA}{{\bf A}}
\newcommand{\mB}{{\bf B}}
\newcommand{\mC}{{\bf C}}
\newcommand{\mE}{{\bf E}}
\newcommand{\mF}{{\bf F}}
\newcommand{\mG}{{\bf G}}
\newcommand{\mH}{{\bf H}}
\newcommand{\mI}{{\bf I}}
\newcommand{\mJ}{{\bf J}}
\newcommand{\mK}{{\bf K}}
\newcommand{\mL}{{\bf L}}
\newcommand{\mM}{{\bf M}}
\newcommand{\mN}{{\bf N}}
\newcommand{\mO}{{\bf O}}
\newcommand{\mP}{{\bf P}}
\newcommand{\mQ}{{\bf Q}}
\newcommand{\mR}{{\bf R}}
\newcommand{\mS}{{\bf S}}
\newcommand{\mT}{{\bf T}}
\newcommand{\mU}{{\bf U}}
\newcommand{\mV}{{\bf V}}
\newcommand{\mW}{{\bf W}}
\newcommand{\mX}{{\bf X}}
\newcommand{\mY}{{\bf Y}}
\newcommand{\mZ}{{\bf Z}}

\newcommand{\sign}{\mathrm{sign}}
\newcommand{\cnorm}{\omega}
\newcommand{\EE}{\mathbb{E}}
\newcommand{\PP}{\mathbb{P}}
\newcommand{\VV}{\mathbb{V}}

\newcommand{\prox}{\mathop{\mathrm{prox}}\nolimits}
\newcommand{\proxR}{\prox_{\gamma R}}
\newcommand{\proxkR}{\prox_{\gamma^k R}}
\newcommand{\mean}{\overline}
\newcommand{\sumin}{\sum_{i=1}^n}


\newcommand{\Mod}[1]{\ \mathrm{mod}\ #1}

\title{Moshpit SGD: Communication-Efficient\\ Decentralized Training\\ on Heterogeneous Unreliable Devices}

\author{%
  Max Ryabinin\thanks{Equal contribution. Correspondence to \texttt{mryabinin0@gmail.com}.} \\
  Yandex, Russia\\
  HSE University, Russia\\
  \And
  Eduard Gorbunov\footnotemark[1]\\
  MIPT, Russia\\
  HSE University, Russia\\
  Yandex, Russia\\
  \And
  Vsevolod Plokhotnyuk\\
  Yandex, Russia\\
  HSE University, Russia\\
  \And
  Gennady Pekhimenko\\
  University of Toronto, Canada\\
  Vector Institute, Canada
}

\begin{document}

\maketitle

\begin{abstract}
Training deep neural networks on large datasets can often be accelerated by using multiple compute nodes. 
This approach, known as distributed training, can utilize hundreds of computers via specialized message-passing protocols such as Ring All-Reduce.
However, running these protocols at scale requires reliable high-speed networking that is only available in dedicated clusters.
In contrast, many real-world applications, such as federated learning and cloud-based distributed training, operate on unreliable devices with unstable network bandwidth.
As a result, these applications are restricted to using parameter servers or gossip-based averaging protocols.
In this work, we lift that restriction by proposing Moshpit All-Reduce --- an iterative averaging protocol that exponentially converges to the global average.
We demonstrate the efficiency of our protocol for distributed optimization with strong theoretical guarantees.
The experiments show 1.3x speedup for ResNet-50 training on ImageNet compared to competitive gossip-based strategies and 1.5x speedup when training ALBERT-large on preemptible compute nodes.
\end{abstract}

\vspace{-12pt}
\section{Introduction}\label{sect:intro}
\vspace{-4pt}

Our investigation begins with a thought experiment. Imagine a deep neural network with capacity 1000 times greater than today's most powerful architectures: for example, a language model trained on all digitally available texts or a generative model for all images ever uploaded to the Internet. How can we train such a model?

\vspace{-1.5pt}

Viewed from a historical perspective, the 1000-fold increase in capacity is not unrealistic. Over the past decade, the deep learning community has made remarkable progress by training large models on abundant data, and the scale of those models keeps growing. Since the advent of the ImageNet challenge \cite{imagenet_cvpr09} with 1.3M labeled images, the typical size of convolutional neural networks increased from a few megabytes to hundreds of megabytes \cite{alexnet, resnet, huang2019gpipe}. Recent studies report even larger models for datasets with hundreds of millions of images \cite{kolesnikovlarge, jft300data}.

\vspace{-1.5pt}

Another trend from natural language processing is to train large Transformer-like language models~\cite{bert, roberta, kaplan2020scaling}. The data for this task is nearly unlimited, allowing researchers to train models with tens or even hundreds of gigabytes of parameters~\cite{brown2020language,shoeybi2019megatron,zellers2019defending,tnlg}. While we may not need the 1000-fold increase at the moment, planning for it will prepare us for the next big leap in model capacity.

\vspace{-1.5pt}

To be specific, let us focus on training large Transformer networks for the language modeling task. At the time of writing, the largest conventional model for that task is GPT-3 with 175 billion parameters. Scaling it up 1000 times gives us 175 trillion; depending on whether you use single or half-precision, this requires 300--600 terabytes of memory just to store the model. No modern mass-produced hardware accelerator is up to such task. Even high-end servers with 16x V100 accelerators can store only 0.15\% of that model in combined GPU memory, let alone train it.

The dominant way of growing neural network size has so far been to scale up: deploy more powerful computational accelerators in specialized tightly interconnected clusters. However, this approach will only work up to a point. Models such as T-NLG~\cite{tnlg} and Megatron-LM~\cite{shoeybi2019megatron} were already trained on DGX-SuperPOD --- a supercomputer with hundreds of Tesla V100 GPUs spread over tens of servers. As for GPT-3~\cite{brown2020language}, a single \textit{training run} was estimated to cost 4.6 -- 12 million dollars~\cite{gpt3costlambda,gpt3cost}. 

Even today, the need for costly hardware weighs heavily on the research community. Most researchers cannot contribute to the development of large neural networks because conducting the necessary experiments would be too expensive for them. If we continue to increase the model size by scaling up, eventually the only labs that can conduct competitive research will be those with massive budgets.

However, there is another solution: to scale out. Instead of using a supercomputer, researchers could crowdsource the computation from volunteers with regular PCs. %
This paradigm is known as volunteer computing and was successfully applied to solve problems in biology \cite{larson_crowd}, high energy physics \cite{adam2015atlas} and other subject areas. While a single volunteer PC may be slow and unreliable, the combined floating-point performance of such projects is on par with largest supercomputers \cite{gross_folding}.

The main challenge of volunteer computing is how to utilize this performance. Unlike server pods, consumer-grade PCs communicate over the Internet, which is significantly slower, especially in terms of latency. They are also more prone to failures as they lack many reliability features of their server-grade counterparts. Therefore, volunteer computing was traditionally used for tasks that have high computation to communication ratio and can recover from individual node failures.

Unfortunately, existing paradigms of distributed training require nodes to continuously transfer large amounts of intermediate data \cite{Dettmers20158BitAF,Sun2019OptimizingNP}, making them unsuitable for volunteer computing. In this work, we take a different approach. Instead of adopting the existing distributed training strategies, we identify the advantages of volunteer computing and design a new strategy that capitalizes on them.

We summarize the contributions of our paper as follows:

\vspace{-6px}
\begin{minipage}{0.55\textwidth}

\begin{itemize}[leftmargin=*]
    \item We propose Decentralized Mixture of Experts (DMoE) --- a layer designed for training with vast amounts of unreliable consumer-grade hardware;%
    \vspace{1px}\item We describe a framework for training large neural networks composed of DMoE layers;%
    \vspace{1px}\item We confirm the efficiency and reliability of this approach using formal guarantees and experiments;
    \vspace{1px}\item The PyTorch source code that can be used to reproduce our results is available online\footnotemark.
\end{itemize}
\end{minipage}
\hspace{5px}
\begin{minipage}{0.45\textwidth}
\vspace{-6px}
    \centering
    \raisebox{\dimexpr \topskip-\height}{\includegraphics[width=180px]{resources/teasseract3.pdf}}
    \captionof{figure}{High-level scheme of Decentralized Mixture of Experts. See Section \ref{sect:method} for details.}
    \label{fig:teaser}
\end{minipage}
\footnotetext{\url{https://github.com/mryab/learning-at-home}}

\vspace{-14px}
\section{Related work}\label{sect:related}
\vspace{-4px}

\subsection{Volunteer computing}\label{sect:related_volunteer}
\vspace{-4px}

Using volunteer hardware has long been a viable alternative to high-performance computing. Since the development of BOINC \cite{anderson2004boinc} research organizations with sufficient public outreach have been able to run massive scientific computations on devices provided by volunteers. Successful projects such as Folding@home can have over $10^5$ active participants, rivaling the floating-point performance of world's fastest supercomputers\footnote{In January 2019, Folding@home reported 146,091 teraflops; in November 2019, the top-1 supercomputer ``Summit'' reported 148,600 teraflops; see \url{top500.org/lists/2019/11} .}. In fact, Folding@home was the first ``supercomputer'' to reach both 1 and 10 petaflops milestones~\cite{folding_timeline}.

However, unlike traditional HPC, the volunteer nature of these projects imposes some additional limitations. First, the majority of volunteers are only available part-time. 
For instance, a participant can provide an office workstation that only contributes compute outside of business hours. 
Second, volunteer hardware is heterogeneous: different nodes may have different performance, memory limits, and even operating systems. Finally, participants usually communicate over the Internet, which is 2--3 orders of magnitude slower than typical HPC connections. As a result, both compute nodes and communication channels are not nearly as reliable as in traditional supercomputers. 

Due to the limitations mentioned above, volunteer computing works best for tasks that can be split into many independent chunks. A single Folding@home task is to run a physical simulation of a protein for a specified number of frames. Together, volunteers can perform hundreds of thousands of concurrent tasks and only need to communicate with the server to submit their results. Other projects like SETI@home and Einstein@home follow a similar pattern.%

Based on the existing volunteer computing projects, we formulate the following usage scenario:
\vspace{-4px}
\begin{itemize}[leftmargin=*]
    \item \textbf{Large pool of weak computers:} the infrastructure consists of $10^3 \sim 10^6$ heterogeneous PCs\footnote{Typical specifications: 2--8 CPU cores, 4--16GB RAM, and a single customer-grade GPU with 2--12GB of memory and 4--14 float32 TFLOPS (based on \url{https://pcpartpicker.com} and \url{https://techpowerup.com})};
    \item \textbf{Communication:} nodes communicate with speed and reliability of a home internet connection\footnote{We assume 20--250ms latency and 100Mbps symmetric bandwidth, $0.33\%$ packet loss based on \cite{speedtest,li2017case}};
    \item \textbf{Frequent node failures:} a compute node may fail to process a task for a variety of reasons. We expect 5--20\% of computers to have at least one failure a day under normal operating conditions.
\end{itemize}
\vspace{-6px}

\subsection{Distributed training}\label{sect:related_distributed}
\vspace{-3px}

To analyze the existing distributed training approaches from the perspective of volunteer computing, we broadly divide them into several categories.

\textbf{Synchronous data parallel training} \cite{valiant1990bridging}\textbf{.} Each worker stores a copy of model parameters, computing gradients for a fraction of the training batch. The gradients are then averaged across workers and applied to the model, making up the same update on all machines. Due to its simplicity and scalability, this method has been widely used to reduce the training time of large neural networks to the order of minutes \cite{goyal2017accurate,You2020Large}. 
    
However, with low-end or midrange hardware it is not always possible to store the entire model on each worker. In addition, gradient communication, even when overlapped with computation, requires a high-speed connection between all participants, often faster than hundreds of megabytes per second, which is unrealistic when considering typical household Internet connections.
    
\textbf{Asynchronous training} \cite{recht2011hogwild, zhang2015staleness} usually involves a single parameter server and multiple compute nodes fetching the latest parameters, processing batches, and submitting updates back to the server. This technique improves worker throughput, but this improvement comes at a cost. If several workers submit simultaneous updates, they might get applied in an arbitrary order, which leads to the issue of \textit{stale gradients} \cite{stale_gradients_can_win} and possibly hinders model convergence.

\textbf{Model parallel training.} Each node stores a fraction of model layers, each training batch is processed by all nodes in a sequential order determined by the layer distribution scheme. The training batch can be divided into several micro-batches and processed in a pipeline fashion, significantly increasing hardware utilization \cite{huang2019gpipe,zero,pipemare,pipedream}.
    
Unlike the two previous paradigms, this method allows training models that exceed the memory limit of any individual worker. Notable examples of successful model parallel training for large neural networks are \cite{huang2019gpipe} and \cite{shoeybi2019megatron}, yet these systems also have a high-speed network between workers. On top of that, model parallelism is highly vulnerable to node and network failures: if a single worker in a chain turns off or stops sending outputs, the training stops entirely. 

It is possible to combine data and model parallelism to mitigate the outlined issues to some degree, but the requirement for fast worker interconnect holds even in that case. In light of this, the method we design has to maintain high throughput even in the presence of slow and unreliable network connections, possibly sacrificing the latency (time to process a given batch) as a necessary tradeoff. 

This constraint may be justified by the following observation: the wall-clock training time of a neural network (with model and optimizer fixed) mostly depends on how many batches it processes per second. As we show in Section \ref{sect:exp_convergence}, the effect of stale gradients can be mitigated with the right architecture. We summarize the desired properties in Table \ref{tab:distributed}.

\begin{table*}[t]
\caption{Comparison of distributed training schemes in the volunteer computing context. ``Desired'' denotes the algorithm with properties that would be beneficial for this setting. ``Only workers'' means that the system has central components that are not fault-tolerant.}
\setlength{\tabcolsep}{3pt}
\hspace{-6pt}\begin{tabular}{cccccccc} 
\toprule
 \multirow{2}{*}{Training method}& Model            & Training       & \multirow{2}{*}{Scalability}    & \multirow{2}{*}{Fault tolerance}             & Worker         & \multicolumn{2}{c}{Network}  \\
         & size limit       & throughput     &                &             & hot-join       & Bandwidth     & Latency                   \\ 
\midrule
Data parallel  & Worker           & \textbf{High } & Medium         & \textbf{Full}         & \textbf{Yes }  & \textbf{High}        & Low                       \\
Asynchronous   & Worker           & \textbf{High } & \textbf{High}  & Only workers\textbf{} & \textbf{Yes }  & Medium        & \textbf{Any}              \\
Model parallel & \textbf{System}  & Medium         & Low            & No                    & No             & High          & Low                       \\
Federated      & Worker           & Low            & \textbf{High}  & Only workers\textbf{} & \textbf{Yes }  & \textbf{Low}        & \textbf{Any}              \\
Desired        & \textbf{System}  & \textbf{High } & \textbf{High}  & \textbf{Full}         & \textbf{Yes }  & \textbf{Low}  & \textbf{Any}              \\
\bottomrule
\end{tabular}
\label{tab:distributed}
\vspace{-12pt}
\end{table*}

\textbf{Federated learning.} The problem of utilizing large quantities of consumer devices for training a single model has also been discussed within the context of data-private learning. Federated learning \cite{mcmahan2017communication} attempts to mitigate the issue by keeping the data on devices, training a local version of the model, and sending only the parameter updates. These updates are encrypted so that the server can only decrypt their average across several devices.

\vspace{-1px}

Unsurprisingly, federated learning sacrifices performance for privacy. Secure aggregation procedures \cite{bonawitz2017practical} require multiple workers to communicate and scale quadratically with their number. These properties hardly align with the scenario from Section \ref{sect:related_volunteer}, making federated learning a poor fit for jointly training large models.

\textbf{Deep learning with volunteer computing.} To the best of our knowledge, there are three projects that use volunteer computing for training neural networks. The first work~\cite{desell2017} leverages volunteer resources for evaluation of CNN architectures generated by evolution algorithms; each model is trained on a single device.
The second study~\cite{volunteer_dl_async} relies on standard asynchronous training and is therefore inapplicable to models that do not fit into a single consumer-grade GPU. Moreover, the architecture described in that study is only partially decentralized, relying on a centralized parameter server that communicates with all nodes. Lastly, the project known as Leela Chess Zero~\cite{lc0}, relies on volunteer hardware to play massive amounts of chess games for generating self-play data used in reinforcement learning. However, the model itself is trained on a single central server.

Our primary insight from this section is that existing methods for training general large neural networks do not fit well into the volunteer computing scenario. However, there is a subclass of deep learning architectures which is much better suited for this task.

\vspace{-2px}
\subsection{Mixture-of-Experts}\label{sect:related_moe}
\vspace{-2px}

Mixture-of-Experts (MoE) was first proposed almost three decades ago as a method to train multiple neural networks (``experts'') for a common task \cite{moe_first}. The intent is for each expert to specialize in making predictions for a small subset of data. Presented with an input, MoE first determines which experts are best suited to process that input using a separate \textit{gating function}. Then it applies the chosen experts and aggregates their outputs into the final prediction. This work has sparked many follow-ups that reveal different MoE structures \cite{jordan1994hierarchical, yao2009hierarchical,moe_lifelong,rasmussen2002infinite} and individual expert types \cite{moe_svm,moe_dirichlet}.

A subsequent study~\cite{eigen2013learning} demonstrates that Mixture-of-Experts can be used as a layer within larger neural networks and trained jointly by backpropagation. Depending on the task, individual experts can utilize convolutional, recurrent, or other specialized layers. Such MoE can have a large number of experts, but it only needs to compute a few of them to process any given input.

Shazeer et al.~\cite{shazeer2017outrageously} (and later~\cite{Lepikhin2020GShardSG}) brought that idea to the extreme by training ``outrageously'' large mixtures with thousands of experts. The drastic increase in capacity allows authors to achieve superior performance in large-scale machine translation and language modeling. The paper also addresses problems that arise with increased mixture size. When trained na\"ively, the gating function learns to use a small fraction of available experts for all inputs, not taking full advantage of the available capacity. The authors alleviate this issue by adding a regularization term that promotes ``load-balancing'' across all experts.

However, scaling this approach from thousands to millions of experts reveals additional problems in the design of a gating function. In order to choose the most appropriate experts for the task, MoE predicts a ``priority'' value for each expert and selects the ones with the highest priority. As the number of experts approaches millions, such a gating function itself becomes computationally intractable, especially in our decentralized setting.

A popular solution to this problem is to structure the set of experts in a search-friendly way. For instance, Hierarchical Mixture-of-Experts~\cite{jordan1994hierarchical} organizes experts in a tree-like structure. Selecting the best experts is then reduced to a beam search over this tree, which scales logarithmically in the number of experts. More recent study by Lample et al. \cite{pkm} explores this idea at scale by organizing over a million keys in a factorized 1024-by-1024 grid. For this grid, the gating function only needs to predict two vectors of size 1024. This work also demonstrates that such layers can benefit Transformer models in the masked language modeling task.

However, these works require a centralized infrastructure for training. When the gating function picks appropriate experts for the input at hand, it must somehow find these experts across all nodes. In our scenario, even maintaining the dynamic ``address book'' of all active experts would be infeasible for any single participant.

\nocite{puigcerver2020scalable}

\vspace{-2px}

\subsection{Distributed Hash Tables}\label{sect:related_dht}

\vspace{-2px}

Fortunately, there is a way to implement bookkeeping in a decentralized system --- the distributed hash table (DHT). This is a family of distributed data structures that store key-value pairs across multiple computers in a network. A single computer within such structure only needs to ``know'' $O(\log N)$ out of $N$ computers; at the same time it can look up any key with at most $O(\log N)$ requests to his peers. There are several DHT variants, but they all have common properties:
\vspace{-4px}
\begin{itemize}[leftmargin=*]
    \item \textbf{Decentralization:} nodes form and maintain DHT without any central coordination;
    \item \textbf{Scalability:} DHT can scale to millions of active nodes that are continually joining and leaving; 
    \item \textbf{Fault tolerance:} a failure in one or a few nodes does not affect DHT integrity and availability;
\end{itemize} 

A DHT-like protocol was first proposed in 1998 by \cite{tewari1998beyond} and popularized in early 2000s by four protocols: CAN~\cite{can}, Chord~\cite{chord}, Pastry~\cite{pastry} and Tapestry~\cite{tapestry}. By far, the most popular DHT variation is Kademlia~\cite{kademlia} with numerous applications such as BitTorrent, I2P, and Ethereum. A more recent work~\cite{kaashoek2003koorde} further improves theoretical performance for either lookup time or the number of connections; however, this version is less widespread due to being significantly harder to implement.


\begin{figure}[ht]
    \centering
    \subfigure[Initialization]{\includegraphics[width=0.32\textwidth]{figures/grad_ratio_init.png}}
    \subfigure[Near Convergence]{\includegraphics[width=0.32\textwidth]{figures/grad_ratio_conv.png}}
    \subfigure[Misclassification]{\includegraphics[width=0.32\textwidth]{figures/grad_ratio_misc.png}}
    \caption{Gradient Ratio}
    \label{fig:grad_ratio}
\end{figure}



\section{Theoretical Analysis and Alternative Loss functions}

In the previous section we described how \Endd can be done by maximising the log-likelihood of the ensemble's output distributions under a conditional Dirichlet Prior. However, we empirically observed significant convergence issues when applying this approach to tasks with large numbers of classes. Thus, in this section we examine the gradients of the Dirichlet NLL loss and propose an alternate training approach which overcomes them.


\textbf{First-Order Analysis}

The setup which will consider in our analysis is the following. First, we have a Prior Network model which is initialized such that it always returns a uniform Dirichlet distribution ($\bm{\alpha} = \bm{1}$), while the target distribution whose probability is being maximized is a sparse K-length vector of probabilities:
\begin{empheq}{align*}
    \bm{\pi}_{tgt} = \big[1-\epsilon, \epsilon/(K-1), \epsilon/(K-1), \cdots \big]^{\tt T},\quad \epsilon = \text{1e-4}
\end{empheq}
Second, we have a Prior Network which is \emph{near convergence} with the following output distribution:
\begin{empheq}{align*}
    \bm{\alpha}_{cnv} =&\ \bm{\pi}_{cnv} \cdot \alpha_0,\ \alpha_0 = 90K,\quad 
    \bm{\pi}_{cnv} =\ \big[1-5\epsilon, \frac{5\epsilon}{K-1}, \frac{5\epsilon}{K-1}, \cdots \big]^{\tt T}
\end{empheq}
Finally, we have a Prior Network which has made a strong mistake, which represents a situation which could occur somewhere in the middle on training, far from convergence:
\begin{empheq}{align*}
    \bm{\alpha}_{msc} =&\ \bm{\pi}_{msc} \cdot \alpha_0,\ \alpha_0 = 90K,\quad 
    \bm{\pi}_{msc} =\ \big[\frac{5\epsilon}{K-1}, \frac{5\epsilon}{K-1}, \cdots, 1-5\epsilon \big]^{\tt T}
\end{empheq}

First, lets consider the standard cross-entropy loss between a predicted and target discrete distribution and it's gradient with respect to the logit $z_k$:
\begin{empheq}{align}
    \mathcal{L}^{\text{CE}} =&\ -\sum_{k=1}^K \hat \pi_k \ln\big(\frac{\alpha_k}{\alpha_0}\big),\quad 
    \frac{\partial\mathcal{L}^{\text{CE}}}{\partial z_k} =\ \frac{\alpha_k}{\alpha_0} - \hat \pi_k 
\end{empheq}

Second, consider the NLL loss of a Dirichlet distribution and its gradient with respect to logit $z_k$:
\begin{empheq}{align}
   \mathcal{L} \small{=} \sum_{k=1}^K\Gamma(\alpha_k) \small{-}(\alpha_k \small{-} 1)\sum_{m=1}^M\frac{\ln\pi_k^{(m)}}{M} \small{-} \Gamma(\alpha_0), \   \frac{\partial\mathcal{L}}{\partial z_k} \small{=} \big(\psi(\alpha_k) \small{-} \psi(\alpha_0)  \small{-}\sum_{m=1}^M\frac{\ln\pi_k^{(m)}}{M}\big) \cdot \alpha_k
\end{empheq}

Finally, consider the dimensionality normalized ratio of the gradient with respect to the logit 1 and logit 2, which represents the relative contribution of the gradients with respect to the class we are interested in modelling to the long tail. 
\begin{empheq}{align}
\begin{split}
        \rho = \frac{1}{K} \Big| \frac{\partial\mathcal{L}}{\partial z_1} \Big| \Big/ \Big|\frac{\partial\mathcal{L}}{\partial z_2}\Big|
\end{split}
\end{empheq}
Figure~\ref{fig:grad_ratio} shows that, at initialization, as the number of classes is increased the standard cross-entropy loss primarily focuses on the high probability class and ignores the long tail. In contrast, the Dirichlet NLL loss displays a diminishing contribution. This means that the loss will focus on modelling the probability distribution of the high-probability classes only after it \emph{perfectly} models the long tail. As the loss is also very sensitive, it means that on complex tasks the model is perpetually stuck modelling the probabilities of tail classes. Note that even near convergence, the ratio $\rho$ is far smaller for the NLL criterion than for discrete cross-entropy. Finally, if a significant error is made on the training data, $\rho$ becomes very large for cross-entropy, and increasingly small for Dirichlet NLL as the number of classes increases. This analysis shows that a desirable property of the loss which ensures good convergence is that the ratio $\rho$ is high and either constant or increasing as the number of classes grows, otherwise the model focuses on modelling the distribution of tail-class probabilities across the ensemble.

An additional issue to consider is that the NLL noise is also noisy, as for each input $\bm{x}$ we only have a few discrete distributions - it may be necessary to use far more samples to get a good estimate of the ensemble's distribution. Furthermore, this distribution may be poorly matched to the Dirichlet, which introduces additional issues. Thus, a natural solution to consider would be to introduce a \emph{Proxy Dirichlet Distribution} to which we can minimize either the \emph{KL-divergence} or \emph{reverse KL divergence}. We leave discussion of the details of the Proxy Dirichlet until later and only consider the gradients which arise from minimizing either loss.  

For this analysis we consider a target Dirichlet distribution with parameters $\bm{\beta} = \bm{\pi}_{tgt}*\beta_0$ where $\beta_0 = 100K$. The explicit forms of the KL-divergence between two Dirichlet distributions, as well the gradient of the forward and reverse KL-divergence are provided below:
\begin{empheq}{align}
\begin{split}
       & \mathcal{L}^{\text{KL}} =\ \ \sum_{k=1}^K\Gamma(\alpha_k) - \sum_{k=1}^K\Gamma(\beta_k) + \Gamma(\beta_0) - \Gamma(\alpha_0) + \sum_{k=1}^K(\beta_k - \alpha_k)\Big(\psi(\beta_k)-\psi(\beta_0)\Big)
\end{split} \\
\begin{split}
       & \mathcal{L}^{\text{RKL}} =\ \ \sum_{k=1}^K\Gamma(\beta_k) - \sum_{k=1}^K\Gamma(\alpha_k) + \Gamma(\alpha_0) - \Gamma(\beta_0) + \sum_{k=1}^K(\alpha_k - \beta_k)\Big(\psi(\alpha_k)-\psi(\alpha_0)\Big)
\end{split} \\
\begin{split}
        &\frac{\partial\mathcal{L}^{\text{KL}}}{\partial z_k} =\ \big(\psi(\alpha_k) - \psi(\alpha_0) - \psi(\beta_k) + \psi(\beta_0)\big) \cdot \alpha_k
\end{split} \\
\begin{split}
        &\frac{\partial\mathcal{L}^{\text{\tiny RKL}}}{\partial z_k} =\ \big((\alpha_k - \beta_k)\psi'(\alpha_k) - (\alpha_0 - \beta_0)\psi'(\alpha_0)\big) \cdot \alpha_k
\end{split}
\end{empheq}

Figure~\ref{fig:grad_ratio} additionally displays the ratio $\rho$ for both the forward and reverse KL-divergence losses. The forward KL-divergence displays the same issues as the NLL loss and $\rho$ continues to decrease as the number of classes in increased. This is unsurprising, as the NLL is equivalent to the KL-divergence in the limit. However, the \emph{reverse KL-divergence} displays the desirable properly that $\rho$ grows and stabilizes as the number of classes is increased. This suggests that if we were to minimize the \emph{reverse KL-divergence} to an appropriately chosen \emph{Proxy-Target Dirichlet distribution}, then we would be able to avoid convergence issues. 

% \textbf{Second-Order Analysis}

% In addition to the first-order analysis provided above, we also conduct a second order analysis by considering the eigenvalues of the Hessian of the loss.


\textbf{Proxy-Dirichlet distribution}

\begin{figure*}[ht]
    \centering
    \subfigure[Naive \Endd]{\includegraphics[scale=0.067]{figures/naive-distillation-2.png}}
    \subfigure[Proxy \Endd]{\includegraphics[scale=0.067]{figures/proxy_distillation.png}}
    \caption{Schematic of Distillation Approaches}
    \label{fig:distillation overview}
\end{figure*}

It is important to remember that the ensemble may be poorly modelled via a Dirichlet distribution, so it is necessary to ask which properties of the ensemble we are actually interested in capturing. Clearly, we would like to capture the mean of the ensemble, as that typically has better predictive accuracy and calibration. Additionally, we would like to capture \emph{bulk-diversity properties} of the ensemble, such that the measures of divergence derived from the Proxy Dirichlet are similar to those of the original ensemble and therefore provide a similar rank-ordering of data. At the same time, we are \emph{not} interested modelling properties like multi-modality and skew. 

Clearly, obtaining the mean of the ensemble is trivial. Obtaining an estimate of the precision $\beta_0$ is more challenging. One approach based on Sterling's approximation is described in~\cite{minka2000estimating} and proposes the following estimate:
\begin{empheq}{align}
\begin{split}
        \hat \pi_k (\bm{x})=&\ \frac{1}{M}\sum_{m=1}^M {\tt P}(y=\omega_k|\bm{x}, \bm{\theta}^{(m)}) \\
        \tilde \beta_0(\bm{x}) =& \frac{K-1}{2 \sum_{k=1}^K\hat \pi_k (\ln \hat \pi_k - \frac{1}{M}\sum_{m=1}^M\ln \pi_k^{(m)})},\ \bm{\beta}_k (\bm{x}) = \ \hat \pi_k(\bm{x}) \cdot \tilde \beta_0(\bm{x}) + 1
\end{split}
\end{empheq}

We found that it is important to also add 1 to all the target concentration parameters. Figure~\ref{fig:grad_ratio_smooth} shows that for the reverse KL loss, adding 1 to \emph{both} the target Proxy-Dirichlet as well as \emph{the model} yields an improved ratio $\rho$ both at initialization and near convergence. Heuristically, it seems to make the loss more linear and stable by preventing the digamma and trigamma functions $\psi$ and $\psi'$ in the reverse-KL loss from dropping into the highly non-linear regime when $\alpha_k < 1$ and $\beta_k < 1$.
\begin{figure}[ht]
    \centering
    \subfigure[Initialization]{\includegraphics[scale=0.49]{figures/grad_ratio_init_smooth.png}}
    \subfigure[Near Convergence]{\includegraphics[scale=0.49]{figures/grad_ratio_conv_smooth.png}}
    \caption{Gradient Ratio}
    \label{fig:grad_ratio_smooth}
\end{figure}

Note, while the solution may seem to similar to work done in \cite{malinin-rkl-2019}, the fundamental underlying reason for using this loss is altogether different. Here, the issue is due to large gradients from low-probability tail classes, while in~\cite{malinin-rkl-2019} the reverse KL loss is used to avoid inducing a multi-modal target Dirichlet distribution in expectation. 

\begin{empheq}{align}
\begin{split}
{\tt KL}[{\tt p}(\bm{\pi}|\bm{x},\bm{\theta}) \| {\tt p}(\bm{\pi}|\bm{\hat \beta})] =&\ \underbrace{ \beta_0\cdot\mathbb{E}_{{\tt p}(\bm{\pi}|\bm{x},\bm{\theta})}\big[-\sum_{k=1}^K\hat \pi_k\ln \pi_k\big]}_{\text{Reconstruction term}} + \underbrace{{\tt KL}[{\tt p}(\bm{\pi}|\bm{x},\bm{\theta}) \| {\tt p}(\bm{\pi}|\bm{1})]}_{\text{Prior}} +Z
\end{split}
\end{empheq}
% \textbf{Alternative solutions (if it fits) }

% If not, we'll move that to the appendix (along with comparisons)

% \begin{itemize}
%     \item Top-k aggregation
%     \item Softplus parametrization
% \end{itemize}

\section{Experiments}
\label{sec:experiments}

In this section, we evaluate \Endd via minimization of Reverse KL-divergence between the model and a Proxy Dirichlet target. We apply distribution distillation to ensembles of convolutional networks trained on the ImageNet dataset and to ensemble of Transformer models trained on WMT'17 En-De. Our goal here is to demonstrate that given an ensemble, we can successfully distribution-distill it into a single model. Note that we do not provide results for \Endd accomplished by optimizing Dirichlet NLL or forward KL-divergence, because we could not get them to even begin to converge on the tasks considered here. 

\subsection{Setup}
\label{sec:experiments_setup}
We consider two large-scale tasks involving classification: 1000-class image classification and sequence-to-sequence modeling of natural language. For each task, we first train the ensemble of regular models and then distill it with \Endd. For comparison, we also report the average single-model performance along with the following baselines:

\begin{itemize}
\item \textbf{Ensemble} refers to the performance of an ensemble of independently trained models, which was previously shown to yield high quality uncertainty estimates~\cite{deepensemble2017} and to outperform more sophisticated methods using only a few models~\cite{ashukha2020pitfalls}.
\item \textbf{Ensemble Distillation} (EnD) is a common approach to model and ensemble distillation, first proposed in~\cite{hinton2015distilling}. It involves training the student model with the soft target distribution of averaged ensemble predictions. Notably, we do not add the cross-entropy loss for ground truth labels, because we focus on the comparison of distillation objectives and not only classification performance.
\end{itemize}

% TODO discuss why not MC-dropout?
We do not use Hydra~\cite{hydra} or similar multi-head approaches for distilling each separate ensemble member, because with a high number of models in the ensemble and even 1000 classes the computation overhead is no longer negligible. In all experiments with \Endd, we add 1 both to the predicted parameters of the Dirichlet distribution and the Dirichlet proxy parameters.
% : we evaluate the performance of versions without these modifications in Section~\ref{sec:experiments_ablation}

Both for error rejection and out-of-distribution detection, we use several information-theoretic measures uncertainty; in particular, we use entropy of the expected predictive distribution (EoE) for total uncertainty and Reverse Mutual Information (RMI) for knowledge uncertainty throughout this section.
Derivations of these measures both for \Endd and ensembles are available in~\cite{malinin-thesis} and~\cite{malinin-structured-2020}.
For Single and EnD single-model baselines, we use
entropy of the output distribution as the only valid uncertainty estimate.
% the same measures of uncertainty by interpreting exponents of logits as parameters of a Dirichlet distribution. 
% As we show later, in some setups the performance of such models can be surprisingly competitive with that of \Endd and even ensembles; we leave the study of this phenomenon to future work.

\subsection{Large-scale image classification}
\label{experiments:imagenet}

For the first experiment, we run distillation of the ensemble that contains 10 ResNet-50~\cite{resnet} models trained on the ImageNet~\cite{imagenet} image classification dataset. We use the standard training setup outlined in~\cite{touvron2019FixRes}; specifically, we train for 90 epochs using stochastic gradient descent with momentum of 0.9 and a learning rate of $0.1\times B/256$ (first proposed in~\cite{goyal2018accurate}), where B is the per-device batch size multiplied by the number of GPUs. 
In our experiments, we use a single-GPU batch size of 256 and 8 NVIDIA V100 GPUs. The learning rate is divided by 10 every 30 epochs. For data augmentations, we use a standard combination of random resized crops and horizontal flips implemented in the Albumentations library~\cite{albumentations}.
In all experiments, we found it beneficial to initialize the last batch normalization $\gamma$ in each residual branch to zero, which agrees with previous results~\cite{goyal2018accurate, zhang2018residual, rezero}.

For a thorough evaluation of all methods, we use several different characteristics of performance. First, we measure the in-domain classification accuracy on the original ImageNet validation subset~\cite{imagenet}, which is commonly used for comparison of image classification models. Second, we compare the robustness of all approaches to different domain shifts, also measured by accuracy on datasets corresponding to these shifts. In particular, we use adversarial examples from ImageNet-A~\cite{hendrycks2021nae}, corrupted and perturbed versions of original ImageNet validation data from ImageNet-C~\cite{hendrycks2019robustness}, and artistic renditions from ImageNet-R~\cite{hendrycks2020many}. Next, these domain shift and the original validation dataset are used to compare calibration of models with Expected Calibration Error (ECE).
Finally, we measure the out-of-distribution detection error in terms of Receiver Operating Characteristic area under curve (ROC AUC) on the domain shift datasets together with ImageNet-O~\cite{hendrycks2021nae}.

% \begin{itemize}
% \item In-domain classification accuracy
% \item Robustness is specifically classification accuracy for out-of-domain data
% \item Expected calibration error (ECE)
% \item Out-of-domain detection error: , measured in terms of the area under the ROC AUC curve
% \end{itemize}

% Finally, we also evaluate the out-of-distribution detection error, measured in terms of Receiver Operating Characteristic area under curve (ROC AUC).


We report the results for all metrics in Tables~\ref{tab:imagenet_pred} and~\ref{tab:imagenet_ood} for prediction quality and out-of-distribution detection respectively.
Here, the metrics on ImageNet-C are averaged over all degrees of corruption; in Figure~\ref{fig:imagenet_breakdown}, we provide the detailed results of evaluation on each degree separately.
For out-of-distribution detection, we also provide the results of the Dirichlet Proxy to verify that this approximation of the ensemble predictive distribution does not significantly affect its performance.

Table~\ref{tab:imagenet_pred} shows that \Endd is capable of accurate emulation of the ensemble in terms of classification performance: in terms of accuracy, the method displays results on par or slightly better than regular distillation while also having smaller calibration errors. Also, in Table~\ref{tab:imagenet_ood}, it can be seen that for most datasets (except the hardest ImageNet-O) Proxy-Dirichlet distillation can closely match the out-of-distribution performance of the ensemble. As expected, both distillation methods outperform training a single model from scratch while having the same computational complexity.

Furthermore, Figure~\ref{fig:imagenet_breakdown} shows that as the domain shift increases, all models suffer from a drop in accuracy and calibration quality; notably, EnD and \Endd have the same calibration performance on original data, but Dirichlet network distillation has lower calibration errors for the highest degrees of corruption. Unsurprisingly, the further the data is from the original training images, the better the models are at out-of-distribution detection.

\begin{table}
\centering
\small
\caption{Prediction quality results for image classification.}
\label{tab:imagenet_pred}
\begin{tabular}{lcccccccc}
\toprule
{} & \multicolumn{2}{c}{ImageNet-val} & \multicolumn{2}{c}{ImageNet-A} & \multicolumn{2}{c}{ImageNet-C} & \multicolumn{2}{c}{ImageNet-R} \\
{} &          Acc &      ECE &        Acc &       ECE &        Acc &       ECE &        Acc &       ECE \\
\midrule
Single   &     75.9±0.1 &  4.8±0.1 &    4.4±0.2 &  51.1±0.3 &   39.1±0.7 &  11.3±0.7 &   35.0±0.2 &  21.3±0.4 \\
Ensemble &         79.0 &      2.3 &        3.9 &      42.0 &       43.5 &       4.5 &       38.8 &       9.8 \\
EnD      &         77.0 &      1.6 &        3.8 &      46.6 &       40.6 &       5.9 &       36.9 &      16.1 \\
\Endd    &         77.1 &      1.6 &        3.9 &      42.8 &       40.6 &       4.5 &       37.0 &      11.8 \\
\bottomrule
\end{tabular}
\end{table}

\begin{table}
\centering
\small
\caption{Out-of-distribution detection results for image classification.}
\label{tab:imagenet_ood}
\begin{tabular}{lcccccccc}
\toprule
{} & \multicolumn{2}{c}{ImageNet-O} & \multicolumn{2}{c}{ImageNet-A} & \multicolumn{2}{c}{ImageNet-C} & \multicolumn{2}{c}{ImageNet-R} \\
{} &        EoE &   RMI &        EoE &   RMI &        EoE &   RMI &        EoE &   RMI \\
\midrule
Single   &   50.7±0.3 &     - &   85.8±0.1 &     - &   79.9±0.4 &     - &   83.0±0.2 &     - \\
Ensemble &       54.6 &  62.7 &       88.8 &  86.7 &       82.0 &  77.5 &       86.1 &  84.1 \\
Proxy    &       54.6 &  62.9 &       88.8 &  86.5 &       82.0 &  77.3 &       86.1 &  84.0 \\
EnD      &       48.4 &     - &       87.2 &     - &       80.8 &     - &       83.9 &     - \\
\Endd    &       52.0 &  53.2 &       86.8 &  84.6 &       80.1 &  76.9 &       83.7 &  81.4 \\
\bottomrule
\end{tabular}
\end{table}

\begin{figure}
    \centering
    \includegraphics[width=\textwidth]{figures/breakdown.pdf}
    \caption{Performance of image classification models depending on the level of ImageNet-C corruption.  No corruption corresponds to the original ImageNet validation data.}
    \label{fig:imagenet_breakdown}
\end{figure}

\subsection{Machine translation}
\label{experiments:nmt}
For this experiment, we train standard Transformer-big~\cite{vaswani2017attention} models on the WMT'17 English-German machine translation dataset with the vocabulary of 40,000 Byte-Pair Encoding tokens~\cite{sennrich-etal-2016-neural}. Each of the 10 ensemble members is trained with the setup described in~\cite{ott2018scaling}: in particular, we train them for 193,000 steps with Adam~\cite{adam} on 8 NVIDIA V100 GPUs with a batch size of 4096 tokens per GPU. We train all distillation models for 20,000 steps with the increased batch size of 32K tokens. Because our approach requires fitting all 10 ensemble members in GPU memory, we reduce the immediate batch size for each step to 1024, but compensate for it with gradient accumulation over 32 steps. For output generation and estimation of uncertainty measures (where applicable), we use beam search with beam size 5.

To compare the approaches in terms of translation quality, we use the BLEU score~\cite{papineni2002bleu} computed with SacreBLEU~\cite{sacrebleu} and sequence-level Prediction Rejection Ratio~\cite{malinin-thesis} on the newstest14 English-German test set. For out-of-distribution detection, we also compute ROC AUC and use several datasets with different characteristics and degrees of domain shift: sentences with permuted tokens in the input, LibriSpeech~\cite{librispeech} test-clean speech transcriptions, and source sentences from newstest14 in German and French languages respectively. We average the results of both distillation methods over 5 random seeds and provide standard deviations of all metrics.

\begin{table}
\centering
\small
\caption{Prediction quality results for machine translation.}
\label{tab:wmt_pred}
\begin{tabular}{lccc}
\toprule
{} &      BLEU &       EoE &       RMI \\
\midrule
Single   &  28.8±0.1 &  36.0±1.3 &  - \\
Ensemble &      30.1 &      30.2 &      26.0 \\
EnD      &  29.4±0.1 &  35.6±0.4 &  - \\
\Endd    &  29.5±0.1 &  35.9±0.8 &  35.8±0.5 \\
\bottomrule
\end{tabular}
\end{table}

\begin{table}
\centering
\small
\caption{Out-of-distribution detection results for machine translation.}
\label{tab:wmt_ood}
\begin{tabular}{lcccccccc}
\toprule
{} & \multicolumn{2}{c}{Permuted} & \multicolumn{2}{c}{Speech} & \multicolumn{2}{c}{German} & \multicolumn{2}{c}{French} \\
{} &       EoE &       RMI &       EoE &       RMI &       EoE &       RMI &       EoE &       RMI \\
\midrule
Single   &  80.7±1.5 &         - &  73.7±1.2 &         - &  32.8±2.8 &         - &  27.1±6.3 &         - \\
Ensemble &      83.7 &      97.4 &      67.8 &      73.7 &      39.5 &      82.4 &      25.0 &      73.6 \\
EnD      &  79.5±1.1 &         - &  75.9±0.6 &         - &  35.4±1.6 &         - &  15.6±3.2 &         - \\
\Endd    &  78.3±1.6 &  97.1±0.3 &  77.0±0.3 &  78.5±0.2 &  38.3±1.6 &  70.9±0.7 &  15.9±3.0 &  60.1±3.6 \\
\bottomrule
\end{tabular}
\end{table}

Table~\ref{tab:wmt_pred} further confirms the findings made in the previous section: \Endd via Dirichlet-Proxy outperforms regular ensemble distillation in terms of translation quality and sequence-level error detection. Furthermore, in Table~\ref{tab:wmt_ood} we see that, compared to image classification, the OOD performance gap between total uncertainty and knowledge uncertainty is significantly larger. This might be explained by a significantly larger output space (40,000 classes instead of 1000) or the sequential nature of NMT predictions: because the model generates candidates in a large output space of all possible sequences, its prediction entropy might be high regardless of presence of a domain shift.

% \subsection{Ablation study}
% \label{sec:experiments_ablation}


% \begin{table}[t]
% \centering
% \begin{tabular}{@{}lllll@{}}
% \toprule
%  &  & \multicolumn{3}{l}{OOD detection} \\
%  & Accuracy & Imagenet-C & Imagenet-R & Imagenet-A \\ \midrule
% END\textasciicircum{}2 &  &  &  &  \\
% END\textasciicircum{}2+RKL mediator etc &  &  &  &  \\ \midrule
% - target smoothing &  &  &  &  \\
% - shifted parametrization &  &  &  &  \\
% RKL -\textgreater Forward KL &  &  &  &  \\ \bottomrule
% \end{tabular}%
% \end{table}

% Several additional modifications are given in the Appendix\ref{TODOappendix_ablation}

\vspace{-6pt}
\section{Conclusion and future work}
\vspace{-4pt}
In this work, we propose Moshpit All-Reduce, a decentralized averaging protocol intended for distributed optimization in unstable and network-constrained environments. It has favorable theoretical properties when compared to gossip-based approaches and achieves considerable speedups in distributed training for image classification and masked language modeling.

Our approach was primarily designed for cloud-based training and federated learning, as well as for distributed training on unreliable instances; future work might explore additional settings, such as collaborative training of neural networks.
Another potential research direction is to study the interactions of Moshpit All-Reduce with other methods that improve communication efficiency of distributed optimization, such as gradient compression.
Finally, the idea of arranging All-Reduce nodes into groups can be improved to address specific issues that may arise in practice, such as the varying number of workers and their geographical distribution. 

\vspace{-6pt}
\section*{Acknowledgements}
\vspace{-4pt}
We would like to thank Anastasia Koloskova, Liudmila Prokhorenkova and Anton Osokin for helpful feedback and discussions. We are also grateful to the anonymous reviewers for their suggestions on improving the paper. Finally, we would like to thank Dmitry Afanasiev, Vladimir Aliev, Anand Jayarajan and Michael Solotky for their suggestions on the technical aspects of our study. 
This project was supported in
part by the Canada Foundation for Innovation JELF grant,
NSERC Discovery grant, AWS Machine Learning Research
Award, and Facebook Faculty Research Award. The paper was also partially supported by by a grant for research centers in the field of artificial intelligence, provided by the Analytical Center for the Government of the Russian Federation in accordance with the subsidy agreement (agreement identifier 000000D730321P5Q0002) and the agreement with the Moscow Institute of Physics and Technology dated November 1, 2021 No. 70-2021-00138. The computational resources for the experiments were provided by the Amazon Research Awards program and Yandex.

\bibliographystyle{unsrt}
\bibliography{bibliography}










\clearpage
\part*{Supplementary Material}
\appendix

\section{GPU instance costs}
\label{sect:cloud_costs}

This section provides a brief cost analysis of typical deep learning compute resources both in the cloud and on-premises.
For brevity, we limit this analysis to the popular GPUs available at the time of submission. Note that the exact costs will depend on a variety of factors such as the cloud provider, the region, electricity costs, and market fluctuations. Therefore, we warn the reader to consider this analysis only as a rough estimate. 

Specifically, we estimate the compute costs for the occasional usage scenario: running a single set of experiments over several weeks or conducting infrequent experiments. This scenario covers most research scientists and small organizations. The most straightforward way to provision a GPU server in such a scenario is to rent it from a cloud provider (e.g., GCP or AWS) or a public marketplace (e.g., Vast.ai or Golem).

While the exact server specifications vary from one provider to another, there are two broad categories of GPU machines: regular and preemptible. Regular instance types typically offer 1--8 GPUs per node with tight uptime guarantees (typically $99.99\%$) and a high-bandwidth network (tens of Gb/s). In turn, preemptible instances provide the same resource type at a significant discount with the condition that the machine can be terminated at any time after short notice.

To account for individual variations, we report the average rent price over three popular cloud providers.
We consider three popular instance types: two high-end instances with 8 Tesla V100 or A100 GPUs and a low-end instance with a single Tesla T4 GPU.
We also describe several low-end servers and workstations available on a public marketplace. Unlike cloud VMs, these instances are hosted on non-curated hardware with less uptime guarantees (typically 95\% -- 99.9\%), slower network and significant variation in performance. However, marketplace instances are the cheapest in terms of cost per TFLOPS. To quantify this, we report the average over three most affordable instances that fit the chosen minimum requirements.

As a point of comparison, we also measure each system's training performance for BERT-Large~\cite{bert} fine-tuning on SQuAD v1.1~\cite{squad} in PyTorch with mixed precision. We follow the official benchmarking protocol by~\cite{nvidia_perf} and reuse the official performance results for V100, A100, and T4 instances. The only exception is GTX 1080Ti, where we use full 32-bit precision because that device does not support efficient half-precision operations.

\begin{table}[h]
\small
\setlength{\tabcolsep}{2pt}
\renewcommand{\arraystretch}{1}
\centering
\caption{Cloud and marketplace GPU instance pricing for short-term usage.}
\label{fig:cloud_costs}
\begin{tabular}{@{}ccccccc@{}}
\toprule
\multicolumn{4}{c}{Minimum system specifications} & \multicolumn{2}{c}{Average cost, \$/hour} & \multirow{2}[2]{*}{\shortstack{BERT-Large\\ training samples/s}} \\
\cmidrule(lr){1-4}\cmidrule(lr){5-6}
GPU & CPU cores & CPU type & RAM, GB & Regular & Preemptible &  \\ \midrule
\multicolumn{7}{c}{Cloud instances} \\ \midrule
8$\times$ V100 & 64 & Intel Xeon Broadwell & 480 & 23.47 & 7.13 & 354 \\
8$\times$  A100 & 96 & AMD Epyc ROME & 960 & 30.65 & 10.18 & 755 \\
1$\times$  T4 & 4 & Intel Xeon Cascade Lake & 16 & 0.46 & 0.18 & 18 \\ \midrule
\multicolumn{7}{c}{Marketplace instances} \\ \midrule
6$\times$ 3090 & 32 & AMD Epyc Rome & 480 & 5.04 & 4.17 & 154 \\
4$\times$  2080Ti & 16 & Intel Xeon Haswell & 240 & 0.96 & 0.84 & 83.4 \\
1$\times$  RTX 1080Ti & 8 & Intel Xeon Haswell & 16 & 0.22 & 0.16 & 12 \\ \bottomrule
\end{tabular}
\end{table}

Table~\ref{fig:cloud_costs} shows two main tendencies. First, preemptible \textit{cloud} instances are, on average, three times cheaper than their non-preemptible counterparts\footnote{The cost can be up to $11{\times}$ cheaper for some instance types, e.g. Azure V100 instances in the central US region at the time of writing.}. Second, the high-end HPC-grade servers that offer the highest raw performance are less cost-effective than lower-tier servers and marketplace instances. In theory, one could match the raw floating-point performance of a $8{\times}$V100 instance at a fraction of its cost using multiple lower-tier workstations, such as $4{\times}$ RTX 2080Ti, with a smaller total cost.
However, in practice, running distributed training with these workstations is challenging due to their unreliability and slow network connection.

Note that this analysis does not represent the cloud costs for sustained GPU usage. If an organization plans to constantly use GPU resources over a period of multiple years, they can reduce the costs by deploying their own compute infrastructure or relying on the sustained usage discounts reaching up to 60--70\%. Thus, the long-term compute costs are much harder to analyze and depend on a number of additional factors, such as local electricity prices for on-premise infrastructure. However, this scenario offers similar trade-offs: HPC-grade infrastructure offers greater interconnectivity, but requires expensive network interface cards, high-end switches and a more complex setup process.

\section{Additional Related Work}
\label{sect:post_related}

In this section, we review some of the papers relevant to our work, but omitted from the main part due to space constraints. 

\subsection{Decentralized training}\label{sect:post_related_gossip}
In this subsection, we give additional details about the dependence of gossip-based optimization methods on the spectral properties on the communication graph through the spectral properties of the mixing matrix~\cite{xiao2004fast,scaman2019optimal} or the Laplacian matrix~\cite{merris1994laplacian,uribe2020dual} of the network. 
That is, gossip finds approximate average on nodes with accuracy $\varepsilon$ after $\cO\left((1-\lambda_2(\mM))^{-1}\log(\varepsilon^{-1})\right)$ iterations, where $\mM$ is the mixing matrix and $\lambda_2(\mM)$ is the second largest eigenvalue of $\mM$ when sorted by absolute value. 
The quantity $\eta = 1-\lambda_2(\mM)$ is called the spectral gap of the mixing matrix $\mM$, and $\eta^{-1}$ is typically a polynomial of the total number of nodes $N$ when the maximal degree of the node is $\cO(1)$. For example, for uniformly averaging $\mM$ one can show that $\eta^{-1} = \cO(N^2)$ for the ring topology (node degree $2$), $\eta^{-1} = \cO(N)$ for the two-dimensional torus topology (node degree  $2$), and $\eta^{-1} = \cO(1)$ for the fully connected graph (node degree $N-1$); one can find more examples in~\cite{aldous2002reversible}. Similarly, the communication complexity of decentralized optimization methods often has multiplicative dependence on either $\cO(\eta^{-1})$ (see~\cite{xu2020distributed} and references therein) or $\cO(\eta^{-\nicefrac{1}{2}})$~\cite{scaman2019optimal,uribe2020dual,fallah2019robust,kovalev2020optimal}, which is not improvable for gossip-based methods~\cite{arjevani2015communication,scaman2017optimal}.

Contrary to this, Moshpit All-Reduce does not depend on a fixed communication graph and the properties of its mixing matrix.
However, it depends on the number of averaging groups and the total number of peers (see Theorem~\ref{thm:quality_of_avg_deterministic_vectors}), which can be viewed as properties of a time-varying random communication graph. Fortunately, this dependence is often much better than in gossip: as we mentioned in the main part of the paper, even if workers are randomly split into pairs at each iteration, the simplified version of Moshpit All-Reduce makes the average distortion (the left-hand side of Equation~\ref{eq:determ_quality_of_avg}) at least $2$ times smaller after each round on average.

\subsection{Compressed communication}
Another popular approach to address the communication bottleneck is communication compression~\cite{seide20141,alistarh2017qsgd,suresh2017distributed, ramezani2021nuqsgd, faghri2020adaptive}: before sending any information (e.g., iterates, gradients, Hessians or more sophisticated data) over the network, peers compress this information by applying a possibly random transformation. As the result, peers send fewer bits for each communication round, but the total number of communication rounds needed to achieve the predefined accuracy of the solution increases. However, compression can be useful in situations when the reduction in communication costs of one round is more important than the increase in the number of these rounds~\cite{horvath2019natural}.

There are two distinct groups of works on distributed training with compressed communication: ones that focus on unbiased compression operators (e.g., Rand-K, $\ell_p$-quantization) and ones studying algorithms with biased compressors (e.g., Top-K); see a detailed summary of  popular compression operators in~\cite{beznosikov2020biased}). 
Quantized SGD (QSGD)~\cite{alistarh2017qsgd} and TernGrad~\cite{wen2017terngrad} were among the first compression methods with convergence guarantees. Next, the convergence analysis of these methods was generalized and tightened in the (strongly) convex case in~\cite{mishchenko2019distributed}. Moreover, the authors of \cite{mishchenko2019distributed} proposed a modification of QSGD called DIANA: this algorithm is based on the quantization of gradients' differences, which helps it achieve linear convergence in the strongly convex case when peers compute full gradients. Next, DIANA was generalized to arbitrary unbiased compression in~\cite{horvath2019stochastic}, where authors also developed and analyzed the variance-reduced version of DIANA. After that, several further modifications, such as Accelerated DIANA~\cite{li2020acceleration} and DIANA with bidirectional compression~\cite{gorbunov2020linearly,philippenko2020artemis}, were proposed. Finally, we refer the reader to~\cite{li2020unified,haddadpour2020federated,das2020improved, pmlr-v139-gorbunov21a} for state-of-the-art results for distributed methods with unbiased compression in the non-convex case.

However, naïve application of biased compression operators can lead to significantly worse performance in practice. For instance, as it was shown recently in~\cite{beznosikov2020biased}, parallel SGD with Top-1 compression can diverge exponentially fast. Therefore, biased compressors are used jointly with so-called error-compensation~\cite{seide20141}. The first analysis of Error-Compensated SGD (EC-SGD) was proposed in~\cite{stich2018sparsified,karimireddy2019error} which then was generalized and tightened in~\cite{beznosikov2020biased}. Next, several further improvements, such as an accelerated version of EC-SGD~\cite{qian2020error} and linearly converging EC-SGD~\cite{gorbunov2020linearly}, were recently proposed. However, current theory does not show any superiority of distributed methods with biased compressors to the ones with unbiased compression operators.
In addition, one can combine decentralized communication with compression. Such combinations with unbiased compression operators were studied in~\cite{reisizadeh2019exact,kovalev2020linearly} and with biased operators in~\cite{pmlr-v97-koloskova19a,Koloskova2020Decentralized}.
In this paper, we do not study the interaction of different compression methods and Moshpit Averaging, leaving this promising direction to future work.

\subsection{Multiple local steps}
Alternatively, to reduce the impact of the communication bottleneck, it is possible to perform several local optimization steps on each peer between the communication rounds.
This approach is based on the idea that the increased computational load of peers will decrease the number of communication rounds required to obtain the optimal parameters; it is frequently used in federated learning~\cite{konevcny2016federated,kairouz2019advances}. In particular, one of the most popular methods with multiple local steps is called Local-SGD or Federated Averaging~\cite{konevcny2016federated,Stich18local}. The first results on its convergence were given in \cite{Stich18local,LinSPJ2018local}, and later they were tightened and generalized both for homogeneous~\cite{khaled2020tighter,woodworth2020local} and heterogeneous  cases~\cite{khaled2020tighter,woodworth2020minibatch}. Recently, further modifications of Local-SGD were proposed and analyzed: these modifications include acceleration \cite{yuan2020federated}, variance reduction \cite{gorbunov2020local}, communication compression \cite{basu2019qsparse,haddadpour2020federated,das2020improved}, decentralization \cite{li2019communication,koloskova2020unified}, adaptive and proximal methods \cite{reddi2021adaptive,yuan2020federated_comp}, and resistance to client drift \cite{karimireddy2020scaffold}.
Moshpit SGD can perform multiple local gradient steps before synchronization by design, as shown in Algorithm~\ref{alg:moshpit_local_sgd}.


\subsection{Asynchronous methods}
In the previous subsections, we mostly discussed synchronous distributed methods, since they are more widespread and better studied than asynchronous ones. Mainly, this is because asynchronous methods are more difficult to implement, debug and analyze under general assumptions. However, such methods can be more efficient in terms of using computational resources, which leads to faster wall-clock convergence \cite{assran2020advances}. In recent years, several asynchronous stochastic methods~\cite{recht2011hogwild,zhao2016fast,leblond2017asaga}, methods with no shared memory~\cite{peng2016arock,mishchenko2018delay}, and methods with delayed updates~\cite{agarwal2011distributed,feyzmahdavian2016asynchronous,arjevani2020tight,gorbunov2020linearly} were proposed and analyzed: one can find more details in a recent survey~\cite{assran2020advances}.
Moshpit SGD belongs to this family of asynchronous approaches as well, because the averaging steps happen in smaller groups and can be interleaved with local parameter updates.

\subsection{Distributed Hash Tables}
\label{sect:related_dht}

In this work, we set out to improve distributed averaging with a dynamic matchmaking protocol. Without a central server, this protocol relies on decentralized data structures to organize peers. The main data structure we use is the Distributed Hash Table, or DHT. On a high level, DHT is a distributed fault-tolerant ``dictionary'' that can be accessed by every participant. Each key-value pair is stored on a subset of peers determined by the $\mathrm{hash}$ function of the key.

Each participant has a unique identifier (ID) sampled uniformly from the $\mathrm{hash}$ function output range. When storing a $(key,\ value)$ pair, one must find $k$ peers whose IDs are nearest to $\mathrm{hash}(key)$ according to a chosen metric. After that, the participant requests each of those peers to store $(key,\ value)$. When retrieving a value for a key, one should compute $\mathrm{hash}(key)$, search for peers with IDs nearest to that $\mathrm{hash}$ value and request the value from those peers.

Specific DHT versions, such as Chord~\cite{chord} or Kademlia~\cite{kademlia}, employ different hash types and algorithms for finding nearest peers. For instance, Kademlia DHT sorts peers based on the XOR distance function: $d(x, y) = \mathrm{int}(x \oplus y)$.

In DHT, each participant is directly aware of only a small subset of peers. When storing or retrieving a key, the participant requests additional peers from its neighbors in a semi-greedy search, minimizing the XOR distance until it finds $k$ nearest peers. In Kademlia, nodes form a special navigable graph structure that lets them find nearest peers in at most $\cO(k + \log N)$ requests to other peers, where $N$ is the total number of participants. Due to their scalability and fault-tolerance, DHTs found numerous applications including BitTorrent, Ethereum, I2P and decentralized deep learning~\cite{learning_at_home}.

\section{Proofs of Mixing Properties of Moshpit All-Reduce}\label{sect:missing_proofs}

\textbf{Notation.} Throughout the following sections, we use the standard notation from the literature on stochastic optimization. That is, for any $n$-dimensional vectors $x = (x_1,\ldots,x_n)^\top,y = (y_1,\ldots,y_n)^\top\in\R^n$ we use $\langle x,y\rangle$ to denote the standard inner product: $\langle x, y\rangle = x_1y_1 + \ldots + x_ny_n$. Next, we use $\|x\|$ to denote the $\ell_2$=norm of $x$ ($\|x\| = \sqrt{\langle x, x\rangle}$), $\EE[\xi]$ to denote an expectation of a random variable $\xi$, $\EE[\xi\mid \eta]$ is used for the conditional expectation of $\xi$ given $\eta$, and $\PP\{E\}$ denotes the probability of an event $E$.

\subsection{Computing exact average in a full grid}\label{sect:equiv_to_torus}
As discussed in Section~\ref{sect:method_algorithm}, Moshpit All-Reduce obtains the exact average of parameter vectors from $N$ peers arranged in a grid with $d$ coordinates and $M$ positions per coordinate when $N\equiv M^d$. That is, when the grid is full and each step averages $M$ parameter values along a single grid coordinate without repetitions, the algorithm needs only $d$ steps to compute the actual average across all nodes. In this section, we give a proof of this fact.

First, let us formally define the setting and the averaging steps of Moshpit All-Reduce in this specific case. Let $\theta_{i_1 i_2\ldots i_d}$ be the parameter vector of the worker with coordinates $i_1, i_2,\ldots, i_d$; each coordinate $i_k$ takes values from $1$ to $M$, because the hypercube of peers is completely full (thus, due to the pigeonhole principle, there are no unoccupied coordinates). Next, arrange the coordinates of these vector according to the order of averaging iterations: namely, at iteration 1
\begin{equation}
    \overline{\theta}_{i_1 i_2\ldots  i_d}^1=\frac{1}{M}\sum_{j_1=1}^M \theta_{j_1 i_2\ldots i_d},\quad  i_1\in\{1,\ldots,M\},
\end{equation}
which means that for the first iteration, we take the average across the first axis $\overline{\theta}^1$ and replicate it across all $M$ resulting vectors regardless of their index $i_1$. The next averaging steps can be expressed similarly with a simple recurrence relation:
\begin{equation}
\label{eqn:avg_recurrence}
    \overline{\theta}_{i_1 i_2 \ldots i_d}^t=\frac{1}{M}\sum_{j_t=1}^M \overline{\theta}_{i_1\ldots i_{t-1} j_t i_{t+1}\ldots i_d}^{t-1}.
\end{equation}
Given this formal definition, we can now state and prove the exact averaging result:
\begin{theorem}[Exact average in a full $d$-dimensional hypercube after $d$ steps]
Assume that $M^d$ peers are arranged in a $d$-dimensional hypercube with $M$ positions in each dimension. Also, assume that each peer fully participates in every averaging step and $M$-sized groups for each averaging iteration are determined based on the hypercube coordinates. Then, if Moshpit All-Reduce is ran in the above setup for $d$ iterations without repeating groups (i.e. averaging across each dimension exactly once), its result for each participant is the average value of $\theta$ across all $M^d$ peers.
\end{theorem}
\begin{proof}
We can directly obtain the expression for the average by expanding the recurrence and rearranging the sums:
\begin{eqnarray*}
    \overline{\theta}_{i_1 i_2\ldots i_d}^d &=& \frac{1}{M}\sum_{j_d=1}^M\overline{\theta}_{i_1\ldots i_{d-1} j_d}^{d-1}=\frac{1}{M}\sum_{j_d=1}^M\left(\frac{1}{M}\sum_{j_{d-1}=1}^M \overline{\theta}_{i_1 i_2\ldots j_{d-1}j_d}\right)=\ldots\\
    &=& \frac{1}{M}\Bigg(\underbrace{\sum_{j_d=1}^M\Bigg(\frac{1}{M}\sum_{j_{d-1}=1}^M\ldots\sum_{j_2=1}^M\Bigg(\frac{1}{M}\sum_{j_1=1}^M}_{d\textrm{ summations}} \theta_{j_1 \ldots j_d}\Bigg)\Bigg)\Bigg)\\
    &=& \frac{1}{M^d}\sum_{j_d=1}^M\sum_{j_{d-1}=1}^M\ldots\sum_{j_2=1}^M\sum_{j_1=1}^M \theta_{j_1 \ldots j_d} =\frac{1}{M^d}\sum_{j_1, \ldots, j_d=1}^M  \theta_{j_1 \ldots j_d}.
\end{eqnarray*}
But this is exactly the global average of all $\theta$, since there are $M^d$ participants and each vector is represented in the sum because of summation over all possible indices.
\end{proof}

Notice that for a given grid of peers, if some of its indices do not have corresponding parameter vectors, Equation~\ref{eqn:avg_recurrence} may result in different average vectors on different workers due to different numbers of peers along a coordinate for different indices. For example, running two iterations of Moshpit Averaging with $d=2,\ M=2$ and three parameter vectors $\theta_{11},\ \theta_{21},\ \theta_{22}$ results in $\frac{\theta_{11}+\theta_{21}}{2}$ on the first worker and $\frac{\theta_{11}+\theta_{21}}{4}+\theta_{22}$ on other workers, with neither equal to the global average. However, the variance of the averaged vectors does decrease, which is formally proven in Section~\ref{sec:proof_quality_of_avg_deterministic_vectors}.

\subsection{Proof of Theorem~\ref{thm:quality_of_avg_deterministic_vectors_0}}\label{sect:correctness_proof}
Below we provide the complete proof of Theorem~\ref{thm:quality_of_avg_deterministic_vectors_0}. For the readers' convenience, we restate the theorem.
\begin{theorem}[Theorem~\ref{thm:quality_of_avg_deterministic_vectors_0}]\label{thm:quality_of_avg_deterministic_vectors_0_supp}
If all workers have non-zero probability of successfully running a communication round in Moshpit Averaging and the order of $\texttt{peers}_t$ is random, then all local vectors $\theta^t_i$ converge to the global average with probability $1$:
\begin{equation}
    \forall i = 1,\ldots, N\quad \left\|\theta^t_i - \frac1N \sum_{i=1}^N \theta^0_i\right\|^2 \xrightarrow[t\to\infty]{} 0.\label{eq:quality_of_avg_deterministic_vectors_0_supp}
\end{equation}
\end{theorem}
\begin{proof}[Proof of Theorem~\ref{thm:quality_of_avg_deterministic_vectors_0}]
    First of all, we notice that \eqref{eq:quality_of_avg_deterministic_vectors_0_supp} is equivalent to
    \begin{equation}
    \forall i = 1,\ldots, N,\;\forall j=1,\ldots,n\quad \left(\theta^t_i(j) - \frac1N \sum_{i=1}^N \theta^0_i(j)\right)^2 \xrightarrow[t\to\infty]{} 0,\label{eq:quality_of_avg_deterministic_vectors_0_supp_tech_1}
    \end{equation}
    where $\theta_i^t(j)$ denotes $j$-th component of $\theta_i^t$. Consider an arbitrary component $j \in \{1,\ldots,n\}$ and the sequence of intervals $\{I_{j,t}\}_{t\ge 0}$ where $I_{j,t} = \text{conv}\{\theta_1^t(j),\theta_2^t(j),\ldots, \theta_N^t(j)\}$. Then, $\{I_{j,t}\}_{t\ge 0}$ is a sequence of nested intervals ($I_{j,t+1} \subseteq I_{j,t} \forall t\ge 0$), since averaging in groups does not expand the convex hull of $\{\theta_1^t,\theta_2^t,\ldots, \theta_N^t\}$. For convenience, we specify the bounds of the intervals: $I_{j,t} = [a_{j,t}, b_{j,t}]$. Using the Cantor's intersection theorem, we conclude that
    \begin{equation*}
        \bigcap\limits_{t=0}^\infty I_{j,t} = I_j = [a_j, b_j],
    \end{equation*}
    where $\overline{\theta}(j) = \frac{1}{N}\sum_{i=1}^n\theta_i^0(j) \in [a_j, b_j]$. If $[a_j, b_j] = \{\overline{\theta}(j)\}$ with probability $1$, then \eqref{eq:quality_of_avg_deterministic_vectors_0_supp_tech_1} holds with probability $1$ as well. Suppose the opposite: there exist such $j \in \{1,\ldots,n\}$, $[a,b]$ and $\delta,\Delta > 0$ that $\overline{\theta}(j) \in [a,b]$, $b-a = \Delta$ and
    \begin{equation*}
        \PP\Bigg\{\underbrace{[a,b] \subseteq \bigcap\limits_{t=0}^\infty I_{j,t}}_{E}\Bigg\} = \delta > 0\quad \text{ and }\quad \forall \varepsilon > 0\; \PP\Bigg\{\underbrace{[a-\varepsilon,b+\varepsilon] \subseteq \bigcap\limits_{t=0}^\infty I_{j,t}}_{E_{\varepsilon}}\Bigg\} < \delta.
    \end{equation*}
    This implies that for all $\varepsilon > 0$ there exists such $T_{\varepsilon} > 0$ that
    \begin{equation*}
        \PP\Big\{\underbrace{\forall t \ge T_{\varepsilon}\;\; a_{j,t}\in [a-\varepsilon,a], b_{j,t}\in[b,b+\varepsilon]}_{E_{\varepsilon}'}\Big\} = \delta_{\varepsilon} > 0.
    \end{equation*}
    Consider $\varepsilon = \frac{\Delta}{(2N+100)^{2N}}$ and assume that the event $E_{\varepsilon}'$ holds. Next, we introduce new notation: $J_{\text{left}}^t = \{i \in \{1,\ldots, n\}\mid \theta_{i}^t(j) \in [a-\varepsilon,a]\}$ and $J_{\text{right}}^t = \{i \in \{1,\ldots, n\}\mid \theta_{i}^t(j) \in [b,b+\varepsilon]\}$. Since $E_{\varepsilon}'$ holds the sets $J_{\text{left}}^t$ and $J_{\text{right}}^t$ are non-empty for all $t\ge T_{\varepsilon}$ with probability $\delta_{\varepsilon} > 0$:
    \begin{equation}
        \PP\left\{\forall t \ge T_{\varepsilon}\;\; J_{\text{left}}^t \neq \varnothing\text{ and }  J_{\text{right}}^t \neq \varnothing\right\} = \delta_{\varepsilon} > 0. \label{eq:quality_of_avg_deterministic_vectors_0_supp_tech_2}
    \end{equation}
    We notice that every pair of workers $i_1,i_2$ has a non-zero probability of taking part in the averaging inside the common group at each iteration since all workers have a non-zero probability of successfully running a communication round and the order of $\texttt{peers}_t$ is random. This implies that every pair of workers $i_1,i_2$ with probability $1$ take part in the averaging inside the common group infinitely many times when $t$ goes to the infinity.
    
    Next, we choose some $t_0 \ge T_{\varepsilon}$. Let $J_{\text{left}}^{t_0} = \{i_{l,1},\ldots, i_{l,q_l}\}$ and $J_{\text{right}}^{t_0} = \{i_{r,1},\ldots, i_{r,q_r}\}$. Consider the event $E_{\varepsilon,0}' \subseteq E_{\varepsilon}'$ such that in $E_{\varepsilon,0}'$ peer $i_{l,1}$ computes an average in the group containing any peer from $J_{\text{right}}^{t_0}$ at some iteration $t_1 > t_0$. Our observations above imply that $\PP\{E_{\varepsilon,0}'\} = \PP\{E_{\varepsilon}'\} = \delta_{\varepsilon} > 0$. Then, $\theta_{i_{l,1}}^{t_1}(j) \ge \frac{N-1}{N}(a-\varepsilon) + \frac{1}{N}b = a-\varepsilon + \frac{1}{N}(\Delta + \varepsilon) = a - \frac{\Delta}{(2N+100)^{2N}} + \frac{1}{N}\left(\Delta + \frac{\Delta}{(2N+100)^{2N}}\right) > a + \frac{\Delta}{2N}$, i.e., $\theta_{i_{l,1}}^{t_1}(j) \in (a,b]$ meaning that $i_{l,1} \not\in J_{\text{left}}^{t_1}$. The last part of the proof shows that for any $t\ge t_1$, the peer $i_{l,1}$ will never be the part of $J_{\text{left}}^t$ and after a finite number of iterations $J_{\text{left}}^t = \varnothing$ with probability $\delta_{\varepsilon} > 0$ when $E_{\varepsilon,0}'$ holds, implying the contradiction with \eqref{eq:quality_of_avg_deterministic_vectors_0_supp_tech_2}.
    
    To show that, we consider the following set of peers: $\widehat{J}_{\text{left}}^{t_1} = \{i\in\{1,\ldots,n\}\mid \exists t \ge t_1:\; \theta_i^{t}(j)\in [a-\varepsilon, a+\frac{\Delta}{2N})\}$. Next, we consider the event $E_{\varepsilon,1}'\subseteq E_{\varepsilon,0}'$ such that in $E_{\varepsilon,1}'$ peer $i_{l,1}$ computes an average in the group containing some peer $i_{l,avg,1}$ from $\widehat{J}_{\text{left}}^{t_1}$ at some iteration $t_2 > t_1$ (and $t_2$ is the first such moment after $t_1$). Again, our observations imply $\PP\{E_{\varepsilon,1}'\} = \PP\{E_{\varepsilon,0}'\} = \delta_{\varepsilon}>0$. Then, $\theta_{i_{l,1}}^{t_2}(j) = \theta_{i_{l,avg,1}}^{t_2}(j) > \frac{N-1}{N}(a-\varepsilon) + \frac{1}{N}\left(a+\frac{\Delta}{2N}\right) = a + \frac{\Delta}{2N^2} - \frac{(N-1)\Delta}{N(2N+100)^{2N}} > a + \frac{\Delta}{4N^2}$. After that, we consider the event $E_{\varepsilon,2}'\subseteq E_{\varepsilon,1}'$ such that in $E_{\varepsilon,2}'$ peer $i_{l,1}$ or $i_{l,avg,1}$ computes an average in the group containing a peer $i_{l,avg,2}\neq i_{l,avg,1}$ from $\widehat{J}_{\text{left}}^{t_1}$ at an iteration $t_3 > t_2$ (and $t_3$ is the first such moment after $t_2$). Then, $\theta_{i_{l,1}}^{t_3}(j), \theta_{i_{l,avg,1}}^{t_3}(j)$ and $\theta_{i_{l,avg,2}}^{t_3}(j)$ are greater than $\frac{N-1}{N}(a-\varepsilon) + \frac{1}{N}\left(a + \frac{\Delta}{4N^2}\right) = a + \frac{\Delta}{4N^3} - \frac{(N-1)\Delta}{N(2N+100)^{2N}} > a + \frac{\Delta}{8N^3}$.
    
    Therefore, after at least $N-1$ of such averaging iterations, with probability $\delta_\varepsilon$ all $\theta_i^t(j)$ will be greater than $a + \frac{\Delta}{(2N)^N} > a$ while $E_{\varepsilon}'$ holds. This contradicts \eqref{eq:quality_of_avg_deterministic_vectors_0_supp_tech_2}. Therefore, 
    \begin{equation*}
        \bigcap\limits_{t=0}^\infty I_{j,t} = \{\overline{\theta}(j)\}
    \end{equation*}
    with probability $1$, which concludes the proof.
\end{proof}


\subsection{Proof of Theorem~\ref{thm:quality_of_avg_deterministic_vectors}}\label{sec:proof_quality_of_avg_deterministic_vectors}
In this section, we provide the complete proof of Theorem~\ref{thm:quality_of_avg_deterministic_vectors}. For convenience, we restate the theorem below.
\begin{theorem}[Theorem~\ref{thm:quality_of_avg_deterministic_vectors}, averaging convergence rate]\label{thm:quality_of_avg_deterministic_vectors_supp}
    Consider the modification of Moshpit All-Reduce that works as follows: at each iteration $k\geq 1$ 1) peers are randomly split into $r$ disjoint groups of sizes $M_1^k,\ldots, M_r^k$ in such a way that $\sum_{i=1}^r M_i^k = N$ and $M_i^k \ge 1\  \forall i = 1,\ldots,r$ and 2) peers from each group compute their group average via All-Reduce. Let $\theta_1,\ldots,\theta_N$ be the input vectors of this procedure and $\theta_1^T,\ldots,\theta_N^T$ be the outputs after $T$ iterations. Then,
    \begin{eqnarray}
         \EE\left[\frac{1}{N}\sum\limits_{i=1}^N\|\theta_i^T - \overline{\theta}\|^2\right] = \left(\frac{r-1}{N} + \frac{r}{N^2}\right)^T\cdot\frac{1}{N}\sum\limits_{i=1}^N\|\theta_i - \overline{\theta}\|^2, \label{eq:determ_quality_of_avg_supp}
    \end{eqnarray}
    where $\overline{\theta} = \frac{1}{N}\sum_{i=1}^N\theta_i$.
\end{theorem}
\begin{proof}
First of all, let us clarify the procedure of random splitting of peers in $r$ groups. We assume that at iteration $k$ of the modified algorithm we generate a random permutation $\pi^k = (\pi_1^k,\ldots,\pi_N^k)$ of $1,\ldots, N$. Next, $J_1^k = \{\pi_1^k,\ldots,\pi_{M_1^k}^k\}$ form the indices of the first group of workers, $J_2^k = \{\pi_{M_1^k+1}^k,\ldots,\pi_{M_2^k}^k\}$ are the indices of the second group, and $J_r^k = \{\pi_{M_1^k+M_2^k+\ldots+M_{r-1}^k+1}^k,\ldots,\pi_{N}^k\}$ are the indices of group $r$. In other words, we generate a random permutation and take contiguous subgroups of indices corresponding to predefined group sizes $M_i^k$, starting from the first group.

By definition, we have $\bigsqcup_{i=1}^r J_i^k = \{1,2,\ldots,N\}$, where $\sqcup$ defines the disjoint union operator. Moreover, notice that group sizes $M_1^k,\ldots,M_r^k$ can depend on $k$ and even be random: for our analysis, it is sufficient that the randomness defining the permutation is independent from $M_1^k,\ldots,M_r^k$. Next, vectors $\theta_1^k,\ldots,\theta_N^k$ are obtained by the following formula:
\begin{equation*}
    \forall j=1,\ldots,N,\quad \theta_j^k = \frac{1}{M_i^k}\sum\limits_{t\in J_i^k}\theta_t^{k-1},\quad \text{where } J_i^k \text{ is the group for which } j\in J_i^k.
\end{equation*}
Using this, we show that the average of vectors $\{\theta_i^k\}_{i=1}^n$ remains the same throughout the iterations of Moshpit All-Reduce:
\begin{equation*}
    \frac{1}{N}\sum\limits_{j=1}^N\theta_j^k = \frac{1}{N}\sum\limits_{i=1}^rM_i^k\cdot\frac{1}{M_i^k}\sum\limits_{t\in J_i^k}\theta_t^{k-1} = \frac{1}{N}\sum\limits_{i=1}^r\sum\limits_{t\in J_i^k}\theta_t^{k-1} = \frac{1}{N}\sum\limits_{j=1}^N\theta_j^{k-1}.
\end{equation*}
Therefore, the quantity $\frac{1}{N}\sum_{j=1}^N\|\theta_j^k - \overline{\theta}\|^2$ (average distortion) measures the quality of averaging. For this quantity, we can derive the following expression:
\begin{eqnarray}
    \frac{1}{N}\sum\limits_{j=1}^N\|\theta_j^k - \overline{\theta}\|^2 &=& \frac{1}{N}\sum\limits_{i=1}^r M_i^k\left\|\frac{1}{M_i^k}\sum\limits_{t\in J_i^k}\theta_t^{k-1} - \overline{\theta}\right\|^2\notag\\
    &=& \frac{1}{N}\sum\limits_{i=1}^r\frac{1}{M_i^k}\left(\sum\limits_{t\in J_i^k}\|\theta_t^{k-1} - \overline{\theta}\|^2 + 2\sum\limits_{t,l\in J_i^k, t < l}\langle \theta_t^{k-1} - \overline{\theta}, \theta_l^{k-1} - \overline{\theta} \rangle\right).\notag
\end{eqnarray}
Taking the expectation $\EE_{\pi^k}[\cdot]$ with respect to the randomness coming from the choice of $\pi^k$ we get
\begin{eqnarray}
    \EE_{\pi^k}\left[\frac{1}{N}\sum\limits_{j=1}^N\|\theta_j^k - \overline{\theta}\|^2\right] &\notag\\
    &\hspace{-2.5cm}= \frac{1}{N}\sum\limits_{i=1}^r\frac{1}{M_i^k}\left(\EE_{\pi^k}\left[\sum\limits_{t\in J_i^k}\|\theta_t^{k-1} - \overline{\theta}\|^2\!\right] \!+\! 2\EE_{\pi^k}\!\left[\sum\limits_{t,l\in J_i^k, t < l}\langle \theta_t^{k-1} - \overline{\theta}, \theta_l^{k-1} - \overline{\theta} \rangle\right]\right).\notag
\end{eqnarray}
Since $\forall j,j_1,j_2 \in\{1,\ldots,N\},j_1\neq j_2$ and for all $i=1,\ldots,r$
\begin{equation*}
    \PP\left\{j\in J_i^k\right\} = \frac{M_i^k}{N},\quad \PP\left\{j_1,j_2 \in J_i^k\right\} = \frac{M_{i}^k(M_i^k - 1)}{N^2},
\end{equation*}
we have
\begin{eqnarray*}
    \EE_{\pi^k}\left[\frac{1}{N}\sum\limits_{j=1}^N\|\theta_j^k - \overline{\theta}\|^2\right] &=& \frac{1}{N}\sum\limits_{i=1}^r\frac{1}{N}\sum\limits_{j=1}^N\|\theta_j^{k-1} - \overline{\theta}\|^2\\
    &&\quad +\frac{1}{N}\sum\limits_{i=1}^r2\frac{M_i^k - 1}{N^2}\sum\limits_{1 \le j_1 < j_2 \le N}\langle \theta_{j_1}^{k-1} - \overline{\theta}, \theta_{j_2}^{k-1} - \overline{\theta}\rangle\\
    &=& \frac{r}{N^2}\sum\limits_{j=1}^N\|\theta_j^{k-1} - \overline{\theta}\|^2 + 2\frac{N-r}{N^3}\sum\limits_{1 \le j_1 < j_2 \le N}\langle \theta_{j_1}^{k-1} - \overline{\theta}, \theta_{j_2}^{k-1} - \overline{\theta}\rangle\\
    &=& \left(\frac{r}{N^2} - \frac{N-r}{N^3}\right)\sum\limits_{j=1}^N\|\theta_j^{k-1} - \overline{\theta}\|^2 +\frac{N-r}{N^3}\sum\limits_{j=1}^N\|\theta_j^{k-1} - \overline{\theta}\|^2\\
    &&\quad +2\frac{N-r}{N^3}\sum\limits_{1 \le j_1 < j_2 \le N}\langle \theta_{j_1}^{k-1} - \overline{\theta}, \theta_{j_2}^{k-1} - \overline{\theta}\rangle\\
    &=& \frac{N(r-1)+r}{N^3}\sum\limits_{j=1}^N\|\theta_j^{k-1} - \overline{\theta}\|^2 + \frac{N-r}{N^3}\underbrace{\left\|\sum\limits_{j=1}^N(\theta_j^{k-1} - \overline{\theta})\right\|^2}_{\|N\overline{\theta} - N\overline{\theta}\|^2 = 0}\\
    &=& \left(\frac{r-1}{N} + \frac{r}{N^2}\right)\cdot\frac{1}{N}\sum\limits_{j=1}^N\|\theta_j^{k-1} - \overline{\theta}\|^2.
\end{eqnarray*}
Finally, we take the full expectation from the both sides of the above equation and apply the tower property $\EE\left[\EE_{\pi^k}\left[\cdot\right]\right] = \EE\left[\cdot\right]$:
\begin{equation*}
    \EE\left[\frac{1}{N}\sum\limits_{j=1}^N\|\theta_j^k - \overline{\theta}\|^2\right] = \left(\frac{r-1}{N} + \frac{r}{N^2}\right)\EE\left[\frac{1}{N}\sum\limits_{j=1}^N\|\theta_j^{k-1} - \overline{\theta}\|^2\right].
\end{equation*}
Unrolling the recurrence for $k=T$, we establish \eqref{eq:determ_quality_of_avg_supp}.
\end{proof}

\begin{remark}
    The result implies that increasing the group size $\alpha > 1$ times implies almost $\alpha$ times faster convergence to the average.
\end{remark}

\begin{remark}
    Our analysis can be easily generalized to the case when number of groups $r$ can depend on $k$ and be a random variable independent from the choice of permutations and the number of groups at previous steps. In this case, \eqref{eq:determ_quality_of_avg_supp} transforms into
    \begin{equation}
        \EE\left[\frac{1}{N}\sum\limits_{i=1}^N\|\theta_i^T - \overline{\theta}\|^2\right] = \frac{1}{N}\sum\limits_{i=1}^N\|\theta_i - \overline{\theta}\|^2\cdot\prod_{k=1}^T\left(\frac{\EE[r_k]-1}{N} + \frac{\EE[r_k]}{N^2}\right), \label{eq:determ_quality_of_avg_generalized_supp}
    \end{equation}
    where $r_k$ is the number of groups at iteration $k$.
\end{remark}

\subsection{Additional Guarantees For Moshpit Averaging}\label{sec:mix_rand_proof}
In this section,  we derive the result measuring the rate of variance reduction when averaging random vectors with Algorithm~\ref{alg:moshpit}. We start with the following technical lemma:
\begin{lemma}\label{lem:ode_lemma}
    Let $\xi \sim \text{Binom}(M,p)$ have a binomial distribution with parameters $M$ (number of trials) and $p$ (probability of success for each trial). Then
    \begin{eqnarray}
        m_1(M,p) := \EE\left[\min\left\{\frac{1}{\xi},1\right\}\right] &=& (1-p)^M + \sum\limits_{i=1}^M\frac{(1-p)^{M-i} - (1-p)^M}{i}, \label{eq:binom_first_inverse_moment}\\
        m_2(M,p) := \EE\left[\min\left\{\frac{1}{\xi^2},1\right\}\right] &=& (1-p)^M + \sum\limits_{i=1}^M\frac{(1-p)^{M-i} - (1-p)^M}{i}\sum\limits_{j=i}^M\frac{1}{j}. \label{eq:binom_second_inverse_moment}
    \end{eqnarray}
\end{lemma}
\begin{proof}
    We start with the proof of \eqref{eq:binom_first_inverse_moment}. By definition of the expectation, we have
    \begin{eqnarray*}
        \EE\left[\min\left\{\frac{1}{\xi},1\right\}\right] &=& (1-p)^M + \sum\limits_{i=1}^M \frac{1}{i}p^i(1-p)^{M-i}\binom{M}{i}.
    \end{eqnarray*}
    For simplicity of further derivations, we introduce the following notation: $m_1(M,p) = \EE\left[\min\left\{\frac{1}{\xi},1\right\}\right]$ and $m_2(M,p) = \EE\left[\min\left\{\frac{1}{\xi^2},1\right\}\right]$. Taking the derivative of $m_1(M,p)$ by $p$, we obtain
    \begin{eqnarray*}
        m_1'(M,p) &=& -M(1-p)^{M-1} + \sum\limits_{i=1}^Mp^{i-1}(1-p)^{M-i}\binom{M}{i} \\
        &&\quad - \sum\limits_{i=1}^M\frac{M-i}{i}p^i(1-p)^{M-i-1}\binom{M}{i}\\
        &=& -M(1-p)^{M-1} + \frac{1}{p}\left(-(1-p)^M + \sum\limits_{i=0}^Mp^{i}(1-p)^{M-i}\binom{M}{i}\right)\\
        && - \frac{M}{1-p}\sum\limits_{i=1}^M\frac{1}{i}p^i(1-p)^{M-i}\binom{M}{i}\\
        &&\quad + \frac{1}{1-p}\left(-(1-p)^M + \sum\limits_{i=0}^Mp^i(1-p)^{M-i}\binom{M}{i}\right)\\
        &=& -M(1-p)^{M-1} + \frac{1}{p}\left(1 - (1-p)^M\right) - \frac{M}{1-p}\left(m_1(M,p) - (1-p)^M\right)\\
        &&\quad+ \frac{1}{1-p}\left(1- (1-p)^M\right)\\
        &=& \frac{1}{p(1-p)} - \frac{(1-p)^{M-1}}{p} - \frac{M}{1-p}m_1(M,p).
    \end{eqnarray*}
    Rearranging the terms, we get the following linear first-order ODE
    \begin{equation}
        m_1'(M,p) + \frac{M}{1-p}m_1(M,p) = \frac{1}{p(1-p)} - \frac{(1-p)^{M-1}}{p}. \label{eq:first_moment_ODE}
    \end{equation}
    To solve it, we consider the following homogeneous ODE:
    \begin{equation*}
        m_1'(M,p) + \frac{M}{1-p}m_1(M,p) = 0.
    \end{equation*}
    The solution of this ODE is $m_1(M,p) = C(1-p)^M$, where $C\in\R$ is an arbitrary real constant. Next, we go back to the initial ODE \eqref{eq:first_moment_ODE} and try to find a solution of the form $m_1(M,p) = C(p)(1-p)^M$, where $C(p):\R \to \R$ is a differentiable function:
    \begin{eqnarray*}
        \left(C(p)(1-p)^M\right)' + \frac{M}{1-p}C(p)(1-p)^M &=& \frac{1}{p(1-p)} - \frac{(1-p)^{M-1}}{p}\\
        &\Downarrow&\\
        C'(p)(1-p)^M &=& \frac{1}{p(1-p)} - \frac{(1-p)^{M-1}}{p}\\
        &\Downarrow&\\
        C'(p) &=& \frac{1}{p(1-p)^{M+1}} - \frac{1}{p(1-p)}.
    \end{eqnarray*}
    Since 
    \begin{equation}
        \frac{1}{x(1-x)^{k+1}} = \frac{1}{x(1-x)^{k}} + \frac{1}{(1-x)^{k+1}}\label{eq:technical_expansion}
    \end{equation}
    for all $x\not\in \{0,1\}$ and all non-negative integers $k$, we have
    \begin{eqnarray*}
        C'(p) &=& \frac{1}{p} + \frac{1}{1-p} + \frac{1}{(1-p)^2} + \ldots + \frac{1}{(1-p)^{M+1}} - \frac{1}{p} - \frac{1}{1-p}\\
        &\Downarrow&\\
        C'(p) &=& \sum\limits_{i=1}^M(1-p)^{-i-1},
    \end{eqnarray*}
    hence
    \begin{eqnarray*}
        C(p) = \hat{C} + \sum\limits_{i=1}^M\frac{1}{i}(1-p)^{-i},
    \end{eqnarray*}
    where $\hat{C}$ is a real constant. Putting all together, we obtain
    \begin{eqnarray*}
        m_1(M,p) &=& C(p)(1-p)^M = \hat{C}(1-p)^M + \sum\limits_{i=1}^M\frac{1}{i}(1-p)^{M-i}.
    \end{eqnarray*}
    Taking $m_1(M,0) = 1$ into account, we conclude that $\hat{C} = 1 - \sum_{i=1}^M\frac{1}{i}$ and obtain \eqref{eq:binom_first_inverse_moment}.
    
    Using a similar technique, we derive \eqref{eq:binom_second_inverse_moment}. By definition of the expectation, we have
    \begin{eqnarray*}
        m_2(M,p) &=& (1-p)^M + \sum\limits_{i=1}^M \frac{1}{i^2}p^i(1-p)^{M-i}\binom{M}{i}.
    \end{eqnarray*}
    Taking the derivative of $m_2(M,p)$ by $p$, we obtain
    \begin{eqnarray*}
        m_2'(M,p) &=& -M(1-p)^{M-1} + \sum\limits_{i=1}^M\frac{1}{i}p^{i-1}(1-p)^{M-i}\binom{M}{i}\\
        &&\quad - \sum\limits_{i=1}^M\frac{M-i}{i^2}p^i(1-p)^{M-i-1}\binom{M}{i}\\
        &=& -M(1-p)^{M-1} + \frac{1}{p} \sum\limits_{i=1}^M\frac{1}{i}p^{i}(1-p)^{M-i}\binom{M}{i}\\
        && - \frac{M}{1-p}\sum\limits_{i=1}^M\frac{1}{i^2}p^i(1-p)^{M-i}\binom{M}{i} + \frac{1}{1-p}\sum\limits_{i=1}^M\frac{1}{i}p^i(1-p)^{M-i}\binom{M}{i}\\
        &=& -M(1-p)^{M-1} + \frac{1}{p}\left(m_1(M,p) - (1-p)^M\right) \\
        &&\quad + \frac{1}{1-p}\left(-M m_2(M,p) + M(1-p)^M + m_1(M,p) - (1-p)^M\right)\\
        &=& \frac{m_1(M,p)}{p(1-p)} - \frac{(1-p)^{M-1}}{p} - \frac{M}{1-p}m_2(M,p).
    \end{eqnarray*}
    Rearranging the terms, we get the following linear first-order ODE
    \begin{equation}
        m_2'(M,p) + \frac{M}{1-p}m_2(M,p) = \frac{m_1(M,p)}{p(1-p)} - \frac{(1-p)^{M-1}}{p}. \label{eq:second_moment_ODE}
    \end{equation}
    To solve this ODE, we consider the homogeneous ODE:
    \begin{equation*}
        m_2'(M,p) + \frac{M}{1-p}m_2(M,p) = 0.
    \end{equation*}
    The solution of this ODE is $m_2(M,p) = C(1-p)^M$, where $C\in\R$ is an arbitrary real constant. Next, we go back to the initial ODE \eqref{eq:second_moment_ODE} and try to find a solution of the form $m_2(M,p) = C(p)(1-p)^M$, where $C(p):\R \to \R$ is a differentiable function:
    \begin{eqnarray*}
        \left(C(p)(1-p)^M\right)' + \frac{M}{1-p}C(p)(1-p)^M &=& \frac{m_1(M,p)}{p(1-p)} - \frac{(1-p)^{M-1}}{p}\\
        &\Downarrow&\\
        C'(p)(1-p)^M &=& \frac{m_1(M,p)}{p(1-p)} - \frac{(1-p)^{M-1}}{p}\\
        &\Downarrow&\\
        C'(p) &=& \frac{m_1(M,p)}{p(1-p)^{M+1}} - \frac{1}{p(1-p)}.
    \end{eqnarray*}
    Using \eqref{eq:technical_expansion} and \eqref{eq:binom_first_inverse_moment}, we derive
    \begin{eqnarray*}
        C'(p) &\overset{\eqref{eq:binom_first_inverse_moment}}{=}& -\frac{\sum\limits_{i=1}^M\frac{1}{i}}{p(1-p)} + \frac{\sum\limits_{i=1}^M\frac{1}{i}(1-p)^{M-i}}{p(1-p)^{M+1}}\\
        &=& -\sum\limits_{i=1}^M \frac{1}{ip(1-p)} + \sum\limits_{i=1}^M\frac{1}{ip(1-p)^{i+1}}\\
        &\overset{\eqref{eq:technical_expansion}}{=}& -\sum\limits_{i=1}^M\frac{1}{i}\left(\frac{1}{p} + \frac{1}{1-p}\right)\\
        &&\quad + \sum\limits_{i=1}^M\frac{1}{i}\left(\frac{1}{p} + \frac{1}{1-p} + \frac{1}{(1-p)^2} + \ldots + \frac{1}{(1-p)^{i+1}}\right)\\
        &=& \sum\limits_{i=1}^M\frac{1}{i}\left(\frac{1}{(1-p)^2} + \ldots + \frac{1}{(1-p)^{i+1}}\right) = \sum\limits_{i=1}^M \frac{1}{(1-p)^{i+1}}\sum\limits_{j=i}^M\frac{1}{j},
    \end{eqnarray*}
    hence 
    \begin{eqnarray*}
        C(p) = \hat{C} + \sum\limits_{i=1}^M\frac{1}{i}(1-p)^{-i}\sum\limits_{j=i}^M\frac{1}{j},
    \end{eqnarray*}
    where $\hat{C}$ is a real constant. Putting all together, we obtain
    \begin{eqnarray*}
        m_2(M,p) &=& C(p)(1-p)^M = \hat{C}(1-p)^M + \sum\limits_{i=1}^M\frac{1}{i}(1-p)^{M-i}\sum\limits_{j=i}^M\frac{1}{j}.
    \end{eqnarray*}
    Taking $m_2(M,0) = 1$ into account, we conclude that $\hat{C} = 1 - \sum_{i=1}^M\frac{1}{i}\sum_{j=i}^M\frac{1}{j}$ and obtain \eqref{eq:binom_second_inverse_moment}.
\end{proof}

Using this lemma, we derive the following result:
\begin{theorem}\label{thm:quality_of_avg_supp}
    Assume that peers participating in Moshpit Averaging have independent random vectors $\theta_1,\ldots,\theta_N$ with means $\overline{\theta}_1,\ldots,\overline{\theta}_N$ and variances bounded by $\sigma^2$ before the averaging. Let $\theta_1^T,\ldots,\theta_N^T$ be the outputs of Moshpit Averaging after $T$ iterations. Finally, we assume that each peer from the grid can be dropped out for the whole averaging process before averaging independently from other peers, i.e., $N \sim \text{Binom}(M^d,p)$. Then, for all $i = 1,\ldots,N$ we have
    \begin{equation}
        \EE\left[\left\|\theta_i^T - \EE_{\theta}\left[\theta_i^T\right]\right\|^2\right] \leq M^{T-1}\sigma^2 m_1(M-1,p)\left(m_2(M-1,p)\right)^{T-1},\label{eq:variance_bound_supp}
    \end{equation}
    where functions $m_1(M,p)$ and $m_2(M,p)$ are defined in \eqref{eq:binom_first_inverse_moment} and \eqref{eq:binom_second_inverse_moment} respectively, and $\EE_\theta\left[\cdot\right]$ denotes the expectation w.r.t.\ the randomness from $\theta_1,\ldots,\theta_N$. Moreover, if $p \ge \frac{2}{3}$ and $M \ge 11$, then $m_1(M-1,p) \le \frac{2}{M}$, $m_2(M-1,p) \le \frac{3}{M^2}$ and 
    \begin{equation}
        \EE\left[\left\|\theta_i^T - \EE_{\theta}\left[\theta_i^T\right]\right\|^2\right] \leq \frac{2\sigma^2}{M(\nicefrac{M}{3})^{T-1}}.\label{eq:variance_bound_2_supp}
    \end{equation}
\end{theorem}
\begin{proof}
First of all, we recall an equivalent formulation of Moshpit Averaging. Consider a hypercube $\{1,\ldots,M\}^d$. One can consider the elements of this hypercube as hyperindices and assign a unique hyperindex to each peer so that peers can be viewed as vertices in the hypercube. Then, during the $k$-th iteration of Moshpit All-Reduce, each worker computes the average among those peers that have hyperindices with the same values except the $k$-th index; in other words, peers compute averages along the $k$-th dimension of the hypercube. Next, if $N = 0$, we assume that $\theta_i^T = \EE_{\theta}\left[\theta_i^T\right]$ and \eqref{eq:variance_bound_supp} holds for free. Therefore, to derive \eqref{eq:variance_bound_supp}, we assume that $N > 0$.

More formally, we use the following notation: $\theta_{C_i} = \theta_i$ for all $i= 1,\ldots,N$, where $C_{i} = (c_{1}^i, c_2^i,\ldots, c_d^i)$, $c_{j}^i \in \{1,\ldots,M\}$ for all $j = 1,\ldots,M$, and $C_{i} \neq C_k$ for $i\neq k$. Let $\cC$ be the set of hyperindices corresponding to all peers. Next, we use $\theta_{C_i}^t$ to define the vector stored on $i$-th peer after $t$ iterations of Moshpit Averaging. Then, for all $i = 1,\ldots,N$ we have $\theta_{C_i}^0 = \theta_{C_i}$ and for all $t = 1,\ldots,d$
\begin{equation*}
    \theta_{C_i}^{t} = \frac{1}{b_{i,t}}\sum\limits_{k\in J_{i,t}}\theta_{C_k}^{t-1},
\end{equation*}
where $J_{i,t} = \{k \in N\mid C_k = (c_1^k,\ldots,c_d^k) \in \cC \text{ and } c_j^k = c_j^i\; \forall j \neq t\}$ and $b_{i,t} = |J_{i,t}|$. Using this, we derive the following formula for $\theta_{C_i}^t$:
\begin{equation*}
    \theta_i^T \equiv \theta_{C_i}^T = \frac{1}{b_{i,T}}\sum\limits_{i_1\in J_{i,T}}\frac{1}{b_{i_1,T-1}}\sum\limits_{i_2\in J_{i_1,T-1}}\frac{1}{b_{i_2,T-2}}\sum\limits_{i_3\in J_{i_2,T-1}}\ldots\frac{1}{b_{i_{T-1},1}}\sum\limits_{i_T\in J_{i_{T-1},1}}\theta_{i_{T}}.
\end{equation*}
Taking the expectation w.r.t. $\theta_1,\ldots,\theta_N$, we get
\begin{equation*}
    \EE_{\theta}\left[\theta_i^T\right] = \frac{1}{b_{i,T}}\sum\limits_{i_1\in J_{i,T}}\frac{1}{b_{i_1,T-1}}\sum\limits_{i_2\in J_{i_1,T-1}}\frac{1}{b_{i_2,T-2}}\sum\limits_{i_3\in J_{i_2,T-1}}\ldots\frac{1}{b_{i_{T-1},1}}\sum\limits_{i_T\in J_{i_{T-1},1}}\overline{\theta}_{i_{T}}.
\end{equation*}
Using the independence of $\theta_1,\ldots,\theta_N$, we derive
\begin{eqnarray*}
    \EE_\theta\left[\left\|\theta_i^T - \EE_{\theta}\left[\theta_i^T\right]\right\|^2\right] &=& \EE_\theta\left[\left\|\sum\limits_{i_1\in J_{i,T}}\sum\limits_{i_2\in J_{i_1,T-1}}\ldots \sum\limits_{i_{T}\in J_{i_{T-1},1}}\frac{\theta_{i_T} - \overline{\theta}_{i_T}}{b_{i,T} b_{i_1,T-1}\ldots b_{i_{T-1},1}}\right\|^2\right]\\
    &=& \sum\limits_{i_1\in J_{i,T}}\sum\limits_{i_2\in J_{i_1,T-1}}\ldots \sum\limits_{i_{T}\in J_{i_{T-1},1}}\frac{\EE_\theta\left[\|\theta_{i_T} - \overline{\theta}_{i_T}\|^2\right]}{b_{i,T}^2 b_{i_1,T-1}^2\ldots b_{i_{T-1},1}^2}\\
    &\le& \sum\limits_{i_1\in J_{i,T}}\sum\limits_{i_2\in J_{i_1,T-1}}\ldots \sum\limits_{i_{T}\in J_{i_{T-1},1}}\frac{\sigma^2}{b_{i,T}^2 b_{i_1,T-1}^2\ldots b_{i_{T-1},1}^2}\\
    &=& \sum\limits_{i_1\in J_{i,T}}\sum\limits_{i_2\in J_{i_1,T-1}}\ldots \sum\limits_{i_{T-1}\in J_{i_{T-2},2}}\frac{\sigma^2}{b_{i,T}^2 b_{i_1,T-1}^2\ldots b_{i_{T-2},2}^2b_{i_{T-1},1}}.
\end{eqnarray*}
Next, taking the full expectation from the both sides of the previous inequality and using the tower property, we obtain
\begin{equation}
     \EE\!\left[\!\left\|\theta_i^T - \EE_{\theta}\left[\theta_i^T\right]\right\|^2\!\right] \!\le\! \EE\!\left[\!\sum\limits_{i_1\in J_{i,T}}\sum\limits_{i_2\in J_{i_1,T-1}}\ldots \sum\limits_{i_{T-1}\in J_{i_{T-2},2}}\frac{\sigma^2}{b_{i,T}^2 b_{i_1,T-1}^2\ldots b_{i_{T-2},2}^2b_{i_{T-1},1}}\!\right]\!. \label{eq:rand_mix_thm_technical_1}
\end{equation}
Notice that $J_{i_k,T-k} \cap J_{i_{k+1},T-k-1} = \{i_{k+1}\}$ for all $k=0,\ldots,T-1$, where $i_0 = i$. Moreover, for $k_1, k_2 \in\{0,1,\ldots,T\}$, $k_1 < k_2$ either $J_{i_{k_1},T-k_1} \cap J_{i_{k_2},T-k_2} = \{k_2\}$ or $J_{i_{k_1},T-k_1} \cap J_{i_{k_2},T-k_2} = \varnothing$. The first situation is possible iff $i_{k_1} = i_{k_1+1} = \ldots i_{k_2-1}$.

Taking these observations about sets $J_{i_{k}, T-k}$ into account, we consider the sets $J_{i_k,T-k}' = J_{i_k,T-k}\setminus\{i_{k}\}$ for $k = 0, 1, \ldots, T-1$. These sets are pairwise disjoint and their cardinalities $b_{i_k,T-k}' = |J_{i_k,T-k}'|$ satisfy the following relations: $b_{i_k,T-k} = 1 + b_{i_k,T-k}' \ge \max\{1, b_{i_k,T-k}'\} =: \hat{b}_{i_k,T-k}$ for $k = 1, 2, \ldots, T-1$. Moreover, $b_{i,T}', b_{i_1,T-1}',\ldots, b_{i_{T-1},1}'$ are independent random variables from the binomial distribution $\text{Binom}(M-1, p)$. Finally, we notice that the number of terms in \eqref{eq:rand_mix_thm_technical_1} is upper-bounded by $M^{T-1}$, since $|J_{i,t}| \le M$ for all $i = 1,\ldots,N$ and $t=0,\ldots,T$.

Putting all together, we obtain
\begin{eqnarray*}
    \EE\left[\left\|\theta_i^T - \EE_{\theta}\left[\theta_i^T\right]\right\|^2\right] &\le& \EE\left[\sum\limits_{i_1\in J_{i,T}}\sum\limits_{i_2\in J_{i_1,T-1}}\ldots \sum\limits_{i_{T-1}\in J_{i_{T-2},2}}\frac{\sigma^2}{\hat b_{i,T}^2 \hat b_{i_1,T-1}^2\ldots \hat b_{i_{T-2},2}^2\hat b_{i_{T-1},1}}\right]\\
    &\le& M^{T-1}\sigma^2\EE\left[\frac{1}{\hat\xi_{1}^2 \hat\xi_{2}^2\ldots \hat\xi_{T-1}^2\hat\xi_{T}}\right]\\
    &=& M^{T-1}\sigma^2\EE\left[\frac{1}{\hat\xi_{1}^2}\right]\EE\left[\frac{1}{\hat\xi_{2}^2}\right]\ldots \EE\left[\frac{1}{\hat\xi_{T-1}^2}\right]\EE\left[\frac{1}{\hat\xi_{T}}\right],
\end{eqnarray*}
where $\hat \xi_k^2 = \max\{1,\xi_1^2\}$ for $k=1,\ldots,T$ and $\xi_1,\ldots,\xi_T$ are i.i.d.\ random variables having the binomial distribution $\text{Binom}(M-1, p)$. Then one can simplify the inequality above using Lemma~\ref{lem:ode_lemma} and get
\begin{eqnarray*}
    \EE\left[\left\|\theta_i^T - \EE_{\theta}\left[\theta_i^T\right]\right\|^2\right] &\le& M^{T-1}\sigma^2 m_1(M-1,p)\left(m_2(M-1,p)\right)^{T-1},
\end{eqnarray*}
where functions $m_1(M,p)$ and $m_2(M,p)$ are defined in \eqref{eq:binom_first_inverse_moment} and \eqref{eq:binom_second_inverse_moment} respectively.

Next, we simplify the obtained upper bound under the assumption that $M$ and $p$ are not too small; specifically, $M\ge 11$ and $p\ge \nicefrac{2}{3}$. From \eqref{eq:binom_first_inverse_moment}, we have
\begin{eqnarray*}
    m_1(M-1,p) &=& (1-p)^{M-1} + \sum\limits_{i=1}^{M-1}\frac{1}{i}\left((1-p)^{M-1-i} - (1-p)^{M-1}\right)\\
    &\le& (1-p)^{M-1}\sum\limits_{i=1}^{M-1}\frac{1}{i(1-p)^{i}}.
\end{eqnarray*}
Since
\begin{equation*}
    \frac{1}{(k+1)(1-p)^{k+1}}\cdot\frac{k(1-p)^k}{1} = \frac{k}{(k+1)(1-p)} \xrightarrow[k\to\infty]{}\frac{1}{1-p} \ge 3,
\end{equation*}
we have
\begin{equation*}
    (1-p)^{M-1}\sum\limits_{i=1}^{M-1}\frac{1}{i(1-p)^{i}} = \Theta\left((1-p)^M\cdot\frac{1}{M(1-p)^M}\right) = \Theta\left(\frac{1}{M}\right).
\end{equation*}
Using simple algebra, one can prove that for $M\ge 11$ and $p \ge\nicefrac{2}{3}$ the following inequality holds:
\begin{equation*}
    m_1(M-1,p)\le (1-p)^{M-1}\sum\limits_{i=1}^{M-1}\frac{1}{i(1-p)^{i}} \le \frac{2}{M}.
\end{equation*}
Similarly, we analyze $m_2(M-1, p)$:
\begin{eqnarray*}
    m_2(M-1,p) &=& (1-p)^{M-1} + \sum\limits_{i=1}^{M-1}\frac{1}{i}\left((1-p)^{M-1-i} - (1-p)^{M-1}\right)\sum\limits_{j=i}^{M-1}\frac{1}{j}\\
    &\le& (1-p)^{M-1}\sum\limits_{i=1}^{M-1}\frac{1}{i(1-p)^i}\sum\limits_{j=i}^{M-1}\frac{1}{j}.
\end{eqnarray*}
Since
\begin{eqnarray*}
    \frac{\frac{1}{k(1-p)^k}\sum\limits_{j=k}^{M-1}\frac{1}{j}}{\frac{1}{(k-1)(1-p)^{k-1}}\sum\limits_{j=k-1}^{M-1}\frac{1}{j}} &=& \frac{(k-1)\sum\limits_{j=k}^{M-1}\frac{1}{j}}{k(1-p)\left(\frac{1}{k-1} + \sum\limits_{j=k}^{M-1}\frac{1}{j}\right)} \ge \frac{3(k-1)\cdot\frac{1}{k}}{k\left(\frac{1}{k-1}+\frac{1}{k}\right)}\\
    &=& \frac{3(k-1)^2}{k(2k-1)}\xrightarrow[k\to\infty]{}  \frac{3}{2},
\end{eqnarray*}
we have
\begin{equation*}
    (1-p)^{M-1}\sum\limits_{i=1}^{M-1}\frac{1}{i(1-p)^i}\sum\limits_{j=i}^{M-1}\frac{1}{j} = \Theta\left((1-p)^M\cdot\frac{1}{M^2(1-p)^M}\right) = \Theta\left(\frac{1}{M^2}\right).
\end{equation*}
Next, one can prove with simple algebra that for $M\ge 11$ and $p \ge\nicefrac{2}{3}$ the following inequality holds:
\begin{equation*}
    m_2(M-1,p) \le (1-p)^{M-1}\sum\limits_{i=1}^{M-1}\frac{1}{i(1-p)^i}\sum\limits_{j=i}^{M-1}\frac{1}{j} \le \frac{3}{M^2}.
\end{equation*}
Plugging the obtained upper bounds for $m_1(M-1,p)$ and $m_2(M-1,p)$ in \eqref{eq:variance_bound_supp}, we obtain \eqref{eq:variance_bound_2_supp}.
\end{proof}


\section{Convergence Proofs of Moshpit SGD}\label{sect:missing_proofs_local_sgd}
In this section, we provide the complete statements of the theorems establishing the convergence of Moshpit SGD together with the full proofs. First, we introduce all necessary definitions, basic inequalities and auxiliary lemmas; then we prove the convergence in strongly convex and convex cases; lastly, we provide the proofs for the non-convex case.

\subsection{Definitions, Basic Facts and Auxiliary Results}\label{sect:basic_facts}


Below we provide several classical definitions and results which are used in our proofs.

\subsubsection{Standard Definitions from Optimization Theory}

\begin{definition}[$L$-smoothness]\label{def:L_smoothness}
A function $f:\R^n \to \R$ is called $L$-smooth if for all $x,y\in \R^n$, the following inequality holds:
\begin{equation}
    \|\nabla f(x) - \nabla f(y)\| \le L\|x-y\|.\label{eq:L_smoothness_def}
\end{equation}
\end{definition}
If the function $f$ is $L$-smooth, then for all $x,y\in\R^n$
\begin{equation}
    f(y) \le f(x) + \langle\nabla f(x), y-x \rangle + \frac{L}{2}\|y-x\|^2. \label{eq:L_smoothness_cor}
\end{equation}
Next, if $f$ is additionally convex and $x^*$ is its minimizer, then for all $x\in\R^d$
\begin{equation}
    \|\nabla f(x)\|^2 \le 2L\left(f(x) - f(x^*)\right). \label{eq:L_smoothness_cor_2}
\end{equation}


\begin{definition}[$\mu$-strong convexity]\label{def:str_cvx}
    A differentiable function $f:\R^n \to\R$ is called $\mu$-strongly convex if there exists a constant $\mu \ge 0$ such that for all $x,y\in \R^n$
    \begin{equation}
        f(y) \ge f(x) + \langle\nabla f(x), y-x \rangle + \frac{\mu}{2}\|y-x\|^2. \label{eq:str_cvx_def}
    \end{equation}
\end{definition}

\subsubsection{Basic Facts}
For all $a,b,\theta_1,\ldots,\theta_N\in\R^n$ and $\alpha > 0$, the following inequalities hold:
\begin{eqnarray}
    \|a+b\|^2 &\le& 2\|a\|^2 + 2\|b\|^2, \label{eq:a+b}\\
    \left\|\frac{1}{N}\sum\limits_{i=1}^N\theta_i\right\|^2 &\le& \frac{1}{N}\sum\limits_{i=1}^N\|\theta_i\|^2, \label{eq:jensen_ineq}\\
    \langle a,b\rangle &\le& \frac{\|a\|^2}{2\alpha} + \frac{\alpha\|b\|^2}{2}. \label{eq:young_inequality}
\end{eqnarray}

\subsubsection{Properties of Expectation}
\textbf{Variance decomposition.} For a random vector $\eta \in \R^d$ and any deterministic vector $x \in \R^d$, the variance satisfies
\begin{equation}\label{eq:variance_decomposition}
	\EE\left[\left\|\eta - \EE\eta\right\|^2\right] = \EE\left[\|\eta-x\|^2\right] - \left\|\EE\eta - x\right\|^2
\end{equation}

\textbf{Tower property of expectation.} For any random variables $\xi,\eta\in \R^d$ we have
\begin{equation}
	\EE\left[\xi\right] = \EE\left[\EE\left[\xi\mid \eta\right]\right]\label{eq:tower_property}
\end{equation}
under the assumption that $\EE[\xi]$ and $\EE\left[\EE\left[\xi\mid \eta\right]\right]$ are well-defined.

\subsubsection{Auxiliary Results}
For the readers' convenience, we list all auxiliary results that we use in our proofs below. The first result is classical and establishes that the gradient descent step is a contractive operator.
\begin{lemma}[Lemma 6 from \cite{karimireddy2020scaffold}]\label{lem:gd_contraction}
    For any $L$-smooth and $\mu$-strongly convex function $f:\R^n\to\R$, points $x,y\in \R^n$, and stepsize $\gamma \in (0,\nicefrac{1}{L}]$, the following inequality holds:
    \begin{equation}
        \|x - \gamma\nabla f(x) - y + \gamma\nabla f(y)\|^2 \le (1-\gamma\mu)\|x-y\|^2. \label{eq:gd_contraction}
    \end{equation}
\end{lemma}

The next two lemmas are useful for estimating typical recurrences appearing in the analysis.
\begin{lemma}[Lemma~I.2 from \cite{gorbunov2020local}]\label{lem:lemma_i_2_gorbunov}
    Let $\{r_k\}_{k\ge 0}$ satisfy
    \begin{equation*}
        r_K \le \frac{a}{\gamma W_K} + c_1\gamma + c_2\gamma^2
    \end{equation*}
    for all $K \ge 0$ with some constants $a,c_2 \ge 0$, $c_1 \ge 0$, where $w_k = (1-\gamma\mu(1-\delta_{pv,1}))^{-(k+1)}$, $W_K = \sum_{k=0}^Kw_k$, $\mu > 0$, $\delta_{pv,1}\in [0,1)$ and $\gamma \le \gamma_0$ for some $\gamma_0 > 0$, $\gamma_0 \le \nicefrac{1}{\mu(1-\delta_{pv,1})}$. Then, for all $K$ such that
    \begin{align*}
        \text{either  } & \frac{\ln\left(\max\left\{2, \min\left\{\nicefrac{a\mu^2(1-\delta_{pv,1})^2K^2}{c_1},\nicefrac{a\mu^3(1-\delta_{pv,1})^3K^3}{c_2}\right\}\right\}\right)}{K} \le 1\\
        \text{or  } & \gamma_0 \le \frac{\ln\left(\max\left\{2, \min\left\{\nicefrac{a\mu^2(1-\delta_{pv,1})^2K^2}{c_1},\nicefrac{a\mu^3(1-\delta_{pv,1})^3K^3}{c_2}\right\}\right\}\right)}{(1-\delta_{pv,1})\mu K}
    \end{align*}
    and
    \begin{equation*}
        \gamma = \min\left\{\gamma_0, \frac{\ln\left(\max\left\{2, \min\left\{\nicefrac{a\mu^2(1-\delta_{pv,1})^2K^2}{c_1},\nicefrac{a\mu^3(1-\delta_{pv,1})^3K^3}{c_2}\right\}\right\}\right)}{(1-\delta_{pv,1})\mu K}\right\}
    \end{equation*}
    we have that
    \begin{equation*}
        r_K = \widetilde{\cO}\left(\frac{a}{\gamma_0}\exp\left(-\gamma_0\mu(1-\delta_{pv,1})K\right) + \frac{c_1}{(1-\delta_{pv,1})\mu K} + \frac{c_2}{(1-\delta_{pv,1})^2\mu^2 K^2}\right).
    \end{equation*}
\end{lemma}

\begin{lemma}[Lemma~I.3 from \cite{gorbunov2020local}]\label{lem:lemma_i_3_gorbunov}
    Let $\{r_k\}_{k\ge 0}$ satisfy
    \begin{equation*}
        r_K \le \frac{a}{\gamma K} + c_1\gamma + c_2\gamma^2
    \end{equation*}
    for all $K \ge 0$ with some constants $a,c_2 \ge 0$, $c_1 \ge 0$ where $\gamma \le \gamma_0$ for some $\gamma_0 > 0$. Then for all $K$ and
    \begin{equation*}
        \gamma = \min\left\{\gamma_0, \sqrt{\frac{a}{c_1 K}}, \sqrt[3]{\frac{a}{c_2 K}}\right\}
    \end{equation*}
    we have that
    \begin{equation*}
        r_K = \cO\left(\frac{a}{\gamma_0 K} + \sqrt{\frac{ac_1}{K}} + \frac{\sqrt[3]{a^2c_2}}{K^{\nicefrac{2}{3}}}\right).
    \end{equation*}
\end{lemma}

Finally, the lemma below is useful for our convergence analysis in the non-convex case.
\begin{lemma}[Lemma~I.1 from \cite{gorbunov2020local}]\label{lem:lemma_i_1_gorbunov}
	For any $\tau$ random vectors $\xi_1,\ldots,\xi_\tau\in\R^d$ such that $\forall t=2,\ldots,\tau$ the random vector $\xi_t$ depends on $\xi_{1},\ldots,\xi_{t-1}$ and does not depend on $\xi_{t+1},\ldots,\xi_{\tau}$ the following inequality holds
	\begin{equation}
		\EE\left[\left\|\sum\limits_{t=1}^\tau\xi_t\right\|^2\right] \le e\tau\sum\limits_{t=1}^\tau\EE\left[\left\|\EE_t[\xi_{t}]\right\|^2\right] + e\sum\limits_{t=1}^\tau\EE\left[\left\|\xi_t-\EE_t[\xi_{t}]\right\|^2\right], \label{eq:lemma_i_1_gorbunov}
	\end{equation}
	where $\EE_t[\cdot]$ denotes the conditional expectation $\EE[\ \cdot\mid \xi_{t-1},\ldots,\xi_1]$.
\end{lemma}

\subsection{Convex Case}
In this section, we give the full proof of Theorem~\ref{thm:cvx_convergence} about the convergence of Moshpit SGD for convex and strongly convex problems. The scheme of the proof follows the similar steps as in the state-of-the-art analysis of Local-SGD \cite{khaled2020tighter,woodworth2020local,gorbunov2020local}. We start with the following lemma:
\begin{lemma}\label{lem:key_lemma_cvx}
    Let $f_1 = \ldots = f_N = f$, function $f$ be $\mu$-strongly convex (Def.~\ref{def:str_cvx}) and $L$-smooth (see Def.~\ref{def:L_smoothness}), and Assumptions~\ref{as:bounded_var}~and~\ref{as:averaging_quality} hold with $\Delta_{pv}^k = \delta_{pv,1}\gamma\mu\EE[\|\theta^k-\theta^*\|^2] + \gamma^2\delta_{pv,2}^2$ and $\widetilde{\theta} = \theta^*$, where $\theta^* \in \argmin_{\theta\in\R^n} f(\theta)$ and $\delta_{pv,1}\in [0,1)$, $\delta_{pv,2}\ge 0$. Then, for any $k \ge 0$ the iterates produced by Moshpit SGD with $\gamma \le \nicefrac{1}{4L}$ satisfy
    \begin{eqnarray}
        \gamma\EE\left[f(\theta^k) - f(\theta^*)\right] &\le& (1-\gamma\mu(1-\delta_{pv,1}))\EE\left[\|\theta^k - \theta^*\|^2\right] - \EE\left[\|\theta^{k+1} - \theta^*\|^2\right]\notag\\
        &&\quad+ \frac{3L\gamma}{2}\EE[V_k] + \gamma^2\left(\frac{\sigma^2}{N_{\min}} + \delta_{pv,2}^2\right),\label{eq:key_lemma_cvx}
    \end{eqnarray}
    where $V_k = \frac{1}{N_k}\sum_{i\in P_k}\|\theta_i^k - \theta^k\|^2$ and $\theta^k = \frac{1}{N_k}\sum_{i\in P_k}\theta_i^k$.
\end{lemma}
\begin{proof}
Recall that Assumption~\ref{as:averaging_quality} with $\Delta_{pv}^k = \delta_{pv,1}\gamma\mu\EE[\|\theta^k-\theta^*\|^2] + \gamma^2\delta_{pv,2}^2$ and $\widetilde{\theta} = \theta^*$ states
\begin{equation}
    \EE\left[\langle\theta^{k+1} - \widehat{\theta}^{k+1}, \theta^{k+1}+\widehat{\theta}^{k+1} - 2\theta^*\rangle\right] \le \delta_{pv,1}\gamma\mu\EE[\|\theta^k-\theta^*\|^2] + \gamma^2\delta_{pv,2}^2, \label{eq:key_lemma_cvx_tech_1}
\end{equation}
where $\widehat \theta^{k+1} = \frac{1}{N_{k}}\sum_{i\in P_{k}}(\theta_i^{k}-\gamma g_i^k)$. Next, the definition of $\widehat \theta^{k+1}$ implies
\begin{equation}
    \widehat \theta^{k+1} = \frac{1}{N_k}\sum\limits_{i\in P_{k}}\theta_i^{k} - \frac{\gamma}{N_k}\sum\limits_{i\in P_{k}} g_i^k = \theta^k - \gamma g^k,\notag
\end{equation}
where $g^k = \frac{1}{N_k}\sum_{i\in P_k}g_i^k$. Using this, we derive
\begin{eqnarray}
    \|\theta^{k+1} - \theta^*\|^2 &=& \|\widehat{\theta}^{k+1} - \theta^*\|^2 + 2\langle \theta^{k+1} - \widehat{\theta}^{k+1}, \widehat{\theta}^{k+1} - \theta^* \rangle + \|\theta^{k+1} - \widehat{\theta}^{k+1}\|^2\notag\\
    &=& \|\theta^k - \theta^* - \gamma g^k\|^2 +  \langle\theta^{k+1} - \widehat{\theta}^{k+1}, \theta^{k+1}+\widehat{\theta}^{k+1} - 2\theta^*\rangle \notag\\
    &=& \|\theta^k - \theta^*\|^2 -2\gamma\langle\theta^k - \theta^*, g^k\rangle + \gamma^2\|g^k\|^2\notag\\
    &&\quad +  \langle\theta^{k+1} - \widehat{\theta}^{k+1}, \theta^{k+1}+\widehat{\theta}^{k+1} - 2\theta^*\rangle. \notag
\end{eqnarray}
Taking the conditional expectation $\EE\left[\ \cdot \mid \theta^k\right] := \EE\left[\ \cdot \mid P_k, \theta_i^k, i\in P_k\right]$ from the both sides of the previous equation and using Assumption~\ref{as:bounded_var}, we obtain
\begin{eqnarray}
    \EE\left[\|\theta^{k+1} - \theta^*\|^2\mid \theta^k\right] &=& \|\theta^k - \theta^*\|^2 -2\gamma\left\langle\theta^k - \theta^*, \frac{1}{N_k}\sum\limits_{i\in P_k}\nabla f(\theta_i^k)\right\rangle\notag\\
    &&\quad + \gamma^2\EE\left[\left\|\frac{1}{N_k}\sum\limits_{i\in P_k}g_i^k\right\|^2\mid \theta^k\right] \notag\\
    &&\quad +  \EE\left[\langle\theta^{k+1} - \widehat{\theta}^{k+1}, \theta^{k+1}+\widehat{\theta}^{k+1} - 2\theta^*\rangle\mid \theta^k\right]. \label{eq:key_lemma_cvx_tech_2}
\end{eqnarray}
Next, we estimate the second and the third terms in the right-hand side of \eqref{eq:key_lemma_cvx_tech_2}. First,
\begin{eqnarray}
    -2\gamma\left\langle\theta^k - \theta^*, \frac{1}{N_k}\sum\limits_{i\in P_k}\nabla f(\theta_i^k)\right\rangle &=& \frac{2\gamma}{N_k}\sum\limits_{i\in P_k}\left(\langle\theta^* - \theta_i^k, \nabla f(\theta_i^k) \rangle + \langle\theta_i^k - \theta^k, \nabla f(\theta_i^k) \rangle \right)\notag\\
    &\overset{\eqref{eq:str_cvx_def},\eqref{eq:L_smoothness_cor}}{\le}& \frac{2\gamma}{N_k}\sum\limits_{i\in P_k}\left( f(\theta^*) - f(\theta_i^k) - \frac{\mu}{2}\|\theta_i^k - \theta^*\|^2\right)\notag\\
    &&\quad + \frac{2\gamma}{N_k}\sum\limits_{i\in P_k}\left(f(\theta_i^k) - f(\theta^k) + \frac{L}{2}\|\theta_i^k - \theta^k\|^2\right)\notag\\
    &\overset{\eqref{eq:jensen_ineq}}{\le}& 2\gamma\left(f(\theta^*) - f(\theta^k)\right) -\gamma\mu\|\theta^k - \theta^*\|^2 + L\gamma V_k, \label{eq:key_lemma_cvx_tech_3}
\end{eqnarray}
where $V_k = \frac{1}{N_k}\sum_{i\in P_k}\|\theta_i^k - \theta^k\|^2$. Secondly, since stochastic gradients $\{g_i^k\}_{i\in P_k}$ are computed independently, we get
\begin{eqnarray}
    \gamma^2\EE\left[\left\|\frac{1}{N_k}\sum\limits_{i\in P_k}g_i^k\right\|^2\mid \theta^k\right] &\overset{\eqref{eq:variance_decomposition}}{=}& \gamma^2\left\|\frac{1}{N_k}\sum\limits_{i\in P_k}\nabla f(\theta_i^k)\right\|^2\notag\\
    &&\quad + \gamma^2\EE\left[\left\|\frac{1}{N_k}\sum\limits_{i\in P_k}(g_i^k-\nabla f(\theta_i^k))\right\|^2\mid \theta^k\right]\notag\\
    &\overset{\eqref{eq:jensen_ineq}}{\le}& 2\gamma^2 \left\|\frac{1}{N_k}\sum\limits_{i\in P_k}(\nabla f(\theta_i^k)-\nabla f(\theta^k))\right\|^2 + 2\gamma^2\|\nabla f(\theta^k)\|^2 \notag\\
    &&\quad + \frac{\gamma^2}{N_k^2}\sum\limits_{i\in P_k}\EE\left[\|g_i^k - \nabla f(\theta_i^k)\|^2\mid \theta^k\right]\notag\\
    &\overset{\eqref{eq:jensen_ineq},\eqref{eq:L_smoothness_cor_2},\eqref{eq:bounded_variance}}{\le}& \frac{2\gamma^2}{N_k}\sum\limits_{i\in P_k}\|\nabla f(\theta_i^k)-\nabla f(\theta^k)\|^2 \notag\\
    &&\quad + 4L\gamma^2\left(f(\theta^k) - f(\theta^*)\right) + \frac{\gamma^2\sigma^2}{N_k}\notag\\
    &\overset{\eqref{eq:L_smoothness_def}}{\le}& \underbrace{\frac{2L^2\gamma^2}{N_k}\sum\limits_{i\in P_k}\|\theta_i^k - \theta^k\|^2}_{2L^2\gamma^2 V_k}\notag\\
    &&\quad + 4L\gamma^2\left(f(\theta^k) - f(\theta^*)\right) + \frac{\gamma^2\sigma^2}{N_{\min}}. \label{eq:key_lemma_cvx_tech_4}
\end{eqnarray}
Plugging \eqref{eq:key_lemma_cvx_tech_3} and \eqref{eq:key_lemma_cvx_tech_4} in \eqref{eq:key_lemma_cvx_tech_2}, we obtain
\begin{eqnarray}
    \EE\left[\|\theta^{k+1} - \theta^*\|^2\mid \theta^k\right] &\le& (1-\gamma\mu)\|\theta^k - \theta^*\|^2 - 2\gamma\left(1 - 2L\gamma\right)\left(f(\theta^k) - f(\theta^*)\right)\notag\\
    &&\quad + L\gamma\left(1+2L\gamma\right)V_k + \frac{\gamma^2\sigma^2}{N_{\min}} \notag\\
    &&\quad +  \EE\left[\langle\theta^{k+1} - \widehat{\theta}^{k+1}, \theta^{k+1}+\widehat{\theta}^{k+1} - 2\theta^*\rangle\mid \theta^k\right], \notag
\end{eqnarray}
and
\begin{eqnarray}
    \EE\left[\|\theta^{k+1} - \theta^*\|^2\right] &\overset{\eqref{eq:key_lemma_cvx_tech_1}}{\le}& (1-\gamma\mu(1-\delta_{pv,1}))\EE\left[\|\theta^k - \theta^*\|^2\right] - 2\gamma\left(1 - 2L\gamma\right)\EE\left[f(\theta^k) - f(\theta^*)\right]\notag\\
    &&\quad+ L\gamma\left(1+2L\gamma\right)\EE[V_k] + \gamma^2\left(\frac{\sigma^2}{N_{\min}} + \delta_{pv,2}^2\right)\notag\\
    &\le& (1-\gamma\mu(1-\delta_{pv,1}))\EE\left[\|\theta^k - \theta^*\|^2\right] - \gamma\EE\left[f(\theta^k) - f(\theta^*)\right]\notag\\
    &&\quad+ \frac{3L\gamma}{2}\EE[V_k] + \gamma^2\left(\frac{\sigma^2}{N_{\min}} + \delta_{pv,2}^2\right),\notag
\end{eqnarray}
where in the last inequality we use $\gamma \le \nicefrac{1}{4L}$.
\end{proof}

Next, we estimate the term $\EE[V_k]$ measuring the expected dissimilarity between local iterates and their global average at iteration $k$.

\begin{lemma}\label{lem:V_k_lemma_cvx}
    Let $f_1 = \ldots = f_N = f$, function $f$ be $\mu$-strongly convex (Def.~\ref{def:str_cvx}) and $L$-smooth (see Def.~\ref{def:L_smoothness}), and Assumptions~\ref{as:bounded_var}~and~\ref{as:averaging_quality} hold with $\Delta_{pv}^k = \delta_{pv,1}\gamma\mu\EE[\|\theta^k-\theta^*\|^2] + \gamma^2\delta_{pv,2}^2$ and $\widetilde{\theta} = \theta^*$, where $\theta^* \in \argmin_{\theta\in\R^n} f(\theta)$ and $\delta_{pv,1}\in [0,1)$, $\delta_{pv,2}\ge 0$. Then, for any $k \ge 0$ the iterates produced by Moshpit SGD with $\gamma \le \nicefrac{1}{4L}$ satisfy
    \begin{equation}
        \EE[V_k] \le 2\gamma^2\left(4\delta_{aq}^2 + (\tau-1)\sigma^2\right), \label{eq:V_k_bound_cvx}
    \end{equation}
    where $V_k = \frac{1}{N_k}\sum_{i\in P_k}\|\theta_i^k - \theta^k\|^2$ and $\theta^k = \frac{1}{N_k}\sum_{i\in P_k}\theta_i^k$.
\end{lemma}
\begin{proof}
    First of all, if $k = a\tau$ for some integer $a\ge 0$, then \eqref{eq:V_k_bound_cvx} follows from Assumption~\ref{as:averaging_quality} (eq.~\eqref{eq:quality_of_avg}). Therefore, we consider such $k$ that $k = a\tau + t'$ for some $t'\in (0,\tau)$. Then, for any $i,j \in P_{k}$, $i\neq j$
    \begin{eqnarray*}
        \EE\left[\|\theta_i^k - \theta_j^k\|^2\mid \theta^{k-1}\right] &=& \EE\left[\|\theta_i^{k-1} - \gamma g_i^{k-1} - \theta_j^{k-1} + \gamma g_{j}^{k-1}\|^2\mid \theta^{k-1}\right]\\
        &\overset{\eqref{eq:variance_decomposition}}{=}& \|\theta_i^{k-1} - \gamma \nabla f(\theta_i^{k-1}) - \theta_j^{k-1} + \gamma \nabla f(\theta_j^{k-1})\|^2\\
        &&\quad +\gamma^2\EE\left[\|g_i^{k-1} - \nabla f(\theta_i^{k-1}) + g_{j}^{k-1} - \nabla f(\theta_j^{k-1})\|^2\mid \theta^{k-1}\right].
    \end{eqnarray*}
    Using Lemma~\ref{lem:gd_contraction} and independence of $g_i^{k-1}$ and $g_j^{k-1}$ for given $\theta_i^{k-1}, \theta_j^{k-1}$, $i\neq j$ we derive
    \begin{eqnarray*}
        \EE\left[\|\theta_i^k - \theta_j^k\|^2\mid \theta^{k-1}\right] &\overset{\eqref{eq:gd_contraction}}{\le}& (1-\gamma\mu)\|\theta_i^{k-1} - \theta_j^{k-1}\|^2 +\gamma^2\EE\left[\|g_i^{k-1} - \nabla f(\theta_i^{k-1})\|^2\mid \theta^{k-1}\right]\\
        &&\quad +\gamma^2\EE\left[\|g_j^{k-1} - \nabla f(\theta_j^{k-1})\|^2\mid \theta^{k-1}\right]\\
        &\overset{\eqref{eq:bounded_variance}}{\le}& (1-\gamma\mu)\|\theta_i^{k-1} - \theta_j^{k-1}\|^2 + 2\gamma^2\sigma^2,
    \end{eqnarray*}
    from which we get the following: 
    \begin{equation}
        \EE_g\left[\|\theta_i^k - \theta_j^k\|^2\right] \le (1-\gamma\mu)\EE_g\left[\|\theta_i^{k-1} - \theta_j^{k-1}\|^2\right] + 2\gamma^2\sigma^2 \le \EE_g\left[\|\theta_i^{k-1} - \theta_j^{k-1}\|^2\right] + 2\gamma^2\sigma^2.\notag %
    \end{equation}
    Here, $\EE_g[\cdot]$ denotes the expectation conditioned on $\{P_k\}_{k = a\tau}^{(a+1)\tau-1}$. Unrolling the recurrence, we get
    \begin{eqnarray}
        \EE_g\left[\|\theta_i^k - \theta_j^k\|^2\right] &\le& \EE_g\left[\|\theta_i^{a\tau} - \theta_j^{a\tau}\|^2\right] + 2(k-a\tau)\gamma^2\sigma^2\notag \\
        &\le& \EE_g\left[\|\theta_i^{a\tau} - \theta_j^{a\tau}\|^2\right] + 2(\tau-1)\gamma^2\sigma^2.\label{eq:V_k_lemma_technical_1}
    \end{eqnarray}
    Using this, we estimate $\EE_{g}[V_k]$:
    \begin{eqnarray*}
        \EE_g[V_k] &=& \frac{1}{N_k}\sum\limits_{i\in P_k}\EE_g\left[\left\|\theta_i^k - \frac{1}{N_k}\sum\limits_{j\in P_k}\theta_j^k\right\|^2\right] \overset{\eqref{eq:jensen_ineq}}{\le} \frac{1}{N_k^2}\sum\limits_{i,j \in P_k}\EE_g\left[\|\theta_i^k - \theta_j^k\|^2\right]\\
        &\overset{\eqref{eq:V_k_lemma_technical_1}}{\le}& \frac{1}{N_k^2}\sum\limits_{i,j \in P_k}\EE_g\left[\|\theta_i^{a\tau} - \theta_j^{a\tau}\|^2\right] + 2(\tau-1)\gamma^2\sigma^2 \\
        &\overset{\eqref{eq:a+b}}{\le}& \frac{2}{N_k^2}\sum\limits_{i,j \in P_k}\left(\EE_g\left[\|\theta_i^{a\tau} - \theta^{a\tau}\|^2\right] + \EE_g\left[\|\theta_j^{a\tau} - \theta^{a\tau}\|^2\right]\right) + 2(\tau-1)\gamma^2\sigma^2\\
        &=& \frac{4}{N_k}\sum\limits_{i\in P_k}\EE_g\left[\|\theta_i^{a\tau} - \theta^{a\tau}\|^2\right]+ 2(\tau-1)\gamma^2\sigma^2\\
        &\le& \frac{4}{N_{a\tau}}\cdot\frac{N_{a\tau}}{N_k}\sum\limits_{i\in P_{a\tau}}\EE_g\left[\|\theta_i^{a\tau} - \theta^{a\tau}\|^2\right]+ 2(\tau-1)\gamma^2\sigma^2\\
        &\le& \EE_g\left[\frac{8}{N_{a\tau}}\sum\limits_{i\in P_{a\tau}}\|\theta_i^{a\tau} - \theta^{a\tau}\|^2\right]+ 2(\tau-1)\gamma^2\sigma^2,
    \end{eqnarray*}
    where in the last inequality we use $2N_{(a+1)\tau} = 2|P_{(a+1)\tau}| \ge |P_{a\tau}| = N_{a\tau}$ and $|N_k|\le |N_{k-1}|$ following from Assumption~\ref{as:averaging_quality}. Finally, we take the full expectation from the previous inequality:
    \begin{eqnarray*}
        \EE[V_k] &\overset{\eqref{eq:tower_property}}{\le}& 8\EE\left[\frac{1}{N_{a\tau}}\sum\limits_{i\in P_{a\tau}}\|\theta_i^{a\tau} - \theta^{a\tau}\|^2\right]+ 2(\tau-1)\gamma^2\sigma^2 \overset{\eqref{eq:quality_of_avg}}{\le} 2\gamma^2\left(4\delta_{aq}^2 + (\tau-1)\sigma^2\right).
    \end{eqnarray*}
    This finishes the proof.
\end{proof}

Combining Lemmas~\ref{lem:key_lemma_cvx}~and~\ref{lem:V_k_lemma_cvx}, we get the following result:
\begin{theorem}[Theorem~\ref{thm:cvx_convergence}, convergence in the convex case]\label{thm:cvx_convergence_supp}
    Let $f_1 = \ldots = f_N = f$ be $\mu$-strongly convex (Def.~\ref{def:str_cvx}) and $L$-smooth (see Def.~\ref{def:L_smoothness}), and Assumptions~\ref{as:bounded_var}~and~\ref{as:averaging_quality} hold with $\Delta_{pv}^k = \delta_{pv,1}\gamma\mu\EE[\|\theta^k-\theta^*\|^2] + \gamma^2\delta_{pv,2}^2$ and $\widetilde{\theta} = \theta^*$, where $\theta^* \in \argmin_{\theta\in\R^n} f(\theta)$ and $\delta_{pv,1}\in [0,1)$, $\delta_{pv,2}\ge 0$. Then, for any $K \ge 0$, the iterates produced by Moshpit SGD with $\gamma \le \nicefrac{1}{4L}$ satisfy
    \begin{eqnarray}
        \EE\left[f(\overline{\theta}^K) - f(\theta^*)\right] &\le& (1-\gamma\mu(1-\delta_{pv,1}))^K\frac{R_0^2}{\gamma}\notag\\
        &&\quad + \gamma\left(\frac{\sigma^2}{N_{\min}} + \delta_{pv,2}^2 + 3L\gamma\left(4\delta_{aq}^2 + (\tau-1)\sigma^2\right)\right), \label{eq:str_cvx_bound_supp}
    \end{eqnarray}
    when $\mu > 0$, and
    \begin{equation}
        \EE\left[f(\overline{\theta}^K) - f(\theta^*)\right] \le \frac{R_0^2}{\gamma K} + \gamma\left(\frac{\sigma^2}{N_{\min}} + \delta_{pv,2}^2 + 3L\gamma\left(4\delta_{aq}^2 + (\tau-1)\sigma^2\right)\right), \label{eq:cvx_bound_supp}
    \end{equation}
    when $\mu = 0$, where $R_0 = \|\theta^0 - \theta^*\|$, $\overline{\theta}^K = \frac{1}{W_K}\sum_{k=0}^Kw_k\theta^k = \frac{1}{W_K}\sum_{k=0}^K\frac{w_k}{N_k}\sum_{i\in P_k}\theta_i^k$, $w_k = (1-\gamma\mu(1-\delta_{pv,1}))^{-(k+1)}$, and $W_K = \sum_{k=0}^Kw_k$. That is, Moshpit SGD achieves $\EE[f(\overline{\theta}^K) - f(\theta^*)] \le \varepsilon$ after 
    \begin{equation}
        K = \widetilde{\cO}\left(\frac{L}{(1-\delta_{pv,1})\mu} +  \frac{\sigma^2}{N_{\min}(1-\delta_{pv,1})\mu\varepsilon} + \frac{\delta_{pv,2}^2}{(1-\delta_{pv,1})\mu\varepsilon} + \sqrt{\frac{L((\tau-1)\sigma^2+\delta_{aq}^2)}{(1-\delta_{pv,1})^2\mu^2\varepsilon}}\right)\label{eq:str_cvx_bound_2_supp}
    \end{equation}
    iterations with
    \begin{equation*}
        \gamma = \min\left\{\frac{1}{4L}, \frac{\ln\left(\max\left\{2, \min\left\{\frac{R_0^2\mu^2(1-\delta_{pv,1})^2K^2}{(\delta_{pv,2}^2 + \nicefrac{\sigma^2}{N_{\min}}) },\frac{R_0^2\mu^3(1-\delta_{pv,1})^3K^3}{3L\left(4\delta_{aq}^2 + (\tau-1)\sigma^2\right)}\right\}\right\}\right)}{(1-\delta_{pv,1})\mu K}\right\}
    \end{equation*}
    when $\mu > 0$, and after
    \begin{equation}
        K = \cO\left(\frac{LR_0^2}{\varepsilon} +  \frac{R_0^2\sigma^2}{N_{\min}\varepsilon^2} + \frac{R_0^2\delta_{pv,2}^2}{\varepsilon^2} + \frac{R_0^2\sqrt{L((\tau-1)\sigma^2+\delta_{aq}^2)}}{\varepsilon^{\nicefrac{3}{2}}}\right)\label{eq:cvx_bound_2_supp}
    \end{equation}
    iterations with
    \begin{equation*}
       \gamma = \min\left\{\frac{1}{4L} \sqrt{\frac{R_0}{(\delta_{pv,2}^2 + \nicefrac{\sigma^2}{N_{\min}})K}}, \sqrt[3]{\frac{R_0^2}{3L\left(4\delta_{aq}^2 + (\tau-1)\sigma^2\right) K}}\right\}
    \end{equation*}
    when $\mu = 0$.
\end{theorem}
\begin{proof}
    Plugging the result of Lemma~\ref{lem:V_k_lemma_cvx} in inequality \eqref{eq:key_lemma_cvx} from Lemma~\ref{lem:key_lemma_cvx}, we obtain
    \begin{eqnarray}
        \gamma\EE\left[f(\theta^k) - f(\theta^*)\right] &\le& (1-\gamma\mu(1-\delta_{pv,1}))\EE\left[\|\theta^k - \theta^*\|^2\right] - \EE\left[\|\theta^{k+1} - \theta^*\|^2\right]\notag\\
        &&\quad+ 3L\gamma^3\left(4\delta_{aq}^2 + (\tau-1)\sigma^2\right) + \gamma^2\left(\frac{\sigma^2}{N_{\min}} + \delta_{pv,2}^2\right).\notag
    \end{eqnarray}
    Next, we sum up these inequalities for $k=0,\ldots, K$ with weights $w_k = (1-\gamma\mu(1-\delta_{pv,1}))^{-(k+1)}$ and divide both sides by $\gamma W_K$, where $W_K = \sum_{k=0}^Kw_k$:
    \begin{eqnarray*}
        \frac{1}{W_K}\sum\limits_{k=0}^K w_k\EE\left[f(\theta^k) - f(\theta^*)\right] &\le& \frac{1}{\gamma W_K}\sum\limits_{k=0}^K(1-\gamma\mu(1-\delta_{pv,1}))w_k\EE\left[\|\theta^k - \theta^*\|^2\right]\notag\\
        &&\quad - \frac{1}{\gamma W_K}\sum\limits_{k=0}^K w_k\EE\left[\|\theta^{k+1} - \theta^*\|^2\right]\notag\\
        &&\quad+ \gamma\left(\frac{\sigma^2}{N_{\min}} + \delta_{pv,2}^2 + 3L\gamma\left(4\delta_{aq}^2 + (\tau-1)\sigma^2\right)\right)\\
        &=& \frac{1}{\gamma W_K}\sum\limits_{k=0}^K\left(w_{k-1}\EE\left[\|\theta^k - \theta^*\|^2\right] - w_k\EE\left[\|\theta^{k+1} - \theta^*\|^2\right]\right)\notag\\
        &&\quad+ \gamma\left(\frac{\sigma^2}{N_{\min}} + \delta_{pv,2}^2 + 3L\gamma\left(4\delta_{aq}^2 + (\tau-1)\sigma^2\right)\right)\\
        &=& \frac{w_{-1}\|\theta^0 - \theta^*\|^2 - w_K\EE\left[\|\theta^{K+1}-\theta^*\|^2\right]}{\gamma W_K}\\
        &&\quad+ \gamma\left(\frac{\sigma^2}{N_{\min}} + \delta_{pv,2}^2 + 3L\gamma\left(4\delta_{aq}^2 + (\tau-1)\sigma^2\right)\right)\\
        &\le& \frac{\|\theta^0 - \theta^*\|^2}{\gamma W_K} \\
        &&\quad + \gamma\left(\frac{\sigma^2}{N_{\min}} + \delta_{pv,2}^2 + 3L\gamma\left(4\delta_{aq}^2 + (\tau-1)\sigma^2\right)\right).
    \end{eqnarray*}
    Since $f$ is convex, we apply the Jensen's inquality
    \begin{eqnarray*}
        f\left(\frac{1}{W_K}\sum\limits_{k=0}^K w_k\theta^k\right) &\le& \frac{1}{W_K}\sum\limits_{k=0}^K w_k f(\theta^k)
    \end{eqnarray*}
    to the previous result and get
    \begin{eqnarray*}
        \EE\left[f(\overline{\theta}^K) - f(\theta^*)\right] &\le& \frac{R_0^2}{\gamma W_K} + \gamma\left(\frac{\sigma^2}{N_{\min}} + \delta_{pv,2}^2 + 3L\gamma\left(4\delta_{aq}^2 + (\tau-1)\sigma^2\right)\right),
    \end{eqnarray*}
    where $R_0 = \|\theta^0 - \theta^*\|$ and $\overline{\theta}^K = \frac{1}{W_K}\sum_{k=0}^Kw_k\theta^k = \frac{1}{W_K}\sum_{k=0}^K\frac{w_k}{N_k}\sum_{i\in P_k}\theta_i^k$. If $\mu > 0$, then $W_K \ge w_K \ge (1-\gamma\mu(1-\delta_{pv,1}))^{-K}$, implying \eqref{eq:str_cvx_bound_supp}. Next, $w_k = 1$ and $W_K = K$ when $\mu = 0$ gives \eqref{eq:cvx_bound_supp}. It remains to estimate the total number of iterations $K$ required by Moshpit SGD to find an $\varepsilon$-solution, i.e., to achieve $\EE[f(\overline{\theta}^K) - f(\theta^*)] \le \varepsilon$. Applying Lemma~\ref{lem:lemma_i_2_gorbunov} to \eqref{eq:str_cvx_bound_supp}, we get the following result: if $\mu > 0$ and 
    \begin{equation*}
        \gamma = \min\left\{\frac{1}{4L}, \frac{\ln\left(\max\left\{2, \min\left\{\frac{R_0^2\mu^2(1-\delta_{pv,1})^2K^2}{\delta_{pv,2}^2 + \nicefrac{\sigma^2}{N_{\min}} },\frac{R_0^2\mu^3(1-\delta_{pv,1})^3K^3}{3L\left(4\delta_{aq}^2 + (\tau-1)\sigma^2\right)}\right\}\right\}\right)}{(1-\delta_{pv,1})\mu K}\right\},
    \end{equation*}
    then $\EE\left[f(\overline{\theta}^K) - f(\theta^*)\right]$ equals
    \begin{equation*}
        \widetilde{\cO}\left(LR_0^2\exp\left(-\frac{\mu}{L}(1-\delta_{pv,1})K\right) + \frac{\delta_{pv,2}^2 + \nicefrac{\sigma^2}{N_{\min}}}{(1-\delta_{pv,1})\mu K} + \frac{L\left(\delta_{aq}^2 + (\tau-1)\sigma^2\right)}{(1-\delta_{pv,1})^2\mu^2 K^2}\right),
    \end{equation*}
    implying \eqref{eq:str_cvx_bound_2_supp}. Similarly, we apply Lemma~\ref{lem:lemma_i_3_gorbunov} to \eqref{eq:cvx_bound_supp} and get that for $\mu = 0$ and 
    \begin{equation*}
        \gamma = \min\left\{\frac{1}{4L} \sqrt{\frac{R_0}{(\delta_{pv,2}^2 + \nicefrac{\sigma^2}{N_{\min}})K}}, \sqrt[3]{\frac{R_0^2}{3L\left(4\delta_{aq}^2 + (\tau-1)\sigma^2\right) K}}\right\},
    \end{equation*}
    \begin{equation*}
        \EE\left[f(\overline{\theta}^K) - f(\theta^*)\right] = \cO\left(\frac{LR_0^2}{K} + \sqrt{\frac{R_0^2(\delta_{pv,2}^2 + \nicefrac{\sigma^2}{N_{\min}})}{K}} + \frac{\sqrt[3]{R_0^4L\left(\delta_{aq}^2 + (\tau-1)\sigma^2\right)}}{K^{\nicefrac{2}{3}}}\right),
    \end{equation*}
    implying \eqref{eq:cvx_bound_2_supp}.
\end{proof}








\subsection{Non-Convex Case}
In this section, we give the full proof of Theorem~\ref{thm:non_cvx_convergence} about convergence of Moshpit SGD for general non-convex problems. The proof follows the similar steps as in the state-of-the-art analysis of Local-SGD in non-convex case~\cite{li2019communication,koloskova2020unified}. We start with the following lemma:
\begin{lemma}\label{lem:key_lemma_non_cvx}
    Let $f_1 = \ldots = f_N = f$, function $f$ be $L$-smooth and bounded from below by $f_*$, and Assumptions~\ref{as:bounded_var}~and~\ref{as:averaging_quality} hold with $\Delta_{pv}^k = \delta_{pv,1}\gamma\EE[\|\nabla f(\theta^k)\|^2] + L\gamma^2\delta_{pv,2}^2$, $\delta_{pv,1}\in [0,\nicefrac{1}{2})$, $\delta_{pv,2}\ge 0$. Then, for any $K \ge 0$ the iterates produced by Moshpit SGD with $\gamma \le \nicefrac{(1-2\delta_{pv,1})}{8L}$ satisfy
    \begin{eqnarray}
         \frac{(1-2\delta_{pv,1})\gamma}{4}\sum\limits_{k=0}^{K-1}\EE\left[\|\nabla f(\theta^k)\|^2\right] &\le& f(\theta^0) - f_* + \gamma L^2\sum\limits_{k=0}^{K-1} \EE[V_k]\notag\\
         &&\quad + KL\gamma^2\left(\frac{\sigma^2}{N_{\min}} + \delta_{pv,2}^2\right),\label{eq:key_lemma_non_cvx}
    \end{eqnarray}
    where $V_k = \frac{1}{N_k}\sum_{i\in P_k}\|\theta_i^k - \theta^k\|^2$ and $\theta^k = \frac{1}{N_k}\sum_{i\in P_k}\theta_i^k$.
\end{lemma}
\begin{proof}
    Recall that Assumption~\ref{as:averaging_quality} with $\Delta_{pv}^k = \delta_{pv,1}\gamma\EE[\|\nabla f(\theta^k)\|^2] + L\gamma^2\delta_{pv,2}^2$ states
\begin{equation}
    \EE\left[\langle\nabla f(\theta^k), \theta^{k+1}-\widehat{\theta}^{k+1}\rangle + L\|\widehat{\theta}^{k+1} - \theta^{k+1}\|^2\right] \le \delta_{pv,1}\gamma\EE[\|\nabla f(\theta^k)\|^2] + L\gamma^2\delta_{pv,2}^2, \label{eq:key_lemma_non_cvx_tech_1}
\end{equation}
where $\widehat \theta^{k+1} = \frac{1}{N_{k}}\sum_{i\in P_{k}}(\theta_i^{k}-\gamma g_i^k)$. As for the convex case, the definition of $\widehat \theta^{k+1}$ implies
\begin{equation}
    \widehat \theta^{k+1} = \frac{1}{N_k}\sum\limits_{i\in P_{k}}\theta_i^{k} - \frac{\gamma}{N_k}\sum\limits_{i\in P_{k}} g_i^k = \theta^k - \gamma g^k,\notag
\end{equation}
where $g^k = \frac{1}{N_k}\sum_{i\in P_k}g_i^k$. Using this and L-smoothness of $f$, we derive
    \begin{eqnarray*}
        f(\theta^{k+1}) - f(\theta^k) &\overset{\eqref{eq:L_smoothness_cor}}{\le}& \langle\nabla f(\theta^k), \theta^{k+1} - \theta^k \rangle + \frac{L}{2}\|\theta^{k+1} - \theta^k\|^2\\
        &\overset{\eqref{eq:a+b}}{\le}& \langle\nabla f(\theta^k), \widehat{\theta}^{k+1} - \theta^k \rangle + \langle\nabla f(\theta^k), \theta^{k+1} - \widehat{\theta}^{k+1} \rangle\\
        &&\quad+ L\|\widehat{\theta}^{k+1} - \theta^k\|^2 + L\|\theta^{k+1} - \widehat{\theta}^{k+1}\|^2\\
        &=& - \gamma\langle\nabla f(\theta^k), g^k\rangle + L\gamma^2\|g^k\|^2 + \langle\nabla f(\theta^k), \theta^{k+1} - \widehat{\theta}^{k+1} \rangle\\
        &&\quad + L\|\theta^{k+1} - \widehat{\theta}^{k+1}\|^2,
    \end{eqnarray*}
    from which it follows that
    \begin{eqnarray}
        \EE\left[f(\theta^{k+1}) - f(\theta^k)\mid \theta^k\right] &\le& -\gamma\left\langle\nabla f(\theta^k), \frac{1}{N_k}\sum\limits_{i\in P_k}\nabla f(\theta_i^k) \right\rangle\notag\\
        &&\quad + \EE\left[\langle\nabla f(\theta^k), \theta^{k+1} - \widehat{\theta}^{k+1} \rangle\mid \theta^k\right]\notag\\
        &&\quad + \EE\left[L\|\theta^{k+1} - \widehat{\theta}^{k+1}\|^2\mid \theta^k\right]\notag\\
        &&\quad + L\gamma^2\EE\left[\left\|\frac{1}{N_k}\sum\limits_{i\in P_k}g_i^k\right\|^2\mid \theta^k\right],\label{eq:key_lemma_non_cvx_tech_2}
    \end{eqnarray}
    where $\EE\left[\ \cdot \mid \theta^k\right] := \EE\left[\ \cdot \mid P_k, \theta_i^k, i\in P_k\right]$. Next, we estimate the last three terms in the right-hand side of \eqref{eq:key_lemma_non_cvx_tech_2}. First of all,
\begin{eqnarray}
    -\gamma\left\langle\nabla f(\theta^k), \frac{1}{N_k}\sum\limits_{i\in P_k}\nabla f(\theta_i^k)\right\rangle &=& -\gamma\|\nabla f(\theta^k)\|^2 \notag\\
    &&\quad - \gamma\left\langle\nabla f(\theta^k), \frac{1}{N_k}\sum\limits_{i\in P_k}\nabla f(\theta_i^k) - \nabla f(\theta^k)\right\rangle \notag\\
    &\overset{\eqref{eq:young_inequality}}{\le}& -\gamma\|\nabla f(\theta^k)\|^2 + \frac{\gamma}{2}\|\nabla f(\theta^k)\|^2\notag\\
    &&\quad+ \frac{\gamma}{2}\left\|\frac{1}{N_k}\sum\limits_{i\in P_k}(\nabla f(\theta_i^k) - \nabla f(\theta^k))\right\|^2\notag\\
    &\overset{\eqref{eq:jensen_ineq}}{\le}& - \frac{\gamma}{2}\|\nabla f(\theta^k)\|^2 + \frac{\gamma}{2N_k}\sum\limits_{i\in P_k}\|\nabla f(\theta_i^k) - \nabla f(\theta^k)\|^2\notag\\
    &\overset{\eqref{eq:L_smoothness_def}}{\le}& - \frac{\gamma}{2}\|\nabla f(\theta^k)\|^2 + \frac{\gamma L^2}{2}V_k, \label{eq:key_lemma_non_cvx_tech_3}
\end{eqnarray}
where $V_k = \frac{1}{N_k}\sum_{i\in P_k}\|\theta_i^k - \theta^k\|^2$. Secondly, since the stochastic gradients $\{g_i^k\}_{i\in P_k}$ are computed independently, we derive
\begin{eqnarray}
    L\gamma^2\EE\left[\left\|\frac{1}{N_k}\sum\limits_{i\in P_k}g_i^k\right\|^2\mid \theta^k\right] &\overset{\eqref{eq:variance_decomposition}}{=}& L\gamma^2\left\|\frac{1}{N_k}\sum\limits_{i\in P_k}\nabla f(\theta_i^k)\right\|^2\notag\\
    &&\quad + L\gamma^2\EE\left[\left\|\frac{1}{N_k}\sum\limits_{i\in P_k}(g_i^k-\nabla f(\theta_i^k))\right\|^2\mid \theta^k\right]\notag\\
    &\overset{\eqref{eq:jensen_ineq}}{\le}& 2L\gamma^2 \left\|\frac{1}{N_k}\sum\limits_{i\in P_k}(\nabla f(\theta_i^k)-\nabla f(\theta^k))\right\|^2 \notag\\
    &&\quad + 2L\gamma^2\|\nabla f(\theta^k)\|^2 \notag\\
    &&\quad + \frac{\gamma^2L}{N_k^2}\sum\limits_{i\in P_k}\EE\left[\|g_i^k - \nabla f(\theta_i^k)\|^2\mid \theta^k\right]\notag\\
    &\overset{\eqref{eq:jensen_ineq},\eqref{eq:bounded_variance}}{\le}& \frac{2\gamma^2L}{N_k}\sum\limits_{i\in P_k}\|\nabla f(\theta_i^k)-\nabla f(\theta^k)\|^2\notag\\
    &&\quad + 2L\gamma^2\|\nabla f(\theta^k)\|^2 + \frac{\gamma^2L\sigma^2}{N_k}\notag\\
    &\overset{\eqref{eq:L_smoothness_def}}{\le}& \underbrace{\frac{2L^3\gamma^2}{N_k}\sum\limits_{i\in P_k}\|\theta_i^k - \theta^k\|^2}_{2L^3\gamma^2 V_k} + 2L\gamma^2\|\nabla f(\theta^k)\|^2\notag\\
    &&\quad + \frac{\gamma^2L\sigma^2}{N_{\min}}. \label{eq:key_lemma_non_cvx_tech_4}
\end{eqnarray}
Plugging \eqref{eq:key_lemma_non_cvx_tech_3} and \eqref{eq:key_lemma_non_cvx_tech_4} in \eqref{eq:key_lemma_non_cvx_tech_2}, we obtain
\begin{eqnarray}
    \EE\left[f(\theta^{k+1}) - f(\theta^k)\mid \theta^k\right] &\le& -\frac{\gamma}{2}\left(1 - 4L\gamma\right)\|\nabla f(\theta^k)\|^2 + \frac{\gamma L^2}{2}\left(1 + 4L\gamma\right)V_k + \frac{L\gamma^2\sigma^2}{N_{\min}}\notag\\
    &&\quad + \EE\left[\langle\nabla f(\theta^k), \theta^{k+1} - \widehat{\theta}^{k+1} \rangle + L\|\theta^{k+1} - \widehat{\theta}^{k+1}\|^2\mid \theta^k\right].\notag
\end{eqnarray}
Next, we take the full expectation from the both sides of the above inequality, apply the tower property \eqref{eq:tower_property} and take into account that $\gamma \le \nicefrac{(1-2\delta_{pv,1})}{8L}$:
\begin{eqnarray*}
    \EE\left[f(\theta^{k+1}) - f(\theta^k)\right] &\le& -\frac{\gamma}{2}\left(1 - 4L\gamma\right)\EE\left[\|\nabla f(\theta^k)\|^2\right] + \frac{\gamma L^2}{2}\left(1 + 4L\gamma\right)\EE[V_k] + \frac{L\gamma^2\sigma^2}{N_{\min}}\\
    &&\quad + \EE\left[\langle\nabla f(\theta^k), \theta^{k+1} - \widehat{\theta}^{k+1} \rangle + L\|\theta^{k+1} - \widehat{\theta}^{k+1}\|^2\right]\\
    &\overset{\eqref{eq:key_lemma_non_cvx_tech_1}}{\le}& -\frac{\gamma}{2}\left(1 - 2\delta_{pv,1} - 4L\gamma\right)\EE\left[\|\nabla f(\theta^k)\|^2\right] + \frac{\gamma L^2}{2}\left(1 + 4L\gamma\right)\EE[V_k] \notag\\
    &&\quad + L\gamma^2\left(\frac{\sigma^2}{N_{\min}} + \delta_{pv,2}^2\right)\\
    &\le& -\frac{(1-2\delta_{pv,1})\gamma}{4}\EE\left[\|\nabla f(\theta^k)\|^2\right] + \gamma L^2 \EE[V_k]\notag\\
    &&\quad + L\gamma^2\left(\frac{\sigma^2}{N_{\min}} + \delta_{pv,2}^2\right).
\end{eqnarray*}
Summing up the obtained inequalities for $k = 0,\ldots, K-1$ and rearranging the terms, we derive
\begin{eqnarray*}
    \frac{(1-2\delta_{pv,1})\gamma}{4}\sum\limits_{k=0}^{K-1}\EE\left[\|\nabla f(\theta^k)\|^2\right] &\le& \sum\limits_{k=0}^{K-1} \EE\left[f(\theta^k) - f(\theta^{k+1})\right] + \gamma L^2\sum\limits_{k=0}^{K-1} \EE[V_k]\notag\\
    &&\quad + KL\gamma^2\left(\frac{\sigma^2}{N_{\min}} + \delta_{pv,2}^2\right)\\
    &=& f(\theta^0) - \EE[f(\theta^{K})] + \gamma L^2\sum\limits_{k=0}^{K-1} \EE[V_k] \\
    &&\quad + KL\gamma^2\left(\frac{\sigma^2}{N_{\min}} + \delta_{pv,2}^2\right)\\
    &\le& f(\theta^0) - f_* + \gamma L^2\sum\limits_{k=0}^{K-1} \EE[V_k]\\
    &&\quad + KL\gamma^2\left(\frac{\sigma^2}{N_{\min}} + \delta_{pv,2}^2\right),
\end{eqnarray*}
where $f_*$ is a uniform lower bound for $f$.
\end{proof}
The next step towards completing the proof of Theorem~\ref{thm:non_cvx_convergence} gives the upper bound for $\sum_{k=0}^{K-1} \EE[V_k]$ that appeared in \eqref{eq:key_lemma_non_cvx}.

\begin{lemma}\label{lem:V_k_lemma_non_cvx}
    Let $f_1 = \ldots = f_N = f$ be $L$-smooth and bounded from below by $f_*$, and Assumptions~\ref{as:bounded_var}~and~\ref{as:averaging_quality} hold with $\Delta_{pv}^k = \delta_{pv,1}\gamma\EE[\|\nabla f(\theta^k)\|^2] + L\gamma^2\delta_{pv,2}^2$, $\delta_{pv,1}\in [0,\nicefrac{1}{2})$, $\delta_{pv,2}\ge 0$. Then, for any $K \ge 0$ the iterates produced by Moshpit SGD with $\gamma \le \nicefrac{1}{\left(4\sqrt{e}L(\tau-1)\right)}$ satisfy
    \begin{eqnarray}
        \sum\limits_{k=0}^{K-1}\EE[V_k] &\le& 8e\gamma^2(\tau-1)^2\sum\limits_{k=0}^{K-1}\EE[\|\nabla f(\theta^k)\|^2] + 4\gamma^2K\left(2\delta_{aq}^2 + e(\tau-1)\sigma^2\right) ,\label{eq:V_k_lemma_non_cvx}
    \end{eqnarray}
    where $V_k = \frac{1}{N_k}\sum_{i\in P_k}\|\theta_i^k - \theta^k\|^2$ and $\theta^k = \frac{1}{N_k}\sum_{i\in P_k}\theta_i^k$.
\end{lemma}
\begin{proof}
    First of all, consider $k$ such that $k = a\tau + t'$ for some $t'\in [0,\tau)$. Let $\EE_g[\cdot]$ denote the expectation conditioned on $\{P_t\}_{t=a\tau}^{(a+1)\tau-1}$. Then
     \begin{eqnarray}
         \EE_g[V_k] &=& \frac{1}{N_k}\sum\limits_{i\in P_k}\EE_g\left[\|\theta_i^k - \theta^k\|^2\right] \overset{\eqref{eq:variance_decomposition}}{\le} \frac{1}{N_k}\sum\limits_{i\in P_k}\EE_g\left[\|\theta_i^k - \theta^{a\tau}\|^2\right] \notag\\
         &=& \frac{1}{N_k}\sum\limits_{i\in P_k}\EE_g\left[\left\|\theta_i^{a\tau} - \theta^{a\tau} - \gamma\sum\limits_{t=a\tau}^{k-1} g_i^t\right\|^2\right]\notag\\
         &\overset{\eqref{eq:a+b}}{\le}& \frac{2}{N_k} \sum\limits_{i\in P_k}\EE_g\left[\|\theta_i^{a\tau} - \theta^{a\tau}\|^2\right] + \frac{2\gamma^2}{N_k}\sum\limits_{i\in P_k}\EE_g\left[\left\|\sum\limits_{t=a\tau}^{k-1} g_i^t\right\|^2\right]. \label{eq:V_k_lemma_non_cvx_tech_1}
     \end{eqnarray}
     Next, we estimate the second term in the right-hand side of \eqref{eq:V_k_lemma_non_cvx_tech_1} using Lemma~\ref{lem:lemma_i_1_gorbunov}:
     \begin{eqnarray}
         \frac{2\gamma^2}{N_k}\sum\limits_{i\in P_k}\EE_g\left[\left\|\sum\limits_{t=a\tau}^{k-1} g_i^t\right\|^2\right] &\overset{\eqref{eq:lemma_i_1_gorbunov}}{\le}& \frac{2e\gamma^2(k - a\tau)}{N_k} \sum\limits_{i\in P_k} \sum\limits_{t=a\tau}^{k-1}\EE_g[\|\nabla f(\theta_i^t)\|^2]\notag\\
         &&\quad + \frac{2e\gamma^2}{N_k}\sum\limits_{i\in P_k} \sum\limits_{t=a\tau}^{k-1}\EE_g[\|g_i^t - \nabla f(\theta_i^t)\|^2]\notag\\
         &\overset{\eqref{eq:a+b},\eqref{eq:bounded_variance}}{\le}& 4e\gamma^2(\tau-1) \sum\limits_{t=a\tau}^{k-1}\EE_g[\|\nabla f(\theta^t)\|^2] \notag\\
         &&\quad+ 4e\gamma^2(\tau-1) \sum\limits_{t=a\tau}^{k-1}\frac{1}{N_k}\sum\limits_{i\in P_k}\EE_g[\|\nabla f(\theta_i^t) - \nabla f(\theta^t)\|^2] \notag\\
         &&\quad+ 2e\gamma^2 (k - a\tau)\sigma^2\notag\\
         &\overset{\eqref{eq:L_smoothness_def}}{\le}& 4e\gamma^2(\tau-1) \sum\limits_{t=a\tau}^{k-1}\EE_g[\|\nabla f(\theta^t)\|^2]\notag\\
         &&\quad + 4e\gamma^2L^2(\tau-1) \sum\limits_{t=a\tau}^{k-1}\frac{N_t}{N_k}\cdot\frac{1}{N_t}\sum\limits_{i\in P_t}\EE_g[\|\theta_i^t - \theta^t\|^2]\notag\\
         &&\quad + 2e\gamma^2(\tau-1)\sigma^2\notag\\
         &\le& 4e\gamma^2(\tau-1) \sum\limits_{t=a\tau}^{k-1}\EE_g[\|\nabla f(\theta^t)\|^2] \notag\\
         &&\quad + 8e\gamma^2L^2(\tau-1) \sum\limits_{t=a\tau}^{k-1}\EE_g[V_t] + 2e\gamma^2(\tau-1)\sigma^2,\notag
     \end{eqnarray}
     where in the last two inequalities we use $N_k = |P_k| \le |P_{k-1}| = N_{k-1}$ for all $k\ge 1$ and $N_{a\tau} \le 2 N_{(a+1)\tau}$ for all integer $a \ge 0$. Plugging this inequality in \eqref{eq:V_k_lemma_non_cvx_tech_1} and taking the full expectation from the result, we get
     \begin{eqnarray}
         \EE[V_k] &\le& 2\EE\left[\frac{1}{N_k}\sum\limits_{i\in P_k}\|\theta_i^{a\tau} - \theta^{a\tau}\|^2\right] + 4e\gamma^2(\tau-1) \sum\limits_{t=a\tau}^{k-1}\EE[\|\nabla f(\theta^t)\|^2]\notag\\
         &&\quad + 8e\gamma^2L^2(\tau-1) \sum\limits_{t=a\tau}^{k-1}\EE[V_t] + 2e\gamma^2(\tau-1)\sigma^2\notag\\
         &\le& 4\EE\left[\frac{1}{N_{a\tau}}\sum\limits_{i\in P_{a\tau}}\|\theta_i^{a\tau} - \theta^{a\tau}\|^2\right] + 4e\gamma^2(\tau-1) \sum\limits_{t=a\tau}^{k-1}\EE[\|\nabla f(\theta^t)\|^2] \notag\\
         &&\quad + 8e\gamma^2L^2(\tau-1) \sum\limits_{t=a\tau}^{k-1}\EE[V_t] + 2e\gamma^2(\tau-1)\sigma^2\notag\\
         &\overset{\eqref{eq:quality_of_avg}}{\le}& 4e\gamma^2(\tau-1) \sum\limits_{t=a\tau}^{k-1}\EE[\|\nabla f(\theta^t)\|^2] + 8e\gamma^2L^2(\tau-1) \sum\limits_{t=a\tau}^{k-1}\EE[V_t]\notag\\
         &&\quad + 2\gamma^2\left(2\delta_{aq}^2 + e(\tau-1)\sigma^2\right),\notag
     \end{eqnarray}
     where in the second inequality we also use $N_k = |P_k| \le |P_{k-1}| = N_{k-1}$ for all $k\ge 1$ and $N_{a\tau} \le 2 N_{(a+1)\tau}$ for all integer $a \ge 0$. Summing up the obtained inequalities for $k = a\tau, a\tau+1,\ldots, K'$ for some $K' \in[a\tau, (a+1)\tau-1]$ we derive
     \begin{eqnarray*}
         \sum\limits_{k=a\tau}^{K'}\EE[V_k] &\le& 4e\gamma^2(\tau-1)\sum\limits_{k=a\tau}^{K'} \sum\limits_{t=a\tau}^{k-1}\EE[\|\nabla f(\theta^t)\|^2] + 8e\gamma^2L^2(\tau-1) \sum\limits_{k=a\tau}^{K'}\sum\limits_{t=a\tau}^{k-1}\EE[V_t]\\
         &&\quad + 2\gamma^2(K'-a\tau+1)\left(2\delta_{aq}^2 + e(\tau-1)\sigma^2\right)\\
         &\le& 4e\gamma^2(\tau-1)^2\sum\limits_{k=a\tau}^{K'} \EE[\|\nabla f(\theta^k)\|^2] + 8e\gamma^2L^2(\tau-1)^2 \sum\limits_{k=a\tau}^{K'}\EE[V_k]\\
         &&\quad + 2\gamma^2(K'-a\tau+1)\left(2\delta_{aq}^2 + e(\tau-1)\sigma^2\right)\\
         &\le& 4e\gamma^2(\tau-1)^2\sum\limits_{k=a\tau}^{K'} \EE[\|\nabla f(\theta^k)\|^2] + \frac{1}{2} \sum\limits_{k=a\tau}^{K'}\EE[V_k]\notag\\
         &&\quad + 2\gamma^2(K'-a\tau+1)\left(2\delta_{aq}^2 + e(\tau-1)\sigma^2\right),
     \end{eqnarray*}
     where in the last inequality we use $\gamma \le \nicefrac{1}{\left(4\sqrt{e}L(\tau-1)\right)}$. Rearranging the terms, we get that for $K' \ge 0$
     \begin{eqnarray*}
         \sum\limits_{k=a\tau}^{K'} \EE[V_k] &\le& 8e\gamma^2(\tau-1)^2\sum\limits_{k=a\tau}^{K'}\EE[\|\nabla f(\theta^k)\|^2] + 4\gamma^2(K'-a\tau+1)\left(2\delta_{aq}^2 + e(\tau-1)\sigma^2\right),
     \end{eqnarray*}
     where $a\ge 0$ is an integer such that $a\tau \le K' \le (a+1)\tau - 1$. Summing up the obtained inequalities for $K' = \tau-1, 2\tau-1,\ldots, \tau\lfloor\nicefrac{(K-1)}{\tau}\rfloor - 1, K-1$, we derive \eqref{eq:V_k_lemma_non_cvx}.
\end{proof}

Combining Lemmas~\ref{lem:key_lemma_non_cvx}~and~\ref{lem:V_k_lemma_non_cvx}, we get the following result:
\begin{theorem}[Theorem~\ref{thm:non_cvx_convergence}]
    Let $f_1 = \ldots = f_N = f$, function $f$ be $L$-smooth and bounded from below by $f_*$, and Assumptions~\ref{as:bounded_var}~and~\ref{as:averaging_quality} hold with $\Delta_{pv}^k = \delta_{pv,1}\gamma\EE[\|\nabla f(\theta^k)\|^2] + L\gamma^2\delta_{pv,2}^2$, $\delta_{pv,1}\in [0,\nicefrac{1}{2})$, $\delta_{pv,2}\ge 0$. Then, for any $K \ge 0$ the iterates produced by Moshpit SGD with
    \begin{equation*}
        \gamma \le \min\left\{\frac{1-2\delta_{pv,1}}{8L},\frac{\sqrt{1-2\delta_{pv,1}}}{8\sqrt{e}L(\tau-1)}\right\}
    \end{equation*}
    satisfy
    \begin{eqnarray}
        \EE\left[\|\nabla f(\theta_{\text{rand}}^K)\|^2\right] &\le& \frac{8\Delta_0}{(1-2\delta_{pv,1})K\gamma} \notag\\
        &&\quad + \frac{8L\gamma}{1-2\delta_{pv,1}}\left(\frac{\sigma^2}{N_{\min}} + \delta_{pv,2}^2 + 4\gamma L\left(2\delta_{aq}^2 + e(\tau-1)\sigma^2\right)\right), \label{eq:non_cvx_bound_supp}
    \end{eqnarray}
    where $\Delta_0 = f(\theta^0) - f_*$ and $\theta_{\text{rand}}^K$ is chosen uniformly at random from $\{\theta^0,\theta^1,\ldots,\theta^{K-1}\}$. That is, Moshpit SGD achieves $\EE\left[\|\nabla f(\theta_{\text{rand}}^K)\|^2\right] \le \varepsilon^2$ after 
    \begin{eqnarray}
        \cO\Bigg(\frac{L\Delta_0}{(1-2\delta_{pv,1})^2\varepsilon^2}\Bigg[1 +(\tau-1)\sqrt{1-2\delta_{pv,1}} + \frac{\delta_{pv,2}^2 + \nicefrac{\sigma^2}{N_{\min}}}{\varepsilon^2}&\notag\\
        &\hspace{-2cm} + \frac{\sqrt{(1-2\delta_{pv,1})(\delta_{aq}^2+(\tau-1)\sigma^2)}}{\varepsilon}\Bigg]\Bigg)\label{eq:non_cvx_bound_2_supp}
    \end{eqnarray}
    iterations with
    \begin{equation*}
        \gamma = \min\left\{\frac{1-2\delta_{pv,1}}{8L},\frac{\sqrt{1-2\delta_{pv,1}}}{8\sqrt{e}L(\tau-1)}, \sqrt{\frac{\Delta_0}{LK\left(\delta_{pv,2}^2 + \nicefrac{\sigma^2}{N_{\min}}\right)}}, \sqrt[3]{\frac{\Delta_0}{4L^2\left(2\delta_{aq}^2 + e(\tau-1)\sigma^2\right)}}\right\}.
    \end{equation*}
\end{theorem}
\begin{proof}[Proof of Theorem~\ref{thm:non_cvx_convergence}]
    Plugging the result of Lemma~\ref{lem:V_k_lemma_non_cvx} in the inequality \eqref{eq:key_lemma_non_cvx} from Lemma~\ref{lem:key_lemma_non_cvx}, we obtain
    \begin{eqnarray*}
        \frac{(1-2\delta_{pv,1})\gamma}{4}\sum\limits_{k=0}^{K-1}\EE\left[\|\nabla f(\theta^k)\|^2\right] &\le& f(\theta^0) - f_* + 8e\gamma^3L^2\tau(\tau-1)\sum\limits_{k=0}^{K-1}\EE[\|\nabla f(\theta^k)\|^2] \\
        &&\quad + KL\gamma^2\left(\frac{\sigma^2}{N_{\min}} + \delta_{pv,2}^2\right)\\
        &&\quad + 4KL^2\gamma^3\left(2\delta_{aq}^2 + e(\tau-1)\sigma^2\right)\\
        &\le& f(\theta^0) - f_* + \frac{(1-2\delta_{pv,1})\gamma}{8}\sum\limits_{k=0}^{K-1}\EE\left[\|\nabla f(\theta^k)\|^2\right] \\
         &&\quad + KL\gamma^2\left(\frac{\sigma^2}{N_{\min}} + \delta_{pv,2}^2\right)\\
        &&\quad + 4KL^2\gamma^3\left(2\delta_{aq}^2 + e(\tau-1)\sigma^2\right).
    \end{eqnarray*}
    Next,
    \begin{eqnarray*}
        \frac{1}{K}\sum\limits_{k=0}^K\EE\left[\|\nabla f(\theta^k)\|^2\right] &\le& \frac{8\Delta_0}{(1-2\delta_{pv,1})K\gamma} \\
        &&\quad + \frac{8L\gamma}{1-2\delta_{pv,1}}\left(\frac{\sigma^2}{N_{\min}} + \delta_{pv,2}^2 + 4\gamma L\left(2\delta_{aq}^2 + e(\tau-1)\sigma^2\right)\right),
    \end{eqnarray*}
    where $\Delta_0 = f(\theta^0) - f_*$. Since $\theta_{\text{rand}}^K$ is chosen uniformly at random from $\{\theta^0,\theta^1,\ldots,\theta^{K-1}\}$,
    \begin{equation*}
        \EE\left[\|\nabla f(\theta_{\text{rand}}^K)\|^2\right] \overset{\eqref{eq:tower_property}}{=} \frac{1}{K}\sum\limits_{k=0}^K\EE\left[\|\nabla f(\theta^k)\|^2\right]
    \end{equation*}
    and \eqref{eq:non_cvx_bound_supp} holds. Applying Lemma~\ref{lem:lemma_i_3_gorbunov} to \eqref{eq:non_cvx_bound_supp}, we get the following result: if
    \begin{equation*}
        \gamma = \min\left\{\frac{1-2\delta_{pv,1}}{8L},\frac{\sqrt{1-2\delta_{pv,1}}}{8\sqrt{e}L(\tau-1)}, \sqrt{\frac{\Delta_0}{LK\left(\delta_{pv,2}^2 + \nicefrac{\sigma^2}{N_{\min}}\right)}}, \sqrt[3]{\frac{\Delta_0}{4L^2\left(2\delta_{aq}^2 + e(\tau-1)\sigma^2\right)}}\right\},
    \end{equation*}
    then $\EE\left[\|\nabla f(\theta_{\text{rand}}^K)\|^2\right]$ equals
    \begin{equation*}
        \cO\!\left(\!\frac{L\Delta_0\left(1\!+\! (\tau\!-\!1)\sqrt{1\!-\!2\delta_{pv,1}}\right)}{(1\!-\!2\delta_{pv,1})^2K} + \sqrt{\frac{L\Delta_0\left(\delta_{pv,2}^2\! +\! \nicefrac{\sigma^2}{N_{\min}}\right)}{(1\!-\!2\delta_{pv,1})^2K}} + \frac{\sqrt[3]{L^2\Delta_0^2(\delta_{aq}^2\! +\! (\tau\!-\!1)\sigma^2)}}{(1\!-\!2\delta_{pv,1})K^{\nicefrac{2}{3}}}\!\right)\!,
    \end{equation*}
    which implies the desired convergence result from \eqref{eq:non_cvx_bound_2_supp}.
\end{proof}

\section{Decentralized matchmaking}
\label{sect:matchmaking}

In order to run group all-reduce over unreliable devices, Moshpit Averaging must be able to dynamically form groups of active devices that share the same key $C_i$.
In theory, this matchmaking can be implemented precisely as described in Algorithm~\ref{alg:moshpit}: each peer adds itself to a certain DHT key, waits for a said period of time, and then reads the same key to retrieve a list of its groupmates.

However, in practice, this kind of matchmaking would be extremely fragile: if any peer arrives late (for example, due to latency), it may join the group when other peers have already finished matchmaking. As a result, some workers will treat this peer as active, while others will behave as though there is no such peer at all, breaking the consensus and rendering all peers unable to run all-reduce in a stable manner.

To avoid this and other similar inconsistencies, Moshpit All-Reduce employs a more sophisticated matchmaking protocol with the following guarantees 
\begin{enumerate}
    \item Peers that join the same group are guaranteed to have the same list of groupmates;
    \item The group will have the maximum possible number of peers, unless some of them fail;
    \item If some peers fail, matchmaking will still form the group out of the remaining ones.
\end{enumerate}

To achieve this, each peer first declares itself onto the DHT (as in Algorithm~\ref{alg:moshpit}). Then, peers attempt to form groups by calling the \texttt{REQUEST\_JOIN\_GROUP} remote procedure call. Intuitively, if peer A calls this RPC on peer B, then \textit{peer A requests to join peer B's group}, which can be either accepted or rejected by the group ``leader'' B, which may or may not have other ``followers''.

If a peer is accepted to a group, it commits to stay active (i.e. to await other peers) for a set period of time and perform all-reduce with the peers supplied by the group ``leader''. On the other hand, a peer can be rejected if (a) the potential ``leader'' is already a follower in another group, (b) the group is already running all-reduce, or (c) if the ``leader'' failed or left during matchmaking.

To ensure that this protocol forms groups of maximum size, each peer generates a unique ``priority'' based on its local timestamp\footnote{More specifically, the priority is a tuple of $\texttt{(timestamp, peer\_id)}$, where \texttt{peer\_id} is used to break ties.}. Peers prioritize joining the group of neighbors that have the lowest ``priority''. Under normal circumstances, all workers will join the group of a peer that was first to start matchmaking according to its own local time. However, if this peer has failed or already finished matchmaking, the group will be formed around one of the remaining peers.

Matchmaking for 64 peers can take less than 1 second if all workers are located in the same cloud region and are highly synchronized. However, this can grow to 2.9 seconds for two different cloud regions and up to 9 seconds when training with commodity hardware around the world.

To ensure that this latency does not affect the training performance, Moshpit SGD performs matchmaking asynchronously in the background thread, while the model is accumulating gradients. All peers begin matchmaking 15 seconds before the estimated averaging round, so that in $\ge 95\%$ of averaging iterations, the matchmaking step is already finished by the time peers need to run all-reduce.
\section{Training with a dynamic number of peers}
\label{sect:load_state_from_peers}

Many practical setups with unreliable devices allow peers to join or leave at any time, which can produce undesirable side-effects. For instance, consider a participant that joins the ``swarm'' midway through the training process. If this participant starts with the initial model parameters, it can undo some of the progress made by other peers.

To circumvent this issue, we require each new participant to download the latest parameters from a random up-to-date peer discovered through DHT. The same technique is used to synchronize the optimizer statistics and the learning rate schedule. This protocol is also triggered if a peer becomes desynchronized with others, e.g., after a network freeze.

\section{Load balancing via linear programming}
\label{sect:load_balancing}

When running Moshpit Averaging on heterogeneous devices, one must regularly perform Butterfly All-Reduce among peers with uneven network bandwidth.
In order to speed up the protocol, we can make low-throughput peers receive, average, and send smaller partitions of the averaged vector; conversely, the high-throughput peers can process greater fractions of the input vector.
To compute the optimal partitioning, peers must solve an optimization problem that minimizes the total time spent on communication during all-reduce.

Consider a group of $M$ peers with network bandwidths $b_1, ..., b_M$, defined for simplicity as the minimum of the upload and download speed for each peer. Our objective is to find $w_i$ --- a fraction of all input vectors to be processed by the $i$-th peer.

In Butterfly All-Reduce, each peer $i$ splits its vector into parts and sends these parts to corresponding peers. Since there is no need to send $w_i$ to itself, $i$-th peer will upload a total of $1 - w_i$ of the vector to its peers.
On the receiving side, peer $i$ will average $w_i$ of the vector from all peers in its group. To do so, it must download $M-1$ vector parts of size $w_i$ from all other peers.
After that, peers distribute the averaged parts by running the same procedure in reverse (see Figure~\ref{fig:butterfly_allreduce}).

Thus, the communication time for each peer is proportional to $t_i = (1-w_i+(M-1) w_i) \cdot \frac{1}{b_i}$ and the total runtime of Butterfly All-Reduce is the maximum communication time over all peers: $T = \max_i t_i=\max_i (1-w_i+(M-1) w_i) \cdot \frac{1}{b_i}$. Formally, we minimize $T$ with respect to $w_i$ with two constraints on the fraction weights:
\begin{alignat*}{3}
\min_w&\quad &\max_i &(1-w_i +&(M-1)w_i)\cdot\frac{1}{b_i}&\\
\text{subject to}&\quad& \sum_{i=1}^M w_i = 1&&&\\
&&w_i \geq 0 &&&\forall i=1,\ldots,M
\end{alignat*}

Because the functions being maximized and the constraints are linear in $w_i$, this problem can be reduced to linear programming~\cite{kaplan1974application}. Namely, we can minimize a surrogate variable $\xi$ such that $\forall i, \ \xi \geq (1-w_i+(M-1)\cdot w_i) \cdot \frac{1}{b_i}$. The resulting linear program is formulated as follows:

\begin{alignat*}{3}
\min_{w,\xi}&\quad& \xi && &\\
\text{subject to}&\quad& \sum_{i=1}^M w_i& = 1 &&\\
&\quad& w_i& \geq 0 &&\quad \forall i=1,\ldots,M\\
&\quad&\xi&\geq (1-&w_i+(M-1)w_i)\cdot\frac{1}{b_i}&\quad\forall i=1,\ldots,M
\end{alignat*}

We solve this problem using the interior point method~\cite{andersen} implemented as part of the SciPy package (\texttt{scipy.optimize.linprog}).
Note that depending on the conditions given by participant bandwidth, optimal weights of specific peers might be equal to 0 in some cases. In essence, this allows our method to smoothly interpolate between data parallelism~\cite{valiant1990bridging}, parameter server~\cite{parameter_server_first} and sharded parameter server~\cite{sharded_ps_first} in manner similar to BytePS~\cite{byteps}.

\section{Detailed experimental setup}
\label{sect:detailed_setup}

In this section, we provide the detailed hardware configuration of servers used for each of our distributed training experiments.

\subsection{ImageNet training}\label{sect:detailed_setup_resnet}

Both homogeneous and heterogeneous training setups for ImageNet are provisioned in our on-premise infrastructure across multiple data centers and an office space (for the heterogeneous setup only).

\paragraph{Homogeneous.}For the homogeneous setup, we use 16 identical instances with the following specifications:
\begin{itemize}
    \item \textbf{GPU:} V100-PCIe,
    \item \textbf{CPU:} 6 vCPUs (Xeon E5-2650v4),
    \item \textbf{RAM:} 64GB.
\end{itemize}

\paragraph{Heterogeneous.}In turn, the heterogeneous setup contains multiple instance types listed in Table~\ref{fig:tab_setup_resnet}:
\begin{table}[h]
\centering
\caption{\textbf{Heterogeneous} setup for ImageNet training.}
\label{fig:tab_setup_resnet}
\renewcommand{\arraystretch}{1}
\begin{tabular}{@{}cccccc@{}}
\toprule
Instances & GPUs & GPU type & Cores & RAM, GB & CPU type \\ 
\midrule
4            & 1      & V100-PCIe  & 6        & 64     & E5-2650v4 \\
17           & 2      & GTX 1080Ti & 8        & 64     & E5-2650v4 \\
7            & 1      & GTX 1080Ti & 4        & 32     & E5-2650v4 \\
16           & 1      & P40  & 4        & 32     & E5-2667v2 \\
20           & 1      & M40-24GB  & 4        & 32     & E5-2667v2 \\

\bottomrule
\end{tabular}
\end{table}




\subsection{ALBERT training}\label{sect:detailed_setup_albert}


\paragraph{Homogeneous.}For the homogeneous setup, we use a single virtual machine with the following specifications:
\begin{itemize}
    \item \textbf{GPU:} $8{\times}$ V100-PCIe,
    \item \textbf{CPU:} 48 vCPUs (Xeon E5-2650v4),
    \item \textbf{RAM:} 488GB.
\end{itemize}

At the time of writing, the cloud rent cost for this instance is \textbf{\$24.48} per hour.

\paragraph{Heterogeneous.}Our heterogeneous setup is composed of two parts: AWS EC2 Spot instances and crowdsourced machines from the \texttt{Vast.ai} marketplace. For spot instances, we picked the smallest suitable instance size available from the cloud provider and further limited their bandwidth to 1Gb/s\footnote{We use \texttt{tc qdisc} Linux utility to artificially limit the network throughput, similarly to~\cite{MLSYS2019_d09bf415}}. As for marketplace instances, we report the hardware specifications for each worker gathered 1 hour after the start of ALBERT training.

Since both cloud and marketplace instances are preemptible, the actual cost of the server fleet will vary based on the current price. For simplicity, we report the maximum hourly price we ended up paying for this instance (enforced via maximum bid). Finally, some marketplace instances have missing specifications, such as unknown CPU type. This is likely caused by non-standard virtualization configured by the device owner. The resulting fleet configuration, shown in Table~\ref{fig:tab_setup}, costs up to \$15.43/hour, depending on the number of active instances.

\begin{table*}[ht!]
\centering
\caption{\textbf{Heterogeneous} setup for ALBERT training.}
\label{fig:tab_setup}
\small
\setlength{\tabcolsep}{2pt}
\hspace{7pt}\begin{tabular}{@{}ccccccc@{}}
\toprule
GPU           & Cores & RAM, GB & CPU type                       & Download, Mb/s & Upload, Mb/s &
Cost, \$/hour \\ 
\midrule
\multicolumn{7}{c}{Preemptible \texttt{g4dn.xlarge} instances ($32{\times}$)} \\
\midrule
T4            & 4         & 16     & Xeon Platinum 8259CL           & 1000          & 1000        & 0.1578         \\

\midrule
\multicolumn{7}{c}{Marketplace instances} \\    
\midrule
GTX 1070Ti    & 6         & 16     & E5-2640                        & 425           & 255         & 0.036         \\
GTX 1070Ti    & 6         & 16     & i3-6100T                       & 121           & 36          & 0.06          \\
GTX 1080Ti    & 4         & 20     & i3-6096P                       & 817           & 308         & 0.101         \\
GTX 1080Ti    & 20        & 129    & E5-2630v4                      & 660           & 475         & 0.182         \\
GTX 1080Ti    & 1         & 16     & i7-7700K                       & 245           & 210         & 0.302         \\
GTX 1080Ti    & 48        & 97     & Xeon Platinum 8124             & 583           & 539         & 0.217         \\
GTX 1080Ti    & 10        & 16     & Unknown                        & n/a           & n/a           & 0.15          \\
GTX 1080Ti    & 4         & 16     & Xeon Gold 6149                 & 98            & 100         & 0.2           \\ %
GTX 1080Ti    & 4         & 16     & Xeon Gold 6149                 & 99            & 98          & 0.2           \\ %
GTX 1080Ti    & 4         & 16     & Xeon Gold 6149                 & 99            & 99          & 0.2           \\ %
GTX 1080Ti    & 4         & 16     & Xeon Gold 6149                 & 99            & 99          & 0.2           \\ %
RTX 2070S     & 24        & 32     & E5-2620v2                      & 199           & 25          & 0.199         \\
RTX 2070S     & 32        & 97     & E5-2650                        & 162           & 64          & 0.285         \\
RTX 2080      & 6         & 16     & E5-2620v3                      & 271           & 287         & 0.25          \\
RTX 2080      & 24        & 32     & E5-2630v3                      & 199           & 25          & 0.302         \\
RTX 2080S     & 4         & 32     & E5-2697v4                      & 101           & 99          & 0.292         \\ %
RTX 2080S     & 4         & 32     & E5-2697v4                      & 93            & 99          & 0.292         \\ %
RTX 2080S     & 4         & 32     & E5-2697v4                      & 94            & 98          & 0.292         \\ %
RTX 2080S     & 4         & 32     & E5-2697v4                      & 94            & 98          & 0.292         \\ %
RTX 2080S     & 4         & 32     & E5-2697v4                      & 100           & 99          & 0.292         \\ %
RTX 2080Ti   & 4         & 16     & Ryzen Threadripper 3960x       & 279           & 271          & 0.35          \\
RTX 2080Ti   & 8         & 129    & E5-2670v3                      & 616           & 672          & 0.201         \\
RTX 2080Ti   & 6         & 32     & E5-2620v3                      & 217           & 61           & 0.22          \\
RTX 2080Ti   & 8         & 16     & E5-2697v2                      & 100           & 58           & 0.3           \\
RTX 2080Ti   & 8         & 21     & E5-2697v2                      & 145           & 49           & 0.243         \\
RTX 2080Ti    & 12        & 32     & Unknown                        & 111          & 92          & 0.326         \\
RTX 2080Ti    & 12        & 64     & E5-2690v3                      & 205          & 61          & 0.549         \\
RTX 3080      & 16        & 16     & i7-10700K                      & 69           & 49          & 0.462         \\
RTX 3090      & 14        & 32     & E5-2695v3                      & 93           & 37          & 0.498         \\
RTX 3090      & 16        & 32     & Ryzen 9 3950X                  & 338          & 38          & 0.511         \\
Titan RTX     & 4         & 32     & Xeon W-3223                   & 321           & 115          & 1             \\
Titan RTX     & 4         & 32     & Xeon Gold 6149                 & 99           & 100         & 0.702         \\ %
Titan V       & 8         & 32     & i7-7700K                       & 97           & 50          & 0.282         \\
V100-FHHL     & 8         & 60     & Xeon Gold 6148                 & 544          & 584         & 0.39          \\
\midrule
\multicolumn{6}{c}{Total hourly cost (as listed):} &\bf 15.43 \\    
\bottomrule
\end{tabular}
\end{table*}


\section{Additional averaging experiments}
\label{sect:extra_averaging}

In this section, we evaluate the averaging precision with the same methodology as in~\ref{sect:experiments_averaging}, but for multiple different worker configurations. 

Table~\ref{tab:full_averaging} provides the complete results of our experiments that were used to make conclusions in the main experimental section: instead of reporting the mean squared error for different iterations, we provide the number of rounds that was required to achieve the error of $10^{-9}$ and $10^{-4}$.

In Figure~\ref{fig:many_averagings}, plots 1--5 explore several combinations of grid sizes and failure rates, whereas plot 6 (bottom right) demonstrates a setup with the same number of peers ($10^6$) arranged into several different grid sizes and its relation to convergence. Note that $M{=}32$ outperforms the alternatives only for the specific failure rate of $0.001$.

\begin{table}[ht]
\centering
\caption{Averaging performance of different algorithms. Values denote the number of iterations required to achieve the error of $10^{-9}$ ($10^{-4}$ in parentheses), the best result is in bold.}
\vspace{1em}
\label{tab:full_averaging}
\begin{tabular}{@{}llccccc@{}}
\toprule
$N$  & $p$   & All-Reduce  & Gossip      & PushSum     & Random groups & Moshpit   \\ \midrule
512  & 0     & \bf 1.0 (1.0)   & 50.0 (50.0) & 47.6 (15.6) & 6.1 (3.0)     & 8.2 (3.5) \\
512  & 0.001 & \bf 1.6 (1.6)   & 50.0 (50.0) & 47.6 (15.6) & 6.3 (3.0)     & 8.1 (3.7) \\
512  & 0.005 & 10.9 (10.9) & 50.0 (50.0) & 47.8 (15.6) & \bf 6.3 (3.0)     & 8.7 (3.9) \\
512  & 0.01  & 41.7 (41.7) & 50.0 (50.0) & 47.8 (15.6) & \bf 6.6 (3.0)     & 9.1 (3.9) \\ \midrule
768  & 0     & \bf 1.0 (1.0)   & 50.0 (50.0) & 43.2 (13.8) & 6.2 (3.0)     & 6.0 (3.0) \\
768  & 0.001 & \bf 1.8 (1.8)   & 50.0 (50.0) & 43.2 (13.8) & 6.5 (3.0)     & 6.2 (3.0) \\
768  & 0.005 & 28.7 (28.7) & 50.0 (50.0) & 43.2 (14.1) & \bf 6.6 (3.0)     & \bf 6.6 (3.0) \\
768  & 0.01  & 50.0 (50.0) & 50.0 (50.0) & 43.9 (14.2) & 7.0 (3.0)     & \bf 6.8 (3.0) \\ \midrule
900  & 0     & \bf 1.0 (1.0)   & 50.0 (50.0) & 45.0 (14.7) & 6.4 (3.0)     & 5.0 (2.8) \\
900  & 0.001 & \bf 1.8 (1.8)   & 50.0 (50.0) & 45.0 (14.7) & 6.3 (3.0)     & 5.5 (3.0) \\
900  & 0.005 & 50.0 (50.0) & 50.0 (50.0) & 45.2 (14.7) & 6.7 (3.0)     &\bf  5.9 (3.0) \\
900  & 0.01  & 50.0 (50.0) & 50.0 (50.0) & 45.6 (14.9) & 7.0 (3.1)     & \bf 6.4 (3.1) \\ \midrule
1024 & 0     & \bf 1.0 (1.0)   & 50.0 (50.0) & 49.0 (16.2) & 6.2 (3.0)     & 2.0 (2.0) \\
1024 & 0.001 & \bf 2.0 (2.0)   & 50.0 (50.0) & 49.0 (16.3) & 6.5 (3.0)     & 3.4 (2.2) \\
1024 & 0.005 & 42.6 (42.6) & 50.0 (50.0) & 49.5 (16.3) & 6.7 (3.0)     & \bf 5.4 (2.9) \\
1024 & 0.01  & 50.0 (50.0) & 50.0 (50.0) & 49.5 (16.3) & 6.9 (3.1)     & \bf 5.9 (3.0) \\ \bottomrule
\end{tabular}
\end{table}

\begin{figure}[h]
    \centering
    \includegraphics[width=\linewidth]{resources/multiple_graphics.pdf}
    \vspace{-20pt}
    \caption{Averaging error of Moshpit All-Reduce as a function of the iteration number for different configurations and failure rates.}
    \label{fig:many_averagings}
\end{figure}

\section{Additional image classification experiments}
\label{sect:extra_classification}

Aside from the two evaluation scenarios provided in~\ref{sect:experiments_vision}, we also measure the performance of Moshpit-SGD in a non-distributed setup, i.e. on a single server with multiple GPUs. We conduct this experiment on the same $8{\times}$ V100 machine that was used in the \textbf{homogeneous} setup for training ALBERT (see Appendix~\ref{sect:detailed_setup_albert}).

\begin{figure}[h]
    \centering
    \begin{tabular}{cc}
    \hspace{-10pt}
        \includegraphics[width=0.5\textwidth]{resources/resnet50_local.pdf} &
        \includegraphics[width=0.5\textwidth]{resources/resnet50_local_epochs.pdf}
    \end{tabular}
    \caption{
    ResNet-50 top-1 validation accuracy on ImageNet when training on a single node with $8{\times}$ V100-PCIe GPUs.
    \textbf{(Left)} Convergence in terms of training time, \textbf{(Right)} Convergence in terms of training epochs}
    \label{fig:resnet_local}\vspace{-8pt}
\end{figure}

As Figure~\ref{fig:resnet_local} demonstrates, Moshpit SGD is slower than AR-SGD by approximately $25\%$. This result is expected, since our implementation of Moshpit All-Reduce is more general and communicates over a TCP connection, whereas AR-SGD uses direct peer-to-peer GPU communication over PCIe. On average, this incurs a slowdown of $27\%$ in terms of training time.

\end{document}